\chapter*{Введение} % * не проставляет номер
\addcontentsline{toc}{chapter}{Введение} % вносим в содержание

Работа обычного учителя включает в себя не только составление программы для детей и их непосредственное обучение, но и проверку всех написанных ими работ в том числе. Как правило, этот процесс происходит независимо от рабочего времени. Что заставляет тратить несколько часов личного свободного времени.

В современном мире стало довольно популярно оцифровывать механизмы обучения. Сейчас вы можете встретить такие инструменты, как автоматическая проверка тестовой части единого государственного экзамена, онлайн тесты на дистанционных ресурсах. Но по-прежнему процесс автоматической проверки рукописных текстов не является частью школьного и дошкольного образований.

А сейчас появляются тенденции проведения дистанционного обучения, где каждая работа оцифровывается перед тем, как попасть в руки преподавателя. Соответственно, такую работу сложнее проверять и анализировать, так как делать пометки в таком формате работы становится затруднительным. И это помимо того, что проверка уже занимает много личного времени.

Выходом из этой ситуации является программа автоматической проверки рукописных работ за счет анализа цифровых изображений методом нейронных сетей.

Очевидно, что такой программный продукт является актуальным в наше время, поскольку значительно упрощает ручную проверку работ.

\textbf{Целью данной работы} является исследование и анализ существующих алгоритмов, позволяющих считывать текст, написанный от руки, в формате цифрового изображения. И для достижения данной цели необходимо выполнить следующие \textbf{задачи}:
\begin{itemize}
	\item Исследовать определенные типы нейронных сетей;
	\item Исследовать способы разбиения текста на изображении;
	\item Исследовать готовые алгоритмы считывания текста;
	\item Вывод по анализу улучшения существующих алгоритмов.
\end{itemize}






%% Вспомогательные команды - Additional commands
%\newpage % принудительное начало с новой страницы, использовать только в конце раздела
%\clearpage % осуществляется пакетом <<placeins>> в пределах секций
%\newpage\leavevmode\thispagestyle{empty}\newpage % 100 % начало новой строки