\chapter{Распознавание} \label{ch3}

% не рекомендуется использовать отдельную section <<введение>> после лета 2020 года
%\section{Введение} \label{ch3:intro}

Данный этап является самым сложным среди остальных, так как это и есть стадия распознавания рукописных символов. 
	
\section{Нормализация текста} \label{ch3:sec1}

На данном этапе на входе алгоритма организовываются отдельные изображения слов или объектов слов. И перед тем, как начинать распознавать эти изображения, сперва лучше их привести к виду, наиболее пригодному для этой самой обработки. Потому что из-за многочисленных вариантов написания, начертаний, слова могут быть написаны под различными углами, что затрудняет способы распознавания. И задача состоит в том, чтобы привести их к виду, более-менее параллельному горизонтальному направлению, причем не теряя данных, то есть изображение не должно исковеркаться или как-то сжаться после этапа нормализации.

Для реализации существует несколько подходов и многие из них ориентируются, собственно, на углы наклона текста относительно некоторых прямых, а именно угол наклона относительно горизонтальной прямой и угол наклона относительно вертикальной. Но второй вариант подходит для тех элементов, что должны быть вертикальными.

На самом деле задача состоит в выборе правильного наклона и, возможно, перспективы, и такие средства уже давно реализованы в той же программной среде Photoshop.

Основная проблема заключается в определении угла, при котором текст станет выровненным по горизонтали. Для ее решения можно использовать горизонтальные профили. В таком случае нужно искать максимум профиля при изменении угла, тогда слово максимально выравнено. Если оно каким-то образом искривлено или наклонено, тогда профиль будет распределен, что не является результативным свойством.

Но также важен факт того, что потеря информации должна быть сведена к минимуму. И, какой способ бы не использовался, на изображении могут «исчезнуть» некоторые пиксели, из-за чего компоненты связности могут нарушаться и сами буквы или целые слова затем труднее будет распознавать. Для исправления такой ситуации можно использовать анализ исходного варианта, и добавление новых пикселей, а затем сглаживание, к примеру билинейное. Или можно сразу применить сглаживание, но тогда результат все равно может быть несколько иным.


\section{Распознавание слов} \label{ch3:sec2}

Есть два основных подхода для реализации распознавания слов: сначала разбить на отдельные символы, распознать их, а затем склеить в слово, и анализировать слово целиком.

Первый и второй подходы могут работать вместе, когда сначала слово разбивается на символы поочередно, начиная с первой, затем с каждой последующей буквой формируется наиболее вероятное слово. В таком случае избегается возможность ошибки при распознавании. Но поговорим про каждый способ сначала в отдельности.

Задача разбить слово на символы является сама по себе сложной. Тем более если речь идет о рукописном тексте. И на данный момент нет эффективного решения. Интервалы между буквами не всегда очевидны, в особенности если они связаны между собой, а не пишутся раздельно. Для последнего случая, отметим, есть те же самые диаграммы Вороного. Но в общем случае разбить на символы может быть и вовсе невыполнимо, тогда можно разбивать на определенные участки. Можно применить слабо работающий, но в перспективе улучшенный способ в две стадии. На первой стадии сперва нужно разбить на вероятные интервалы. Для этого составляются вертикальные профили изображения и ищутся локальные минимумы. Тут во внимание нужно взять величину отношения высоты символа к его ширине. В среднем эта величина равна 0,3. То есть мы можем определить интервалы поиска локальных минимумов, посмотрев высоту слова, и подсчитав среднюю ширину символа для конкретного слова. Так же следует учитывать среднюю яркость целого участка. Вторая стадия исключает ложные разделения и определяет явные возможные варианты.

Также можно разбить на достаточно малые интервалы слово, и идти вдоль изображения, разбирая каждый участок. Составляя вероятные карты можно попробовать определить, какие символы встречаются, отсекать их, и искать дальше.

Хорошим вариантом является применение нейронной сети. Причем можно предварительно поделить изображение на участки равной длины исходя из соображения подсчитываемой ширины символа. И тогда использовать сеть по типу персептрона. Если вероятности для каждого класса примерно равные, значит сеть не смогла точно определить символы, можно снова сегментировать изображение. Либо использовать сверточную сеть, которую наиболее часто используют в таких задачах, потому что она как раз создана для распознавания и определения видимых признаков. Тогда так же проходить по изображению вдоль горизонтального направления и определять классы встречаемых участков. Или же делить на более мелкие участки, и использовать многоуровневую сверточную нейронную сеть.

Если говорить о подходе определения слова целиком, то сперва лучше определить примерный контекст текста, чтобы завести базу данных возможных слов. Или же использовать полную БД всех слов. Затем задействовать для алгоритма обученную нейронную сеть, которая будет получать возможные варианты слова. Сеть не обязана получить сразу верное значение. Сохраняются вероятности для каждого случая.

Алгоритм последовательного считывания символов предполагает, что каждый последующий будет определяться на основе составляющегося слова. Предварительно можно построить префиксные деревья. Тогда определив текущий символ, по дереву достаточно быстро и просто определяются возможные следующие варианты символов. Таким образом сразу исключаются возможные ошибки или же можно определить, является ли следующий символ ошибкой в исходном слове. Но тогда алгоритм ломается после такого случая. Если же не учитывать параллельную обработку ошибок, тогда следующий символ ищется заново с корректировками. 

\section{Улучшение результатов}  \label{ch3:sec3}

На этапе использования нейронной сети для определения слова целиком был получен массив вероятностей подходящих вариантов для каждого слова. Достаточно легко определить границы предложений по знакам, которые не так сложно ищутся. Тогда, используя семантику языка, возможно определить наиболее подходящий вариант следующего после текущего слова. И, разумеется, обрабатывая одновременно этот фактор и фактор массива вероятностей, получается точный результат. 

Если распознавание шло по сегментации слова на символы или последовательности символов, то нужно использовать словарь, именно печатный словарь слов. По нему искать наиболее подходящее слово и заменять результат на него. В случае нескольких возможных совпадений использовать предыдущий подход.








%% Вспомогательные команды - Additional commands
%
%\newpage % принудительное начало с новой страницы, использовать только в конце раздела
%\clearpage % осуществляется пакетом <<placeins>> в пределах секций
%\newpage\leavevmode\thispagestyle{empty}\newpage % 100 % начало новой страницы