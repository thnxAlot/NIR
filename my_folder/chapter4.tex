\chapter{Нейронные сети} \label{ch4}

% не рекомендуется использовать отдельную section <<введение>> после лета 2020 года
%\section{Введение} \label{ch4:intro}

Нейронные сети активно развиваются в современном мире. В крупных проектах, где есть возможность делиться информацией, таких как Instagram, Facebook, Вконтакте давно уже начали использовать нейронные сети для распознавания образов. Такие типы сетей позволяют определить и квалифицировать объект, как правило, на изображении. Они позволяют своего рода придать зрение компьютеру или другому вычислительному устройству, чтобы тот смог увидеть и определить, что ему показывают. При распознавании текста на изображениях будет использоваться тот же механизм, ведь необходимо увидеть и прочитать текст, который представляет из себя пиксели на картинке. Из используемых далее в алгоритмах сетей выделяются многоуровневые персептроны и сверточные нейронные сети. \cite{neiron1}
	
\section{Персептрон} \label{ch4:sec1}
Такая нейронная сеть представляет из себя граф, где есть начальные узлы - входные данные, и есть конечные узлы - выходные данные (классы). Они соединены между собой ребрами, которые в свою очередь имеют веса. Эти веса изначально неизвестны. Это был одноуровневый персептрон, но в общем случае мы можем добавить между ними дополнительные уровни из узлов, так называемые скрытые слои. И каждый слой связан с предыдущим наборов ребер. Первая задача состоит в том, чтобы научить такую сеть классифицировать объект, причем необязательно как бы зрительно. То есть нужно определить изначальные веса для каждого ребра, чтобы подавая на входные узлы графа, получить узел, который будет показывать наиболее вероятный класс такого объекта. И нам нужно говорить, правильный был ответ сети, или же нет, указывая на верный вариант. Впоследствии она научится сама уже квалифицировать объекты. Такой метод еще называется методом обратного распространения ошибки.

В нашем случае, допустим, возьмем символ, который представлен матрицей пикселей, где ячейка, принадлежащая символу, отмечена единицей, или близким к ней значением, а фон нулем. Тогда подав на вход многоуровнего персептрона эту матрицу, мы должны, после обучения, разумеется, получить наиболее вероятный символ.  \cite{cit1}


\section{Сверточная нейронная сеть} \label{ch4:sec2}

Сверточная нейронная сеть устроена несколько сложнее, чем предыдущий вариант, но более тесно связана с распознаванием образов, так как обладает специальным механизмом, увеличивающим эффективность и масштабирование в отличие от персептрона. Задача стоит в том, чтобы от мелких деталей, которые могут содержаться на изображении, к примеру линии, кривые, переходить к более сложным, что позволит распознать достаточно сложный объект, сложнее, чем символ. Такая сеть проходит по изображению фильтрами, представляющие из себя матрицы, и которые будут сигнализировать о том, что в этом участке, где в данный момент находится фильтр, имеется та деталь, которую этот фильтр ищет, к примеру прямую под определенным углом. Затем они сворачиваются в матрицы, к которым снова применяются фильтры. Матрицы фильтров задаются посредством обучения сети. Затем на выходе мы получаем определенный класс принадлежности. \cite{neiron2}

%\FloatBarrier % заставить рисунки и другие подвижные (float) элементы остановиться

\section{Выводы} \label{ch4:conclusion}

Изучив эти два типа нейронных сетей, можно сделать вывод о том, что для более мелких и простых случаев лучше использовать персептрон, но если же нужно найти на изображении некоторый объект или определить более сложный предмет, то лучше использовать сверточную нейронную сеть. 

%% Вспомогательные команды - Additional commands
%
%\newpage % принудительное начало с новой страницы, использовать только в конце раздела
%\clearpage % осуществляется пакетом <<placeins>> в пределах секций
%\newpage\leavevmode\thispagestyle{empty}\newpage % 100 % начало новой страницы