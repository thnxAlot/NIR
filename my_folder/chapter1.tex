\chapter*{Предисловие} %\label{ch1}

% не рекомендуется использовать отдельную section <<введение>> после лета 2020 года
%\section{Введение. Сложносоставное название первого параграфа первой главы для~демонстрации переноса слов в содержании} \label{ch1:intro}
\addcontentsline{toc}{chapter}{Предисловие}

Все изображения, о которых далее идет речь, хранятся в виде растровых изображений на твердотельном накопителе, к примеру, диск компьютера. Такие файлы хранят информацию лишь о цвете каждого отдельного пикселя в матрице изображения. Процесс распознавания текста – это получение каждого отдельного символа в виде его кода в том или ином текстовом формате.

На данный момент существует множество решений для распознавания машинописных и рукопечатных текстов на изображениях. Есть даже готовые программные продукты, такие как FineReader, которые неплохо справляются со своей задачей. Для первого варианта задача является готовой, но для второго она до сих пор полностью не решена, так как является существенно затруднительной по сравнению с предыдущей.

Задача распознавания рукописного текста (handwriting recognition) имеет два типа подхода:
\begin{itemize}
	\item 	Онлайн распознавание – анализ текста сразу при его написании;
	\item Оффлайн распознавание - анализ уже готового написанного текста на изображении.
\end{itemize}

В первом случае программа должна считывать символ сразу при его написании, что гораздо упрощает работу алгоритма, поскольку не нужно распознавать целое изображение и искать в нем текст, выполнять дальнейшие шаги обработки. Текст распознается с большей точностью из-за особенности написания (символы обычно более строго отделяются друг от друга). И алгоритм достаточно прост. Сперва нужно распознать введенный на текущем этапе ввода символ. Сделать это можно, используя нейронные сети. Такая технология позволит получить вероятности того или иного класса среди всех возможных, после чего выбирается наиболее подходящая. Причем здесь можно учитывать введенные символы до текущего, чтобы, возможно, изменить выбор нейронной сети в пользу другого символа. 

Такая задача уже реализована, например ввод текста в клавиатуре смартфона, ввод текста в программных продуктах, таких как google переводчик, что позволяет более гибко печатать иероглифы.

Во втором случае мы имеем дело с отсканированными или сфотографированными изображениями, где текст уже полностью представлен в готовом виде. Это могут быть листы конспектов, работ по одному из учебных предметов, старые рукописи. При этом в зависимости от типа таких изображений определяется направление использования, к примеру, для учебных работ - их проверка, для старых рукописей - наиболее быстрый способ оцифрования таких документов, чем вручную перепечатывать каждую страницу. Но тут как раз добавляется ряд проблем, которые усложняют реализацию.

Первой проблемой является сам вид изображения. Оно может иметь определенные дефекты, такие как пятна, шум. В старых документах такие недостатки являются наиболее встречаемыми, текст и вовсе может быть расплывчатым. И его нужно сперва найти, а такие факторы существенно усложняют этот процесс. Далее идет проблема самого написания, потому что в случае машинописного текста линия текста прямая, параллельная остальным, нахождение строк не составляет трудностей. Человек при написании может изгибать строку, ширина интервалов строк может варьироваться в достаточно широких пределах. Могут быть случаи, когда, к примеру, в тех же старых рукописях из-за размытого текста совсем нет явной границы строк. В учебной работе из-за помарок и исправлений такие интервалы тоже могут быть стерты. А при онлайн распознавании и вовсе строк нет, не нужно определять их, алгоритм пропускает эту стадию и сразу считывает слова. Также текст может быть расположен в разных, независимых частях изображения, тогда решение по сегментации строк вообще становится нетипичным. 

И самая главная проблема состоит в начертаниях: строки, слова и отдельные символы текста могут накладываться друг на друга, пробелы между словами могут быть совершенно различными. В случае пересечения символов нужно сперва их разбить, то есть определить, по какой линии и как разделить сцепившиеся символы. Такой процесс разбиения затруднителен. Впрочем, кляксы, зачеркивания и исправления совсем портят возможность упростить алгоритм распознавания.\cite{li}

В общем случае процесс оффлайн распознавания рукописного текста состоит из следующих этапов:
\begin{itemize}
	\item Первоначальная обработка изображения;
	\item Поиск текста или сегментация строк;
	\item Сегментация слов;
	\item Распознавание слов;
	\item Возможное исправление первичных результатов.
\end{itemize}
	
Стоит отметить, что последний этап не является обязательным и вовсе не нужен в случаях, когда нужно извлечь лишь необходимую информацию. А четвертый этап может потребовать сегментации символов, если известно, что текст может содержать символы помимо букв алфавита. 


%% Вспомогательные команды - Additional commands
%
%\newpage % принудительное начало с новой страницы, использовать только в конце раздела
%\clearpage % осуществляется пакетом <<placeins>> в пределах секций
%\newpage\leavevmode\thispagestyle{empty}\newpage % 100 % начало новой страницы