%%%% Шаблон ВКР <<SPbPU-student-thesis-template>>  %%%%
%%
%%   Создан на основе глубокой переработки шаблона российских кандидатских и докторских диссертаций [1]. 
%%   
%%   Полный список различий может быть получен командами git.
%%   Лист авторов-составителей расположен в README.md файле.
%%   Подробные инструкции по использованию в [1,2].
%%   
%%   Рекомендуем установить TeX Live + TeXstudio
%%   <<Стандартная>> компиляция 2-3 РАЗА с помощью pdflatex + biber (для библиографии)     
%%  
%%%% Student thesis template <<SPbPU-student-thesis-template>> %%%%
%%
%%   Created on the basis of deepl modifification of the Russian candidate and doctorate thesis template [1]. 
%%   
%%   Full list of differences can be achieved by git commands.
%%   List of template authors can be seen in the README.md file.
%%   Detailed instructions of usage, see, please in [1,2].
%%     
%%   [1] github.com/AndreyAkinshin/Russian-Phd-LaTeX-Dissertation-Template 
%%   [2] Author_guide_SPBPU-student-thesis-template.pdf
%%   
%%   It is recommended to install TeX Live + TeXstudio   
%%   Default compilation 2-3 TIMES with pdflatex + biber (for the bibliography)
%%  
%%%% Preamble start %%%%  
%%
%%   Please, do not modify files in the preamble
%%
\newcommand*{\anyptfilebase}{template_settings/bpfont} 
\newcommand*{\anyptsize}{14} 		 
\RequirePackage[l2tabu,orthodox]{nag} 
\documentclass[extrafontsizes,a4paper,*pt,oneside,openany]{memoir}
%% Режим черновика
\makeatletter
\@ifundefined{c@draft}{
  \newcounter{draft}
  \setcounter{draft}{0}  % 0 --- чистовик (максимальное соблюдение ГОСТ)
                         % 1 --- черновик (отклонения от ГОСТ, но быстрая сборка итоговых PDF)
}{}
\makeatother

%% Библиография

%% Внимание! При использовании bibtex8 необходимо удалить все
%% цитирования из  ../common/characteristic.tex
\newcounter{bibliosel}
\setcounter{bibliosel}{1}           % 0 --- встроенная реализация с загрузкой файла через движок bibtex8; 1 --- реализация пакетом biblatex через движок biber

               
%%% Проверка используемого TeX-движка %%%
\usepackage{iftex}[2013/04/04]
\newif\ifxetexorluatex   % определяем новый условный оператор (http://tex.stackexchange.com/a/47579/79756)
\ifXeTeX
    \xetexorluatextrue
\else
    \ifLuaTeX
        \xetexorluatextrue
    \else
        \xetexorluatexfalse
    \fi
\fi

\RequirePackage{etoolbox}[2015/08/02]               % Для продвинутой проверки разных условий

%%% Поля и разметка страницы %%%

\usepackage{pdflscape}                              % Для включения альбомных страниц
\usepackage{geometry}                               % Для последующего задания полей

%%% Математические пакеты %%%
\usepackage{amsfonts,amsmath,amssymb,amscd,amsthm}  % Математические дополнения от AMS
% %amsthm should be loaded after amsmath!!

\usepackage{mathtools}                              % Добавляет окружение multlined

%%%% Установки для размера шрифта 14 pt %%%%
%% Формирование переменных и констант для сравнения (один раз для всех подключаемых файлов)%%
%% должно располагаться до вызова пакета fontspec или polyglossia, потому что они сбивают его работу
\newlength{\curtextsize}
\newlength{\bigtextsize}
\setlength{\bigtextsize}{13.9pt}

\makeatletter
%\show\f@size                                       % неплохо для отслеживания, но вызывает стопорение процесса, если документ компилируется без команды  -interaction=nonstopmode 
\setlength{\curtextsize}{\f@size pt}
\makeatother

%%% Кодировки и шрифты %%%
\ifxetexorluatex
    \usepackage{polyglossia}[2014/05/21]            % Поддержка многоязычности (fontspec подгружается автоматически)
\else
    \RequirePDFTeX                                  % tests for PDFTEX use and throws an error if a different engine is being used
   %%% Решение проблемы копирования текста в буфер кракозябрами
%    \input glyphtounicode.tex
%    \input glyphtounicode-cmr.tex %from pdfx package
%    \pdfgentounicode=1
    \usepackage{cmap}                               % Улучшенный поиск русских слов в полученном pdf-файле
    \defaulthyphenchar=127                          % Если стоит до fontenc, то переносы не впишутся в выделяемый текст при копировании его в буфер обмена
    
%    \usepackage[T2A]{fontenc}                       % Поддержка русских букв
    \usepackage[T2A,T1]{fontenc}
    \usepackage[utf8]{inputenc}[2014/04/30]         % Кодировка utf8
    \usepackage[english, russian]{babel}[2014/03/24]% Языки: русский, английский
\fi
\usepackage{tempora} %TemporaLGCUni of Times type
\usepackage{newtxmath} %math font of Times type
% need to set the monospace=typewritter font
%https://tex.stackexchange.com/questions/213835/using-many-typewriter-fonts-in-a-single-document

\makeatletter %load fonts for cmtt
\providecommand{\EC@ttfamily}[5]{%
	\DeclareFontShape{#1}{#2}{#3}{#4}{
		<-8.5>#50800
		<8.5-9.5>#50900
		<9.5-10.5>#51000
		<10.5-11.5>#51095
		<11.5-13>#51200
		<13-15.5>#51440
		<15.5-18.5>#51728
		<18.5-22>#52074
		<22-27>#52488
		<27-32>#52986
		<32->#53583}{}}
\DeclareFontFamily{T1}{cmtt}{}
\DeclareFontFamily{T2A}{cmtt}{}
\EC@ttfamily{T1}{cmtt}{m}{n}{ectt}
\EC@ttfamily{T1}{cmtt}{m}{sl}{ecst}
\EC@ttfamily{T1}{cmtt}{m}{it}{ecit}
\EC@ttfamily{T1}{cmtt}{m}{sc}{ectc}
\DeclareFontShape{T1}{cmtt}{bx}{n}%
{<->ssub*cmtt/m/n}{}
\DeclareFontShape{T1}{cmtt}{bx}{it}%
{<->ssub*cmtt/m/it}{}
\EC@ttfamily{T2A}{cmtt}{m}{n}{latt}
\EC@ttfamily{T2A}{cmtt}{m}{sl}{last}
\EC@ttfamily{T2A}{cmtt}{m}{it}{lait}
\EC@ttfamily{T2A}{cmtt}{m}{sc}{latc}
\DeclareFontShape{T2A}{cmtt}{bx}{n}%
{<->ssub*cmtt/m/n}{}
\DeclareFontShape{T2A}{cmtt}{bx}{it}%
{<->ssub*cmtt/m/it}{}
\makeatletter

%\makeatletter %load fonts for cmtt
%\providecommand{\EC@ttfamily}[5]{%
%	\DeclareFontShape{#1}{#2}{#3}{#4}{
%		<-8.5>#50800
%		<8.5-9.5>#50900
%		<9.5-10.5>#51000
%		<10.5-11.5>#51095
%		<11.5-13>#51200
%		<13-15.5>#51440
%		<15.5-18.5>#51728
%		<18.5-22>#52074
%		<22-27>#52488
%		<27-32>#52986
%		<32->#53583}{}}
%\DeclareFontFamily{T2A}{cmtt}{\hyphenchar\font\m@ne}
%\EC@ttfamily{T2A}{cmtt}{m}{n}{latt}
%\EC@ttfamily{T2A}{cmtt}{m}{sl}{last}
%\EC@ttfamily{T2A}{cmtt}{m}{it}{lait}
%\EC@ttfamily{T2A}{cmtt}{m}{sc}{latc}
%\DeclareFontShape{T2A}{cmtt}{bx}{n}%
%{<->ssub*cmtt/m/n}{}
%\DeclareFontShape{T2A}{cmtt}{bx}{it}%
%{<->ssub*cmtt/m/it}{}
%\makeatletter

%\makeatletter
%\input{t1lmtt.fd}
%\@namedef{T1+lmtt}{}
%\makeatother


\renewcommand{\ttdefault}{cmtt}
%\renewcommand{\ttdefault}{lcmtt} %покрупнее
%\usepackage[scaled=.85]{DejaVuSansMono} %слишком похож на рубленый
%\newfont{\wasyten}{wasy10} %название команды для вызова / название шрифта



%Другие шрифты:
% математика
%\usepackage[lite]{mtpro2}
%https://pctex.com/mtpro2.html
% текст        
% https://www.ctan.org/pkg/paratype
%       \usepackage[scaled=0.925]{XCharter}[2017/06/25] % Подключение русифицированных шрифтов XCharter
%\usepackage{pscyr}
%    \IfFileExists{pscyr.sty}{}{}  % Красивые русские шрифты
%\fi

%https://tex.stackexchange.com/questions/8260/what-are-the-various-units-ex-em-in-pt-bp-dd-pc-expressed-in-mm
\usepackage{printlen} %для измерения и вывода параменторов шрифтов, отступов, интервалов

\usepackage{bm} %для жирных начертаний символов

\usepackage{csquotes} %to check quotes

%%% Оформление абзацев %%%
\usepackage{indentfirst}                            % Красная строка

%%% Цвета %%%
%\usepackage[dvipsnames,usenames]{color}
\usepackage{colortbl}
\usepackage[dvipsnames, table, hyperref, cmyk]{xcolor} % Вероятно, более новый вариант, вместо предыдущих двух строк. Конвертация всех цветов в cmyk заложена как удовлетворение возможного требования типографий. Возможно конвертирование и в rgb.

%%% Таблицы %%%
\usepackage{longtable}                              % Длинные таблицы
\usepackage{multirow,makecell}                      % Улучшенное форматирование таблиц:
													% multirow - строки на несколько ячеек, 
												
													% makecell - сесколько строк в ячейке.
													% не работает, если внутри, например, \verb|text| -> \texttt{text}
													% аналоги
%https://tex.stackexchange.com/questions/2441/how-to-add-a-forced-line-break-inside-a-table-cell								
						
													

%%% Общее форматирование
%\usepackage{soul} % используется ulem
\usepackage{soulutf8}                               % Поддержка переносоустойчивых подчёркиваний и зачёркиваний
\usepackage{icomma}                                 % Запятая в десятичных дробях



%%% Предметный указатель  ГОСТ 7.78-99 Index %%%
%c обобщенными рубриками или развернутый
%или указатель терминов (в общем случае - произвольное число указателей)
%подключать до hyperref

%\usepackage{makeidx} %возможно, необходимо подключить И/ИЛИ пройти Tools-> Commands -> MakeIndex

\usepackage{imakeidx} 
%\indexsetup{level=\section*,toclevel=section,noclearpage}
\makeindex[program=makeindex,
options=-s template_settings/common/myindex.ist, %подключаем стилевой файл для форматирования вывода
name=ru, % префикс для русских указателей 
% если убрать <<ru>>, то для работы дефолтового придется вручную включать Tools-> Commands -> MakeIndex
title={\chapterLight{} 
%   \hrule{}
	Предметный указатель
%	\hrule{}
} 
%,columns=1 %по умолчанию 2
]
\makeindex[program=makeindex,
options=-s template_settings/common/myindex.ist, %подключаем стилевой файл для форматирования вывода
name=en, % префикс для английских указателей
title={\chapterLight{}
%	\hrule{}
	Index
%	\hrule{}
} 
%,columns=1 %по умолчанию 2
] 
%убрать добавление <<title>> в содержание:
%\noindexintoc %not to add index title in PURE makeidx %intoc is false by default with imakeidx


%       https://tex.stackexchange.com/a/132415/44348
%\makeatletter
%% we want hyphenation also in the first word
\renewcommand{\@idxitem}{\par\hangindent40\p@\hspace{0pt}\ignorespaces}
%% we don't want a page break before a subitem %implemented in the previous one
%%\renewcommand\subitem{\@idxitem\nobreak\hspace*{20\p@}}
%\makeatother


%%% Фиксация плавающих объектов





%%% Гиперссылки %%%
\usepackage{hyperref}[2012/11/06]

%%% Изображения %%%
\usepackage{graphicx}[2014/04/25]                   % Подключаем пакет работы с графикой

%%% Списки %%%
\usepackage[shortlabels]{enumitem} % shortlabels для того, чтобы изменять токены в списках с дефолтных (иерархическая структура) на произвольныею

%%% Подписи %%%
\usepackage{caption}[2013/05/02]                    % Для управления подписями (рисунков и таблиц) % Может управлять номерами рисунков и таблиц с caption %Иногда может управлять заголовками в списках рисунков и таблиц


\usepackage{subcaption}[2013/02/03]                 % Работа с подрисунками и подобным

%%% Счётчики %%%
%\usepackage[figure,table]{totalcount}               % Счётчик рисунков и таблиц. Взамен используется xassoccnt 
\usepackage{totcount}                               % Пакет создания счётчиков на основе последнего номера подсчитываемого элемента (может требовать дважды компилировать документ)
\usepackage{totpages}                               % Счётчик страниц, совместимый с hyperref (ссылается на номер последней страницы). Желательно ставить последним пакетом в преамбуле

\usepackage{xassoccnt} % для подсчета сумм приложений, рисунков, таблиц 


%%% Продвинутое управление групповыми ссылками (пока только формулами) %%%
\ifxetexorluatex
    \usepackage{cleveref}                           % cleveref корректно считывает язык из настроек polyglossia
\else
    \usepackage[russian]{cleveref}                  % cleveref имеет сложности со считыванием языка из babel. Такое решение русификации вывода выбрано вместо определения в documentclass из опасности что-то лишнее передать во все остальные пакеты, включая библиографию.
\fi
\creflabelformat{equation}{#2#1#3}                  % Формат по умолчанию ставил круглые скобки вокруг каждого номера ссылки, теперь просто номера ссылок без какого-либо дополнительного оформления



\ifnumequal{\value{draft}}{1}{% Черновик
    \usepackage[firstpage]{draftwatermark}
    \SetWatermarkText{DRAFT}
    \SetWatermarkFontSize{14pt}
    \SetWatermarkScale{15}
    \SetWatermarkAngle{45}
}{}

  
%%% Прикладные пакеты %%% 
%\usepackage{calc}               % Пакет для расчётов параметров, например длины

%%% Для добавления Стр. над номерами страниц в оглавлении
%%% http://tex.stackexchange.com/a/306950
\usepackage{afterpage}

\urlstyle{rm} % links in Times


%\makeatletter
%%расстояние после ToC title до 1ой строчки 
%%для достижения одинаковых отсупов переопределено формирование базового ToC
%\renewcommand{\aftertoctitle}{\par\nobreak\vskip1\curtextsize}
%\makeatother

%https://tex.stackexchange.com/questions/170912/contents-page-in-two-different-languages
%\makeatletter
\newcommand\russiantableofcontents{%
%	\if@twocolumn
%	\@restonecoltrue\onecolumn
%	\else
%	\@restonecolfalse
%	\fi
	%  \begin{otherlanguage}{russian}
	\chapter*{%
	\normalfont\MakeUppercase{Содержание} %слово <<Содержание>> в стилю chaperLight, по факту убираем \bfseries
%		    \contentsname
%		    \@mkboth{\MakeUppercaseСодержание}
%		            {\MakeUppercaseСодержание}%
	}%
%\hrule
\vspace*{-1\curtextsize} %убрать лишний отступ в таблице
	\@starttoc{tuc}%
	%  \end{otherlanguage}
%	\if@restonecol\twocolumn\fi
}
\newcommand{\addtocru}[2]{%
	\addcontentsline{tuc}{#1}{\protect\numberline{\csname the#1\endcsname}#2}%
%	\addcontentsline{tuc}{#1}{#2}%
}
\newcommand{\addtocruNoProtect}[2]{%
%	\addcontentsline{tuc}{#1}{\protect\numberline{\csname the#1\endcsname}#2}%
		\addcontentsline{tuc}{#1}{#2}%
}

%обеспечение красивого порядка вывода содержаний и названий разделов, подразделов и т.п.
\newcommand\englishtableofcontents{%
	%	\if@twocolumn
	%	\@restonecoltrue\onecolumn
	%	\else
	%	\@restonecolfalse
	%	\fi
	%  \begin{otherlanguage}{russian}
	\chapter*{%
		\normalfont\MakeUppercase{Content} %слово <<Содержание>> в стилю chaperLight, по факту убираем \bfseries
		%		    \contentsname
		%		    \@mkboth{\MakeUppercaseСодержание}
		%		            {\MakeUppercaseСодержание}%
	}%
	%\hrule
	\vspace*{-1\curtextsize} %убрать лишний отступ в таблице
	\@starttoc{tec}%
	%  \end{otherlanguage}
	%	\if@restonecol\twocolumn\fi
}
\newcommand{\addtocen}[2]{%
		\addcontentsline{tec}{#1}{\protect\numberline{\csname the#1\endcsname}#2}%
%	\addcontentsline{tec}{#1}{#2}%
}
\newcommand{\addtocenNoProtect}[2]{%for preface, introduction etc
%	\addcontentsline{tec}{#1}{\protect\numberline{\csname the#1\endcsname}#2}%
		\addcontentsline{tec}{#1}{#2}%
}


%стандартный вывод в toc можно использовать, если издание только на английском или русском.
%переопределена, чтобы обеспечить одинаковые отсупы от названия ToC (toc, tec, tuc) до первой строки
\renewcommand\tableofcontents{%
	%	\if@twocolumn
	%	\@restonecoltrue\onecolumn
	%	\else
	%	\@restonecolfalse
	%	\fi
	%  \begin{otherlanguage}{russian}
	\chapter*{%
		\MakeUppercase{Содержание} %слово <<Содержание>> 
		%		    \contentsname
		%		    \@mkboth{\MakeUppercaseСодержание}
		%		            {\MakeUppercaseСодержание}%
	}%
	%\hrule
%	\vspace*{-0.58\curtextsize} %убрать/добавить отступ от таблицы
	\@starttoc{toc}%
	%  \end{otherlanguage}
	%	\if@restonecol\twocolumn\fi
}
\newcommand{\addetoc}[2]{%
		\addcontentsline{toc}{#1}{\protect\numberline{\csname the#1\endcsname}#2}%
}
%\newcommand{\addtocru}[2]{%
%	\addcontentsline{tuc}{#1}{\protect\numberline{\csname the#1\endcsname}#2}%
%	%	\addcontentsline{tuc}{#1}{#2}%
%}

%\makeatother

%http://latex.org/forum/viewtopic.php?t=5438         
\usepackage{tabularx}

%%https://tex.stackexchange.com/a/362229
\usepackage{datatool-base}
\usepackage{mfirstuc} %первая буква прописная

\usepackage{layouts}

\newenvironment{abstr}{\smallA\itshape}{\normalfont\normalsize}


\usepackage[normalem]{ulem} % для перечеркнутых сроков команда \sout{text}
\newcommand{\soutthick}[1]{%
	\renewcommand{\ULthickness}{2.4pt}%
	\sout{#1}%
	\renewcommand{\ULthickness}{.4pt}% Resetting to ulem default
}

%для подчёркнутых команд
%https://tex.stackexchange.com/questions/270286/uline-not-work-for-command-arguments
\useunder{\uline}{\ulined}{}

\usepackage{environ} % for Uppercase in Keywords
%https://tex.stackexchange.com/questions/249628/uppercase-whole-newenvironment
% недостаток - новые окружения не подхватываются TexStudio

\usepackage{textcase} % for \MakeTextUppercase

%for svg pictures
%\usepackage{svg}


%%% Mailto %%% 
%%%https://tex.stackexchange.com/questions/128424/how-to-create-email-hyperlink-with-predefined-subject-in-latex
%% unfortunatelly Adobe does not handle Recipient name + email, e.g.
%% Vladimir Parkhomenko<parhomenko.v@gmail.com>


%mailto with subject (impossible with href)
%mailto anybody without email body
\makeatletter
\newcommand\mailtoab[3]{%                %\newcommand\tpj@compose@mailto[3]{%
	\edef\@tempa{mailto:#1?subject=#2 }%
	\edef\@tempb{\expandafter\html@spaces\@tempa\@empty}%
	\href{\@tempb}{#3}}
\catcode\%=11
\def\html@spaces#1 #2{#1%20\ifx#2\@empty\else\expandafter\html@spaces\fi#2}
	\catcode\%=14
	\makeatother
	
	
	%${email}{Subject}{email start body}{text in pdf}
	\makeatletter
	\newcommand\mailto[4]{%                %\newcommand\tpj@compose@mailto[3]{%
		\edef\@tempa{mailto:#1?subject=#2\&body=#3 }%
		\edef\@tempb{\expandafter\html@spaces\@tempa\@empty}%
		\href{\@tempb}{#4}}
	%% with %20 instead of spaces
	%\catcode\%=11
	%\def\html@spaces#1 #2{#1%20\ifx#2\@empty\else\expandafter\html@spaces\fi#2}
	%\catcode\%=14
	\makeatother
	
	%% MLABSED 2017 author
	%%${email}{Subject}{email start body}{text in pdf}
	\makeatletter
	\newcommand\mailtoMLABSEDauthor[3]{%                
		\edef\@tempa{mailto:#1?subject=MLABSED 2017\&body=#2 }%
		\edef\@tempb{\expandafter\html@spaces\@tempa\@empty}%
		\href{\@tempb}{#3}}
	%% with %20 instead of spaces
	%\catcode\%=11
	%\def\html@spaces#1 #2{#1%20\ifx#2\@empty\else\expandafter\html@spaces\fi#2}
	%\catcode\%=14
	\makeatother
	
	
	%%Vladimir Parkhomenko
	\makeatletter
	\newcommand\mailtopa[1]{%                %\newcommand\tpj@compose@mailto[3]{%
		\edef\@tempa{mailto:parhomenko.v@gmail.com?subject=#1\&body=Dear Vladimir, }%
		\edef\@tempb{\expandafter\html@spaces\@tempa\@empty}%
		\href{\@tempb}{Vladimir.Parkhomenko@spbstu.ru}}
	\catcode\%=11
	\def\html@spaces#1 #2{#1%20\ifx#2\@empty\else\expandafter\html@spaces\fi#2}
		\catcode\%=14
		\makeatother
		
		%%Alexey Buzmakov
		\makeatletter
		\newcommand\mailtobu[1]{%                %\newcommand\tpj@compose@mailto[3]{%
			\edef\@tempa{mailto:abuzmakov@gmail.com?subject=#1\&body=Dear Alexey, }%
			\edef\@tempb{\expandafter\html@spaces\@tempa\@empty}%
			\href{\@tempb}{abuzmakov@gmail.com}}
		\catcode\%=11
		\def\html@spaces#1 #2{#1%20\ifx#2\@empty\else\expandafter\html@spaces\fi#2}
			\catcode\%=14
			\makeatother
			
			%%Xenia Naidenova
			\makeatletter
			\newcommand\mailtona[1]{%                %\newcommand\tpj@compose@mailto[3]{%
				\edef\@tempa{mailto:ksennaidd@gmail.com?subject=#1\&body=Dear Xenia, }%
				\edef\@tempb{\expandafter\html@spaces\@tempa\@empty}%
				\href{\@tempb}{ksennaidd@gmail.com}}
			\catcode\%=11
			\def\html@spaces#1 #2{#1%20\ifx#2\@empty\else\expandafter\html@spaces\fi#2}
				\catcode\%=14
				\makeatother
				
				
				%%Konstantin Shvetsov
				\makeatletter
				\newcommand\mailtosh[1]{%                %\newcommand\tpj@compose@mailto[3]{%
					\edef\@tempa{mailto:shvetsov@inbox.ru?subject=#1\&body=Dear Konstantin, }%
					\edef\@tempb{\expandafter\html@spaces\@tempa\@empty}%
					\href{\@tempb}{Konstantin.Shvetsov@spbstu.ru}}
				\catcode\%=11
				\def\html@spaces#1 #2{#1%20\ifx#2\@empty\else\expandafter\html@spaces\fi#2}
					\catcode\%=14
					\makeatother


\usepackage{tabu, tabulary}  %таблицы с автоматически подбирающейся шириной столбцов
\usepackage{fr-longtable}    %ради \endlasthead

% Листинги с исходным кодом программ
\usepackage{fancyvrb}
\usepackage{listings}
\lccode`\~=0\relax %Без этого хака из-за особенностей пакета listings перестают работать конструкции с \MakeLowercase и т. п. в (xe|lua)latex

% Русская традиция начертания греческих букв
\usepackage{upgreek} % прямые греческие ради русской традиции

%https://tex.stackexchange.com/a/62351/44348
% Микротипографика
\ifnumequal{\value{draft}}{0}{% Только если у нас режим чистовика
    \usepackage[final,letterspace=150]{microtype}[2016/05/14] % улучшает представление букв и слов в строках, может помочь при наличии отдельно висящих слов
%    \lsstyle for letterspace style of letters
}{}

% Отметка о версии черновика на каждой странице
% Чтобы работало надо в своей локальной копии по инструкции
% https://www.ctan.org/pkg/gitinfo2 создать небходимые файлы в папке
% ./git/hooks
% If you’re familiar with tweaking git, you can probably work it out for
% yourself. If not, I suggest you follow these steps:
% 1. First, you need a git repository and working tree. For this example,
% let’s suppose that the root of the working tree is in ~/compsci
% 2. Copy the file post-xxx-sample.txt (which is in the same folder of
% your TEX distribution as this pdf) into the git hooks directory in your
% working copy. In our example case, you should end up with a file called
% ~/compsci/.git/hooks/post-checkout
% 3. If you’re using a unix-like system, don’t forget to make the file executable.
% Just how you do this is outside the scope of this manual, but one
% possible way is with commands such as this:
% chmod g+x post-checkout.
% 4. Test your setup with “git checkout master” (or another suitable branch
% name). This should generate copies of gitHeadInfo.gin in the directories
% you intended.
% 5. Now make two more copies of this file in the same directory (hooks),
% calling them post-commit and post-merge, and you’re done. As before,
% users of unix-like systems should ensure these files are marked as
% executable.
\ifnumequal{\value{draft}}{1}{% Черновик
   \IfFileExists{.git/gitHeadInfo.gin}{                                        
      \usepackage[mark,pcount]{gitinfo2}
      \renewcommand{\gitMark}{rev.\gitAbbrevHash\quad\gitCommitterEmail\quad\gitAuthorIsoDate}
      \renewcommand{\gitMarkFormat}{\color{Gray}\small\bfseries}
   }{}
}{}         
%%%%%%%%%%%%%%%%%%%%%%%%%%%%%%%%%%%%%%%%%%%%%%%%%%%%%%
%%%% Файл упрощённых настроек шаблона диссертации %%%%
%%%%%%%%%%%%%%%%%%%%%%%%%%%%%%%%%%%%%%%%%%%%%%%%%%%%%%

%%% Инициализирование переменных, не трогать!  %%%
\newcounter{intvl}
\newcounter{otstup}
\newcounter{contnumeq}
\newcounter{contnumfig}
\newcounter{contnumtab}
\newcounter{pgnum}
\newcounter{chapstyle}
\newcounter{headingdelim}
\newcounter{headingalign}
\newcounter{headingsize}
\newcounter{tabcap}
\newcounter{tablaba}
\newcounter{tabtita}
\newcounter{docType} 		% тип документа
\newcounter{tskPrint} 		% печать Задания на ВКР двух(одно)сторонняя
\newcounter{tskPages}       % для учёта количества страниц в Задании
\newcounter{tskPageFirst}   % для учёта количества страниц в Задании
\newcounter{tskPageLast}    % для учёта количества страниц в Задании 
\newcounter{sumPrint} 		% печать Реферата на ВКР двух(одно)сторонняя
\newcounter{sumPages}       % для учёта количества страниц в Реферате
\newcounter{sumPageFirst}   % для учёта количества страниц в Реферате
\newcounter{sumPageLast}    % для учёта количества страниц в Реферате 
\newcommand{\Single}{0.78}  % пропорция для одинароного отступа в \Spacing
%%%%%%%%%%%%%%%%%%%%%%%%%%%%%%%%%%%%%%%%%%%%%%%%%%

%%% Область упрощённого управления оформлением %%%

% Управление перенесено в главые файлы компиляции ВКР, Задания, Реферата
\setcounter{tskPrint}{0} %по умолчанию односторонняя печать              
%\setcounter{sumPrint}{0} %по умолчанию односторонняя печать 

%% Интервал между заголовками и между заголовком и текстом
% Заголовки отделяют от текста сверху и снизу тремя интервалами (ГОСТ Р 7.0.11-2011, 5.3.5)
\setcounter{intvl}{3}               % Коэффициент кратности к размеру шрифта

% Заголовки отделяют от текста сверху и снизу тремя интервалами 
\newcommand{\intvlS}{1.5}               % Коэффициент кратности к размеру шрифта SPbPU-student-templates

\newcommand{\intervalS}{\vspace{\intvlS\curtextsize}}

% печать списка источников в Задании
\newcommand{\printbibliographyTask}{\vspace{-0.28\curtextsize}
	\printbibliography[env=tsk] % печать списка литературы в исходных данных
	\vspace{-0.28\curtextsize}}


%% Отступы у заголовков в тексте
\setcounter{otstup}{0}              % 0 --- без отступа; 1 --- абзацный отступ

%% Нумерация формул, таблиц и рисунков
\setcounter{contnumeq}{0}           % Нумерация формул: 0 --- пораздельно (во введении подряд, без номера раздела); 1 --- сквозная нумерация по всей диссертации
\setcounter{contnumfig}{0}          % Нумерация рисунков: 0 --- пораздельно (во введении подряд, без номера раздела); 1 --- сквозная нумерация по всей диссертации
\setcounter{contnumtab}{0}          % Нумерация таблиц: 0 --- пораздельно (во введении подряд, без номера раздела); 1 --- сквозная нумерация по всей диссертации


%% Нумерация подстраничных сносок (ссылок)
%сквозная
\counterwithout{footnote}{chapter} %сквозная нумерация подразделов (во всех главах)


%% Нумерация подразделов
%убрать номер главы в секции
%\counterwithout{section}{chapter} %сквозная нумерация подразделов (во всех главах)
%\renewcommand\thesection{\arabic{section}} %в каждой главе нумерация заново

%\renewcommand\thesection{\arabic{section}}
%\renewcommand\thefigure{\fbox{\arabic{figure}}}
%\renewcommand\thetable{\arabic{table}}
%\renewcommand\theequation{\arabic{equation}}



%\counterwithout{section}{chapter}
%\counterwithout{figure}{chapter}
%\counterwithout{table}{chapter}
%\counterwithout{equation}{chapter}

%\counterwithin{section}{chapter}
%\counterwithin{figure}{chapter}
%\counterwithin{table}{chapter}

%% Оглавление

\setcounter{pgnum}{1}               %NB УДАЛЕНО ФИЗИЧЕСКИ 0 --- номера страниц никак не обозначены; 1 --- Стр. над номерами страниц (дважды компилировать после изменения)  
\settocdepth{subsection} %             до какого уровня подразделов выносить в оглавление
\setsecnumdepth{subsubsection}         % до какого уровня нумеровать подразделы


%% Текст и форматирование заголовков
\setcounter{chapstyle}{1}           % 0 --- разделы только под номером; 1 --- разделы с названием "Глава" перед номером
\setcounter{headingdelim}{2}        % 0 --- номер отделен пропуском в 1em или \quad; 1 --- номера разделов и ений отделены точкой с пробелом, подразделы пропуском без точки; 2 --- номера разделов, подразделов и приложений отделены точкой с пробелом.

%% Выравнивание заголовков в тексте
\setcounter{headingalign}{0}        % 0 --- по центру; 1 --- по левому краю

%% Размеры заголовков в тексте
\setcounter{headingsize}{0}         % 0 --- SPbPU style, все всегда 14 пт; 1 --- пропорционально изменяющийся размер в зависимости от базового шрифта;

%% Подпись таблиц
\setcounter{tabcap}{1}              % 0 --- по ГОСТ, номер таблицы и название разделены тире, выровнены по левому краю, при необходимости на нескольких строках; 1 --- подпись таблицы не по ГОСТ, на двух и более строках, дальнейшие настройки: 
%Выравнивание первой строки, с подписью и номером
\setcounter{tablaba}{2}             % 0 --- по левому краю; 1 --- по центру; 2 --- по правому краю
%Выравнивание строк с самим названием таблицы
\setcounter{tabtita}{1}             % 0 --- по левому краю; 1 --- по центру; 2 --- по правому краю
%Разделитель записи «Таблица #» и названия таблицы
\newcommand{\tablabelsep}{space}   % space = пробел, period =  (определены в подключенных пакетах)

%% Подпись рисунков
%Разделитель записи «Рисунок #» и названия рисунка
\newcommand{\figlabelsep}{period}   % emdash = тире, определён в common/styles; period = точка определён в подключенных пакетах; space
%\newcommand{\figlabelsep}{emdash}   % emdash = тире, определён в common/styles; period = точка определён в подключенных пакетах


%%% Цвета гиперссылок %%%
% Latex color definitions: http://latexcolor.com/

%\definecolor{linkcolor}{rgb}{0.9,0,0}
%\definecolor{citecolor}{rgb}{0,0.6,0}
%\definecolor{urlcolor}{rgb}{0,0,1}


%\definecolor{linkbordercolor}{rgb}{0,0,1}

\definecolor{linkcolor}{HTML}{FF0000} %very light red from the SPbPU brandbook (2nd level)
\definecolor{citecolor}{HTML}{6CF479} %very light green from the SPbPU brandbook (2nd level)
\definecolor{urlcolor}{HTML}{4481BA} %very light blue from the SPbPU brandbook (2nd level)

%\definecolor{linkcolor}{rgb}{0,0,0} %black
%\definecolor{citecolor}{rgb}{0,0,0} %black
%\definecolor{urlcolor}{rgb}{0,0,0} %black               
%%% Переопределение именований, чтобы можно было и в преамбуле использовать %%%
\renewcommand{\chaptername}{Chapter}
\renewcommand{\appendixname}{Приложение} % (ГОСТ Р 7.0.11-2011, 5.7)
       
%%% Кодировки и шрифты %%%
\ifxetexorluatex
    \setmainlanguage[babelshorthands=true]{russian}  % Язык по-умолчанию русский с поддержкой приятных команд пакета babel
    \setotherlanguage{english}                       % Дополнительный язык = английский (в американской вариации по-умолчанию)
    \setmonofont{Courier New}
    \newfontfamily\cyrillicfonttt{Courier New}
    \ifXeTeX
        \defaultfontfeatures{Ligatures=TeX,Mapping=tex-text}
    \else
        \defaultfontfeatures{Ligatures=TeX}
    \fi
    \setmainfont{Times New Roman}
    \newfontfamily\cyrillicfont{Times New Roman}
    \setsansfont{Arial}
    \newfontfamily\cyrillicfontsf{Arial}
\else
    \IfFileExists{pscyr.sty}{\renewcommand{\rmdefault}{ftm}}{}
\fi

%%% Подписи %%%
\captionsetup{%
singlelinecheck=off,                % Многострочные подписи, например у таблиц
skip=5pt,                           % Вертикальная отбивка между подписью и содержимым рисунка или таблицы определяется ключом
justification=centering            % Центрирование подписей, заданных командой \caption
}

%\setlength{\abovecaptionskip}{0pt} %альтернатива для skip, но не распространяется на longtable!
%\setlength{\belowcaptionskip}{0pt}
%\captionwidth{\linewidth}
%\normalcaptionwidth

% для изменения отступов от floats (e.g. table,figure) & minipage
\newlength{\mfloatsep}
\setlength{\mfloatsep}{4mm plus 0.7mm minus 0.7mm} %3mm для A5

% фиксируем расстояния с помощью пакета layouts
\setlength{\textfloatsep}{\mfloatsep} % расстояние от текста до float, если float прижат к верхнему или нижнему краю
\setlength{\floatsep}{\mfloatsep} % расстояние от float до float (если оба сверху/снизу)
\setlength{\intextsep}{\mfloatsep} % расстояние от текста до float, если float снизу и сверху ограничен текстом 
%
%% фактически из-за бокса, внутрь которого помещается \captionof{figure} происходит увеличаение на 1мм отступа в соответствующем элементе!
%
%% по требованиям СПбПУ как раз необходим отступ 4мм от рисунка


%Возможно более гибко задавать отступы, например:
%\setlength{\floatsep}{12pt plus 2pt minus 2pt}
%\setlength{\textfloatsep}{20pt plus 2pt minus 4pt}
%\setlength{\intextsep}{\floatsep}

%https://tex.stackexchange.com/questions/60477/remove-space-after-figure-and-before-text
%https://tex.stackexchange.com/questions/26521/how-to-change-the-spacing-between-figures-tables-and-text




%%% Парный к \smallA шрифт 13bp в подписи%%%
%TO-DO как напрямую связать со \smallA
%\DeclareCaptionFont{font13bp}{\smallA\selectfont} %к сожалению, приводит к отсупу после номера рисунка
\DeclareCaptionFont{font13bp}{\fontsize{13bp}{16.77bp}\selectfont} %аналог задания вручную
\DeclareCaptionFont{font12bp}{\small\selectfont} %аналог задания вручную



%%% Рисунки %%%
\DeclareCaptionLabelSeparator*{emdash}{~--- }             % (ГОСТ 2.105, 4.3.1)

\DeclareCaptionLabelFormat{figwithoutspace}{#1#2}
%\captionsetup[figure]{labelformat=figwithoutspace,labelsep=none,name=Fig.}

\captionsetup[figure]{labelformat=figwithoutspace,labelsep=\figlabelsep,position=bottom,labelfont={font12bp},textfont={font12bp}}

%\setlength{\belowcaptionskip}{0pt} %расстояние между 
%\caption* -- подрисуночной подписи и
%\caption  -- подписи к рисунку с номером
%необходимо менять вслед за добавлением \vskip в \captionsetup

%\setfloatadjustment{figure}{%
%	\setlength{\belowcaptionskip}{-3pt}   % чтобы обивка после рисунков была 3mm, так как caption добавляет около 1мм к 3мм. 
%}




%%% Таблицы %%%
\ifnumequal{\value{tabcap}}{0}{%
    \newcommand{\tabcapalign}{\raggedright}  % по левому краю страницы или аналога parbox

    \DeclareCaptionFormat{tablecaption}{\tabcapalign #1#2#3}
    \captionsetup[table]{labelsep=emdash}        % тире как разделитель идентификатора с номером от наименования
}{%
    \ifnumequal{\value{tablaba}}{0}{%
        \newcommand{\tabcapalign}{\raggedright}  % по левому краю страницы или аналога parbox
    }{}

    \ifnumequal{\value{tablaba}}{1}{%
        \newcommand{\tabcapalign}{\centering}    % по центру страницы или аналога parbox
    }{}

    \ifnumequal{\value{tablaba}}{2}{%
        \newcommand{\tabcapalign}{\raggedleft}   % по правому краю страницы или аналога parbox
    }{}

    \ifnumequal{\value{tabtita}}{0}{%
        \newcommand{\tabtitalign}{\raggedright}  % по левому краю страницы или аналога parbox
    }{}

    \ifnumequal{\value{tabtita}}{1}{%
        \newcommand{\tabtitalign}{\centering}    % по центру страницы или аналога parbox
    }{}

    \ifnumequal{\value{tabtita}}{2}{%
        \newcommand{\tabtitalign}{\raggedleft}   % по правому краю страницы или аналога parbox
    }{}

    \DeclareCaptionFormat{tablecaption}{\tabcapalign #1#2\par %\hline  % Идентификатор таблицы на отдельной строке
        \tabtitalign{#3}}                                       % Наименование таблицы строкой ниже
    \captionsetup[table]{labelsep=\tablabelsep}                 % разделитель идентификатора с номером от наименования
}
\DeclareCaptionFormat{tablenocaption}{\tabcapalign #1\strut}    % Наименование таблицы отсутствует

\newlength{\ltskip}
\setlength{\ltskip}{2pt}
\captionsetup[table]{format=tablecaption,singlelinecheck=off,position=top,labelfont={font12bp},textfont={font12bp}}  % многострочные наименования и прочее
\DeclareCaptionLabelFormat{continued}{\\[-14pt]Продолжение табл.~\!#2}



%%% Подписи подрисунков %%%
\renewcommand{\thesubfigure}{\alph{subfigure}}           % Буквенные номера подрисунков
\captionsetup[subfigure]{font={font12bp},               % Шрифт подписи названий подрисунков (отличается от основного)
	labelfont={font12bp},textfont={font12bp},
    labelformat=brace,                                    % Формат обозначения подрисунка
    singlelinecheck=off,
%    position=left,
    justification=raggedright 							 %выравнивание влево
%    justification=centering,                              % Выключка подписей (форматирование), один из вариантов            
}

%%% Подписи подрисунков SPbPU%%%
% реализован подход по первой ссылке, он позволяет масштабировать количество подрисунков
%https://tex.stackexchange.com/a/273169/44348
%https://tex.stackexchange.com/a/225914/44348
\usepackage[export]{adjustbox}



%%% Настройки гиперссылок %%%
\ifLuaTeX
    \hypersetup{
        unicode,                % Unicode encoded PDF strings
    }
\fi


\newcommand{\thesisTitle}{Название выпускной квалификационной работы}


\hypersetup{
    linktocpage=true,           % ссылки с номера страницы в оглавлении, списке таблиц и списке рисунков
%    linktoc=all,                % both the section and page part are links
%    pdfpagelabels=false,        % set PDF page labels (true|false)
    plainpages=false,           % Forces page anchors to be named by the Arabic form  of the page number, rather than the formatted form
    %colorlinks,                 % ссылки отображаются раскрашенным текстом, а не раскрашенным прямоугольником, вокруг текста
    citebordercolor={0.287 0.89 0.349}, %(RGB colour) with default {0 1 0}: The colour of the box around citations
    filebordercolor={0 .5 .5}, % (RGB colour) with default {0 .5 .5}: The colour of the box around links to files
    linkbordercolor={0.93 0 0}, % (RGB colour) with default {1 0 0}: The colour of the box around normal links
    menubordercolor={1 0 0}, % (RGB colour) with default {1 0 0}: The colour of the box around Acrobat menu links
    urlbordercolor={0.313 0.776 0.878}, % (RGB colour) with default {0 1 1}: The colour of the box around links to URLs
    pdfborder={0 0 1}, % The style of box around links; defaults to a box with lines of 1pt thickness, but the colorlinks option resets it to produce no border.
%    linkcolor={linkcolor},
%    citecolor={citecolor},      % цвет ссылок-цитат
%    urlcolor={urlcolor},        % цвет гиперссылок
    %hidelinks,                  % Hide links (removing color and border)
%    pdftitle={\thesisTitle},    % Заголовок pdf-файла
%    pdfauthor={\AuthorFull},  % Автор
%    pdfsubject={\thesisSpecialtyNumber\ \thesisSpecialtyTitle},      % Тема
%    pdfcreator={Создатель},     % Создатель, Приложение
%    pdfproducer={Производитель},% Производитель, Производитель PDF
%    pdfkeywords={\keywords},    % Ключевые слова
    pdflang={eng}, %eng %ru
    % % bookmarks settings
    bookmarks=true,
    bookmarksnumbered=true, % put section numbers
    bookmarksopen=true, %open when the pdf is opened
    bookmarksopenlevel=0, %chapter's level is enough to see
    bookmarksdepth=0 %set the depth of the levels in the pdf navigation bar
}

% %improve the bookmarksnumbered representation:
\makeatletter
\renewcommand{\Hy@numberline}[1]{#1. } %add the dot after a number
\makeatother


\ifnumequal{\value{draft}}{1}{% Черновик
    \hypersetup{
        draft,
    }
}{}

%%% Шаблон %%%
\DeclareRobustCommand{\todo}{\textcolor{red}}       % решаем проблему превращения названия цвета в результате \MakeUppercase, http://tex.stackexchange.com/a/187930/79756 , \DeclareRobustCommand protects \todo from expanding inside \MakeUppercase
\AtBeginDocument{%
    \setlength{\parindent}{2.5em}                   % Абзацный отступ. Должен быть одинаковым по всему тексту и равен пяти знакам (ГОСТ Р 7.0.11-2011, 5.3.7).
}

%%% Списки %%%
% Используем короткое тире (endash) для ненумерованных списков (ГОСТ 2.105-95, пункт 4.1.7, требует дефиса, но так лучше смотрится)
\renewcommand{\labelitemi}{\normalfont\bfseries{--}}

%% Перечисление строчными буквами латинского алфавита (ГОСТ 2.105-95, 4.1.7)
\renewcommand{\theenumi}{\Alph{enumi}} % первый уровень иерархии %латинскийалфавит заглавные
\renewcommand{\labelenumi}{\theenumi.} 
%\renewcommand{\theenumii}{\alph{enumii}} % второй уровень иерархии %латинскийалфавит
%\renewcommand{\labelenumii}{\theenumii)} 
%
%
%% Перечисление строчными буквами русского алфавита (ГОСТ 2.105-95, 4.1.7)
\makeatletter
\AddEnumerateCounter{\asbuk}{\russian@alph}{щ}      % Управляем списками/перечислениями через пакет enumitem, а он 'не знает' про asbuk, потому 'учим' его
\makeatother
%%\renewcommand{\theenumi}{\asbuk{enumi}} %первый уровень нумерации
%%\renewcommand{\labelenumi}{\theenumi)} %первый уровень нумерации 
%%\renewcommand{\theenumii}{\asbuk{enumii}} %второй уровень нумерации
%%\renewcommand{\labelenumii}{\theenumii)} %второй уровень нумерации 
\renewcommand{\theenumii}{\arabic{enumii}} %второй уровень нумерации %арабские цифры
\renewcommand{\labelenumii}{\theenumii.} %второй уровень нумерации 
%\renewcommand{\theenumi}{\arabic{enumi}} %первый уровень нумерации %арабские цифры
%\renewcommand{\labelenumi}{\theenumi)} %первый уровень нумерации 
%
%\renewcommand{\theenumiii}{\asbuk{enumiii}} %третий уровень нумерации %русский алфавит
\renewcommand{\theenumiii}{\alph{enumiii}} %третий уровень нумерации %английский алфавит
\renewcommand{\labelenumiii}{\theenumiii)} %третий уровень нумерации 
%\renewcommand{\theenumiii}{\arabic{enumiii}} %третий уровень нумерации %арабские цифры
%\renewcommand{\labelenumiii}{\theenumiii)} %третий уровень нумерации 



\setlist{nosep,%                                    % Единый стиль для всех списков (пакет enumitem), без дополнительных интервалов.
    labelindent=\parindent,leftmargin=*%            % Каждый пункт, подпункт и перечисление записывают с абзацного отступа (ГОСТ 2.105-95, 4.1.8)
}



% asm packages required! In particular amsthm
%http://tex.stackexchange.com/questions/37472/spacing-before-and-after-with-newtheoremstyle

%theoremstyle{}
%plain : italic text, extra space above and below;
%definition : upright text, extra space above and below;
%remark : upright text, no extra space above or below.

\newtheoremstyle{myplain} %
{0} %space above
{0} %space below
{\itshape}
{\parindent}
{\bfseries}
{.}
{.5em}
{}

\newtheoremstyle{mydefinition} %
{0} %space above
{0} %space below
{}
{\parindent}
{\bfseries}
{.}
{.5em}
{}

\theoremstyle{myplain} %improved plain style
\newtheorem{m-theorem}{Теорема}[chapter] % reset theorem numbering for each chapter
\newtheorem{m-corollary}{Следствие}[chapter] % definition numbers are 
\newtheorem{m-proposition}{Утверждение}[chapter] % definition numbers are dependent on theorem numbers
\newtheorem{m-lemma}{Лемма}[chapter]
\newtheorem{m-axiom}{Аксиома}[chapter]

\theoremstyle{mydefinition} %improved definition style
\newtheorem{m-example}{Пример}[chapter] % same for example numbers
\newtheorem{m-definition}{Определение}[chapter]  % definition numbers
\newtheorem{m-condition}{Условие}[chapter]
\newtheorem{m-problem}{Проблема}[chapter]
\newtheorem{m-exercise}{Упражнение}[chapter]
\newtheorem{m-question}{Вопрос}[chapter]
\newtheorem{m-hypothesis}{Гипотеза}[chapter]
\newtheorem{m-task}{Задание}[chapter]



%%control skip of thm, plain style - ANOTHER VARIANT
%%http://tex.stackexchange.com/questions/85400/how-to-change-space-around-theorem-environments
%\makeatletter
%\def\thm@space@setup{%
%	\thm@preskip=0cm %
%	%	\thm@preskip=0cm plus 0.2cm minus 0.2cm
%	\thm@postskip=0cm % or whatever, if you don't want them to be equal
%	%	\thm@postskip=\thm@preskip % or whatever, if you don't want them to be equal
%}
%\makeatother
    
%%% Изображения %%%
\graphicspath{{images/}{Dissertation/images/}}         % Пути к изображениям

%%% Макет страницы %%%
% Выставляем значения полей (ГОСТ 7.0.11-2011, 5.3.7)
\makeatletter
\geometry{a4paper,top=2cm,bottom=2cm,left=3cm,right=1cm,
	headsep=0.5cm, %отступ от колонтитула от живописного поля
	head=1cm, %верхняя граница колонтитула
	headheight=1cm,
	nofoot,
%includefoot,
	nomarginpar
%	,showframe
} 
%\setlength{\topskip}{0pt}   %размер дополнительного верхнего поля
\makeatother

%%% Интервалы %%%
%% По ГОСТ Р 7.0.11-2011, пункту 5.3.6 требуется полуторный интервал
%% Реализация средствами класса (на основе setspace) ближе к типографской классике.
%% И правит сразу и в таблицах (если со звёздочкой) 
%\DoubleSpacing*     % Двойной интервал
\OnehalfSpacing*    % Полуторный интервал % * to force it in the floats
%\setSpacing{1.42}   % Полуторный интервал, подобный Ворду (возможно, стоит включать вместе с предыдущей строкой)

%https://tex.stackexchange.com/questions/65849/confusion-onehalfspacing-vs-spacing-vs-word-vs-the-world/276516#276516
%https://tex.stackexchange.com/questions/13742/what-does-double-spacing-mean
%https://tex.stackexchange.com/questions/30073/why-is-the-linespread-factor-as-it-is/30114#30114

%A possible definition of \onehalfspacing and \doublespacing is that the ratio between font size and \baselineskip should be 1.5 resp. 2.....
%\baselineskip (vertical skip between the base lines of two successive lines of type) of XXpt. 


%%% Выравнивание и переносы %%%
%% http://tex.stackexchange.com/questions/241343/what-is-the-meaning-of-fussy-sloppy-emergencystretch-tolerance-hbadness
%% http://www.latex-community.org/forum/viewtopic.php?p=70342#p70342
\tolerance 1414
\hbadness 1414
\emergencystretch 1.5em % В случае проблем регулировать в первую очередь
\hfuzz 0.3pt
\vfuzz \hfuzz
%\raggedbottom
%\sloppy                 % Избавляемся от переполнений
\clubpenalty=10000      % Запрещаем разрыв страницы после первой строки абзаца
\widowpenalty=10000     % Запрещаем разрыв страницы после последней строки абзаца

%%% Блок управления параметрами для выравнивания заголовков в тексте %%%
\newlength{\otstuplen}
\setlength{\otstuplen}{\theotstup\parindent}
\ifnumequal{\value{headingalign}}{0}{% выравнивание заголовков в тексте
    \newcommand{\hdngalign}{\centering}                % по центру
    \newcommand{\hdngaligni}{}% по центру
    \setlength{\otstuplen}{0pt}
}{%
    \newcommand{\hdngalign}{}                 % по левому краю
    \newcommand{\hdngaligni}{\hspace{\otstuplen}}      % по левому краю
} % В обоих случаях вроде бы без переноса, как и надо (ГОСТ Р 7.0.11-2011, 5.3.5)







%%% Оглавление %%%

\renewcommand{\cftchapterdotsep}{\cftdotsep}                % отбивка точками до номера страницы начала главы/раздела



%% снятие жирности %%
%\cftKleader defines the leader between the title and the page number; it can be
%changed by \renewcommand. The spacing between any dots in the leader is controlled
%by \cftKdotsep
\makeatletter
\renewcommand{\cftchapterpagefont}{\normalfont}        % нежирные номера страниц у глав в оглавлении
\renewcommand{\cftchapterleader}{\cftdotfill{\cftchapterdotsep}}% нежирные точки до номеров страниц у глав в оглавлении
\renewcommand{\cftchapterfont}{}                       % нежирные названия глав в оглавлении
\renewcommand{\cftchapterpagefont}{}                       % нежирные названия номеров глав в оглавлении
\makeatother


%% Форматирование SPbPU %%
% Варианты форматирования
%https://tex.stackexchange.com/questions/394227/memoir-toc-indent-the-second-line-by-numberspace-width-in-the-previous-line-or



%% работа с расстояниями между точками, переносами слов
\makeatletter
\renewcommand{\cftdotsep}{0.1}
%\renewcommand{\@dotset}{0.1} %old macro DOES NOT WORK
\setpnumwidth{2.84538em} %2.84538em = 1cm  
%\renewcommand{\@pnumwidth}{0em} %old macro
%%\setrmarg > \setpnumwidth !!!
\setrmarg{2.84539em}
%set large number to restrict hyphenation
%plus1fil makes the distance between the words smaller!
%it helps to make the equal indent
\makeatother

%\usepackage{tocloft}    % tocloft for table of contents style

%% отступы %%
\makeatletter
\renewcommand{\cftchapterbreak}{}        % set a page break before rather than after the entry
%\renewcommand{\cftparskip}{10em} % эта настройка не работает, вместо неё изменен \parskip непостредственно перед \tableofcontents
\setlength{\cftbeforechapterskip}{0pt plus 0pt} %delete skip after chapter block (last section) %%%<-SPbPU pure
\setlength{\cftbeforepartskip}{0pt plus 0pt} %delete skip after chapter block (last section) %%%<-SPbPU pure
\makeatother



%% Продолжение редактирования оглавления настройками CandDoctDiss %%		


\ifnumgreater{\value{headingdelim}}{0}{%
	%<- SPbPU точка после номера страницы
	\renewcommand\cftchapteraftersnum{.\space}       % добавляет точку с пробелом после номера раздела в оглавлении
	%\renewcommand\cftchapteraftersnumb{\enspace\textperiodcentered\enspace} %\enspace - настоящий пробел, \space не работает
	%\renewcommand\chapternumberlinebox[2]{#2}
}{}
\ifnumgreater{\value{headingdelim}}{1}{%%%<-SPbPU 
	%	   	\renewcommand\cftsectionpresnum{..}       % добавляет smth перед number %выравнивает в box
	% точка после номера страницы
	\renewcommand\cftsectionaftersnum{.\space}       % добавляет точку с пробелом после номера подраздела в оглавлении
	% last is \hfil !
	%   	\renewcommand\cftsectionaftersnumb{...}       % добавляет точки перед названием %можно удалить пробел
	\renewcommand\cftsubsectionaftersnum{.\space}    % добавляет точку с пробелом после номера подподраздела в оглавлении
	\renewcommand\cftsubsubsectionaftersnum{.\space} % добавляет точку с пробелом после номера подподподраздела в оглавлении
	\AtBeginDocument{% без этого polyglossia сама всё переопределяет
		\setsecnumformat{\csname the#1\endcsname.\space}
		%\setsecnumformat{\csname the#1\endcsname:\quad}
	}
}{%
	\AtBeginDocument{% без этого polyglossia сама всё переопределяет
		\setsecnumformat{\csname the#1\endcsname\quad} %
	}
}

\renewcommand*{\cftappendixname}{\appendixname\space} % Слово Приложение в оглавлении


%%% Различные варианты форматирования SPbPU %%%

%% форматирование отсупов до номеров страниц стр. 151 мемуара !!!
%\renewcommand*{\cftsectionnumwidth}{} %выставление абсолютного значения
%\addtolength{\cftsectionnumwidth}{5em} %не работает

%убираем фиксированные размеры of the box %%%<-SPbPU pure
\AtBeginDocument{%
\renewcommand\numberlinebox[2]{#2} % for sections %%%<-SPbPU pure
\renewcommand\chapternumberlinebox[2]{#2} % for chapters 
%\newcommand\chapternumberlinebox[2]{%
%	\hb@xt@#1{#2\hfil}}
%
%\newcommand\chapternumberlinebox[2]{%
%	#1{\hfil#2}}

%\numberlinebox{hlengthi}{hcodei} %выставление абсолютного значения
%\chapternumberlinebox{hlengthi}{hcodei} %выставление абсолютного значения
}

%убираем растояния до \cftsectionpresnum в размере одного абзацного отступа %%%<-SPbPU pure
%\cftsetindents{hkindi}{hindenti}{hnumwidthi}


%https://tex.stackexchange.com/questions/306851/multiline-unnumbered-chapter-in-table-of-contents
%https://tex.stackexchange.com/questions/40022/customized-table-of-contents-same-indentation-for-every-line-of-multi-line-titl
%\parindent % standart padding
% это здорово экономит место, но нужно тогда синхронизировать стиль обычных отступов в перечислениях
% недостаток - не видно выравнивания по первому слову в названии предыдущего раздела
\AtBeginDocument{
	\cftsetindents{chapter}{0pt}{% первая строка
		-0.05em} % последующие строки от первой
	\cftsetindents{section}{%
		0pt
%3.5ex plus 1ex minus .2ex
	}{%
		\parindent
%2.3ex plus .2ex
}
	\cftsetindents{subsection}{%
	0pt}{% 
		\parindent}
	\cftsetindents{subsubsection}{%
		0pt}{% 
		\parindent}
}



%%% Колонтитулы %%%
% Порядковый номер страницы печатают на середине верхнего поля страницы (ГОСТ Р 7.0.11-2011, 5.3.8)
%сделаем справа внизу
%\makeatletter
\makeevenhead{plain}{}{}{\thepage}
\makeoddhead{plain}{}{}{\thepage}
\makeevenfoot{plain}{}{}{}
\makeoddfoot{plain}{}{}{}
\pagestyle{plain}

%%% добавить Стр. над номерами страниц в оглавлении
%%% http://tex.stackexchange.com/a/306950
\newif\ifendTOC
%
\newcommand*{\tocheader}{
%\ifnumequal{\value{pgnum}}{1}{%
%    \ifendTOC\else\hbox to \linewidth%
%      {\noindent{}~\hfill{Pages}}\par%
%      \ifnumless{\value{page}}{3}{}{%
%        \vspace{0.5\onelineskip}
%      }
%      \afterpage{\tocheader}
%    \fi%
%}{}%
}%


%epigraph with DOI
%\usepackage{quotchap}




%%% SPbPU %%% Оформление заголовков глав, разделов, подразделов %%%

\newcommand{\printTheAbstract}{%распечатать the Abstract
	\begingroup
	\par
	\renewcommand{\cleardoublepage}{}
	\renewcommand{\clearpage}{}
	\vspace{\theintvl\curtextsize}
	\chapter*{Abstract}
	\endgroup %after chapter in case of inline using
	\thispagestyle{empty}%
}


\makechapterstyle{SPbPUstyle}{%
	\chapterstyle{default}
	\setlength{\beforechapskip}{0pt}
	\setlength{\midchapskip}{0pt} 
	\setlength{\afterchapskip}{\intvlS\curtextsize}
	\renewcommand*{\chapnamefont}{\normalfont\bfseries\MakeTextUppercase} %не используется слово <<Глава>>
	\renewcommand*{\chapnumfont}{\normalfont\bfseries\MakeTextUppercase}
%	\renewcommand*{\chaptitlefont}{\normalfont\bfseries\MakeTextUppercase} %не работает \MakeTextUppercase
	\renewcommand\printchaptertitle{\normalfont\bfseries\MakeTextUppercase}% аналог \chaptitlefont
	\renewcommand*{\chapterheadstart}{}
	\ifnumgreater{\value{headingdelim}}{0}{%
		\renewcommand*{\afterchapternum}{.\space}   % добавляет точку с пробелом после номера раздела
	}{%
		\renewcommand*{\afterchapternum}{.\quad}     % добавляет точку и \space (\quad) после номера раздела
	} % настройки добавление в СОДЕРЖАНИЕ (по умолчанию название раздела переходит само)
	\renewcommand*{\printchapternum}{\hdngaligni\hdngalign\chapnumfont \thechapter}
	\renewcommand*{\printchaptername}{}
	\renewcommand*{\printchapternonum}{\hdngaligni\hdngalign}
	}
\newcommand{\chapterLight}{\normalfont\MakeTextUppercase\normalsize} %не менять последовательность команд!
\renewcommand*{\printtoctitle}[1]{\normalfont\MakeTextUppercase #1} %слово <<Content>> в стилю chaperLight, по факту убираем \bfseries
%\chapterLight не действует в этой команде
\makeatletter


\makechapterstyle{SPbPUstylechapname}{% для <<будет вписано слово Глава перед каждым номером раздела в оглавлении>>
	\chapterstyle{SPbPUstyle}
	\renewcommand*{\printchapternum}{\chapnumfont \thechapter}
	\renewcommand*{\printchaptername}{\hdngaligni\hdngalign\chapnamefont \@chapapp} %

}
\makeatother

\chapterstyle{SPbPUstyle}

%% удалить перенос на новую строку перед командой \chapter
\newcommand{\delnewpagebeforech}{
	%\begingroup
	\renewcommand{\cleardoublepage}{}
	\renewcommand{\clearpage}{}
	\vspace{\theintvl\curtextsize}
	%\endgroup %after chapter in case of inline using
}

%% Оформление шрифтов и отсупов подразделов, подподразделов и подподподразделов

\makeatletter
\setsecheadstyle{\normalfont\bfseries\hdngalign}
\setsecindent{\otstuplen} %отступ от левого края живописного поля
\setbeforesecskip{\intvlS\curtextsize} %базовые настройки с плюс/минус точностью, что позволяет более гибко располагать рисунки и изображения на странице
\setaftersecskip{\intvlS\curtextsize}


\setsubsecheadstyle{\normalfont\bfseries\itshape\hdngalign}
\setsubsecindent{\otstuplen}
\setbeforesubsecskip{1\curtextsize}
\setaftersubsecskip{1\curtextsize}

\setsubsubsecheadstyle{\normalfont\itshape\hdngalign}
\setsubsubsecindent{\otstuplen}
\setbeforesubsubsecskip{1\curtextsize}
\setaftersubsubsecskip{1\curtextsize}

%ОLD  ГИА
%\setsubsecheadstyle{\normalfont\hdngalign}
%\setsubsecindent{\otstuplen}
%\setbeforesubsecskip{\intvlS\curtextsize}
%\setaftersubsecskip{\intvlS\curtextsize}
%
%\setsubsubsecheadstyle{\normalfont\hdngalign}
%\setsubsubsecindent{\otstuplen}
%\setbeforesubsubsecskip{\intvlS\curtextsize}
%\setaftersubsubsecskip{\intvlS\curtextsize}
\makeatother

%попытки форматирования \part можно продолжить
%сейчас реализован более простой вариант
\renewcommand{\partnamefont}{\LARGE\MakeTextUppercase}
\renewcommand{\partnumfont}{\LARGE\MakeTextUppercase}
\renewcommand*{\parttitlefont}{\LARGE\MakeTextUppercase}

%[section], чтобы заставить все floats быть до расположиться до окончания подраздела
%\FloatBarrier локальное ограничение, чтобы 
% расставить повсеместно по разделам, то всего лишь подключить [section];
% разрешить до \FloatBarrier размещать foats, то добавить окцию  [above].
\usepackage[above]{placeins} 

\sethangfrom{\noindent #1} %все заголовки подразделов центрируются с учетом номера, как block 

\ifnumequal{\value{chapstyle}}{1}{%
    \chapterstyle{SPbPUstylechapname}
    \renewcommand*{\cftchaptername}{Глава\space} % будет вписано слово Глава перед каждым номером раздела в оглавлении
}{}% вместо Chapter \chaptername

%%% Интервалы между заголовками
%\setbeforesecskip{\theintvl\curtextsize}% Заголовки отделяют от текста сверху и снизу тремя интервалами (ГОСТ Р 7.0.11-2011, 5.3.5).
%\setaftersecskip{\theintvl\curtextsize}
%\setbeforesubsecskip{\theintvl\curtextsize}
%\setaftersubsecskip{\theintvl\curtextsize}
%\setbeforesubsubsecskip{\theintvl\curtextsize}
%\setaftersubsubsecskip{\theintvl\curtextsize}


%%% Блок дополнительного управления размерами заголовков
\ifnumequal{\value{headingsize}}{1}{% Пропорциональные заголовки и базовый шрифт 14 пт
	\renewcommand{\normalfont}{\large\bfseries}
	\renewcommand*{\chapnamefont}{\Large\bfseries}
	\renewcommand*{\chapnumfont}{\Large\bfseries}
	\renewcommand*{\chaptitlefont}{\Large\bfseries}
}{}




% ОФОРМЛЕНИЕ Приложений Appendix - Вариант 2 - действующий
%https://stackoverflow.com/questions/717316/how-to-make-appendix-appear-in-toc-in-latex
\makeatletter
\newcommand\appendix@chapter[1]{%
	\renewcommand*{\chapnamefont}{\normalfont\normalsize\bfseries} %не используется слово <<Глава>>
	\renewcommand*{\chapnumfont}{\normalfont\normalsize\bfseries}
	\renewcommand\printchaptertitle{\normalfont\normalsize\bfseries}
	\renewcommand*{\printchapternum}{\chapnumfont \thechapter}
	\renewcommand*{\printchaptername}{\hdngaligni\hdngalign\chapnamefont \@chapapp} %
	\renewcommand*\thechapter{\arabic{chapter}} % работает
	\settocdepth{chapter} % выводить только названия Приложений
	\refstepcounter{chapter}%
	\def\app@ct{\hfill{}\appendixname{} {}\@arabic\c@chapter %
%	\vspace{\intvlS\curtextsize}
	\newline #1
	\vspace{\curtextsize}
}
	\orig@chapter*{\app@ct}%
	\addcontentsline{toc}{chapter}{\appendixname{} \@arabic\c@chapter. #1}%\app@ct % input to TOC-table
}
\let\orig@chapter\chapter
\g@addto@macro\appendix{\newpage\let\chapter\appendix@chapter\renewcommand*{\afterchapternum}{\par\nobreak\vskip \midchapskip}}
\makeatother


%https://tex.stackexchange.com/questions/250834/dont-break-page-for-new-chapter-unless-chapter-heading-wont-fully-fit-on-curre
\newcommand{\ContinueChapterBegin}{%
\let\clearpage\relax
\renewcommand*{\chapterheadstart}{%
	\FloatBarrier % make floats stop
\par
\ifartopt % если сверху сраницы, то
% ничего не делать
\else % в противном случае
\vspace{\theintvl\curtextsize} % добавить интервал
\fi
}
}%

\newcommand{\ContinueChapterEnd}{%
	\let\clearpage\newpage
\renewcommand*{\chapterheadstart}{% ничего не делаем
\FloatBarrier % make floats stop
}
}%

\newcommand{\NewPage}{% в случае, если на последней странице приложения есть <<висячая>> таблица или рисунок
\newpage\leavevmode\thispagestyle{empty}\newpage %начать новое приложение с новой страницы 
}%




\makeatletter %настройка отображения floates
\setlength{\@fptop}{0pt}%отключить вертикальное центрирование рисунка/таблицы на странице
%\setlength{\@fpsep}{8pt}%отключить вертикальное центрирование рисунка/таблицы на странице
%\setlength{\@fpbot}{0pt plus 1fil}%отключить вертикальное центрирование рисунка на странице
\makeatother



%%% Счётчики %%%

%% DOI
\newcounter{mychapternumber} 
\newcounter{chapterDOI}

%% Упрощённые настройки шаблона диссертации: нумерация формул, таблиц, рисунков
\ifnumequal{\value{contnumeq}}{1}{%
    \counterwithout{equation}{chapter} % Убираем связанность номера формулы с номером главы/раздела
}{}
\ifnumequal{\value{contnumfig}}{1}{%
    \counterwithout{figure}{chapter}   % Убираем связанность номера рисунка с номером главы/раздела
}{}
\ifnumequal{\value{contnumtab}}{1}{%
    \counterwithout{table}{chapter}    % Убираем связанность номера таблицы с номером главы/раздела
}{}


%%%http://www.linux.org.ru/forum/general/6993203#comment-6994589 (используется totcount)
\makeatletter
\def\formbytotal#1#2#3#4#5{%
    \newcount\@c
    \@c\totvalue{#1}\relax
    \newcount\@last
    \newcount\@pnul
    \@last\@c\relax
    \divide\@last 10
    \@pnul\@last\relax
    \divide\@pnul 10
    \multiply\@pnul-10
    \advance\@pnul\@last
    \multiply\@last-10
    \advance\@last\@c
    \total{#1}~#2%
    \ifnum\@pnul=1#5\else%
    \ifcase\@last#5\or#3\or#4\or#4\or#4\else#5\fi
    \fi
}
\makeatother

% xassoccnt to make total values: tables, figures, chapters 
%https://tex.stackexchange.com/questions/295857/calculate-amount-of-figures?noredirect=1
\NewTotalDocumentCounter{mytotalfigures}
\NewTotalDocumentCounter{mytotalfiguresInApp}
\NewTotalDocumentCounter{mytotaltables}
\NewTotalDocumentCounter{mytotaltablesInApp}
\NewTotalDocumentCounter{myappendices}
\DeclareAssociatedCounters{figure}{mytotalfigures}
\DeclareAssociatedCounters{table}{mytotaltables}

%https://tex.stackexchange.com/questions/317434/mytotal-pages-number-warning-and-miscalculated
%\NewTotalDocumentCounter{mytotalpages} % not supported yet in xassoccnt, use totpages package
%\DeclareAssociatedCounters{page}{mytotalpages}

%счетчики для вывода на печать
\newcounter{mypages} % счетчик 
\setcounter{mypages}{0} % 
\newcounter{mytotalpagesInApp} % cчетчик 
\setcounter{mytotalpagesInApp}{0} %
\newcounter{myfigures} % счетчик 
\setcounter{myfigures}{0} % 
\newcounter{mytables} % счетчик 
\setcounter{mytables}{0} %  




\AtBeginDocument{
	%% регистрируем счётчики в системе totcounter
	%% позволяет использовать: 
	%% 1) команду \total{counter} для печати значения
	%% 2) спрягать значения слов с помощью \formbytotal
	\regtotcounter{mypages}      % simple counter
	\regtotcounter{TotPages}     % totpages package
	\regtotcounter{myfigures}      % simple counter
	\regtotcounter{mytotalfigures} % xassoccnt package
	\regtotcounter{mytables}      % simple counter
	\regtotcounter{mytotaltables} % xassoccnt package
	\regtotcounter{myappendices}  % xassoccnt package
}
\newtotcounter{citenum} %счетчик для библиографии из totcount package


\preto\appendix{% когда будет команда \appendix 
	% см. также выше переопределение chapter для appendix
	%% Сохранение сумм: рисунки, таблицы, страницы.
	\setcounter{mytotalpagesInApp}{\value{TotPages}}% 
	% count total values
	\AddAssociatedCounters{figure}{mytotalfiguresInApp}
	\AddAssociatedCounters{table}{mytotaltablesInApp}
	\AddAssociatedCounters{chapter}{myappendices}
	%% Форматирование
	%\renewcommand\thechapter{\arabic{chapter}} % см. переопределение chapter для appendix
	\renewcommand{\appendixname}{Приложение} %
	\renewcommand{\thetable}{П\thechapter.\arabic{table}}
	\renewcommand{\thefigure}{П\thechapter.\arabic{figure}}
	\renewcommand{\theequation}{П\thechapter.\arabic{equation}}
	\renewcommand{\thesection}{П\thechapter.\arabic{section}}
	\renewcommand{\thesubsection}{\thesection.\arabic{subsection}}
	\renewcommand{\thesubsubsection}{\thesubsection.\arabic{subsubsection}}
	\counterwithin{footnote}{chapter} %связанная нумерация глав-сносок
	\renewcommand{\thefootnote}{П\thechapter.\arabic{footnote}}
}


%%% Подсчет сумм: рисунки, таблицы, страницы
%% Вариант 1 (рабочий)
\AtEndDocument{
	\setcounter{myfigures}{\value{mytotalfigures}}%
	\addtocounter{myfigures}{-\value{mytotalfiguresInApp}}%
	\setcounter{mytables}{\value{mytotaltables}}%
	\addtocounter{mytables}{-\value{mytotaltablesInApp}}%
	\setcounter{mypages}{\value{mytotalpagesInApp}}%
%	\addtocounter{mypages}{-\value{mytotalpagesInApp}}%
}
%% Вариант 2 (для отладки)
%% работает только в месте вывода на экран суммы, т.е. в реферате
%\setcounter{myfigures}{\numexpr\TotalValue{mytotalfigures}-\TotalValue{mytotalfiguresInApp}\relax}



%%% Правильная нумерация приложений %%%
%% По ГОСТ 2.105, п. 4.3.8 Приложения обозначают заглавными буквами русского алфавита,
%% начиная с А, за исключением букв Ё, З, Й, О, Ч, Ь, Ы, Ъ.
%% Здесь также переделаны все нумерации русскими буквами.
%\ifxetexorluatex
%    \makeatletter
%    \def\russian@Alph#1{\ifcase#1\or
%       А\or Б\or В\or Г\or Д\or Е\or Ж\or
%       И\or К\or Л\or М\or Н\or
%       П\or Р\or С\or Т\or У\or Ф\or Х\or
%       Ц\or Ш\or Щ\or Э\or Ю\or Я\else\xpg@ill@value{#1}{russian@Alph}\fi}
%    \def\russian@alph#1{\ifcase#1\or
%       а\or б\or в\or г\or д\or е\or ж\or
%       и\or к\or л\or м\or н\or
%       п\or р\or с\or т\or у\or ф\or х\or
%       ц\or ш\or щ\or э\or ю\or я\else\xpg@ill@value{#1}{russian@alph}\fi}
%    \makeatother
%\else
%    \makeatletter
%    \if@uni@ode
%      \def\russian@Alph#1{\ifcase#1\or
%        А\or Б\or В\or Г\or Д\or Е\or Ж\or
%        И\or К\or Л\or М\or Н\or
%        П\or Р\or С\or Т\or У\or Ф\or Х\or
%        Ц\or Ш\or Щ\or Э\or Ю\or Я\else\@ctrerr\fi}
%    \else
%      \def\russian@Alph#1{\ifcase#1\or
%        \CYRA\or\CYRB\or\CYRV\or\CYRG\or\CYRD\or\CYRE\or\CYRZH\or
%        \CYRI\or\CYRK\or\CYRL\or\CYRM\or\CYRN\or
%        \CYRP\or\CYRR\or\CYRS\or\CYRT\or\CYRU\or\CYRF\or\CYRH\or
%        \CYRC\or\CYRSH\or\CYRSHCH\or\CYREREV\or\CYRYU\or
%        \CYRYA\else\@ctrerr\fi}
%    \fi
%    \if@uni@ode
%      \def\russian@alph#1{\ifcase#1\or
%        а\or б\or в\or г\or д\or е\or ж\or
%        и\or к\or л\or м\or н\or
%        п\or р\or с\or т\or у\or ф\or х\or
%        ц\or ш\or щ\or э\or ю\or я\else\@ctrerr\fi}
%    \else
%      \def\russian@alph#1{\ifcase#1\or
%        \cyra\or\cyrb\or\cyrv\or\cyrg\or\cyrd\or\cyre\or\cyrzh\or
%        \cyri\or\cyrk\or\cyrl\or\cyrm\or\cyrn\or
%        \cyrp\or\cyrr\or\cyrs\or\cyrt\or\cyru\or\cyrf\or\cyrh\or
%        \cyrc\or\cyrsh\or\cyrshch\or\cyrerev\or\cyryu\or
%        \cyrya\else\@ctrerr\fi}
%    \fi
%    \makeatother
%\fi


%%% Алгоритмы %%%

%\usepackage[linesnumbered]{algorithm2e}
\usepackage[linesnumbered,vlined,figure,scleft]{algorithm2e}

%% Glogal params %%
%ruled, tworuled, algoruled --- put lines to wrap the caption plus a line at the bottom (top) - one should not use this together with inline captions!
%vlined 										--- instead of begin...end will be lines
%boxed 											--- make a box
% figure 										--- count as Fig. ...


% Settings of caption       --- if one will use \caption{} option 	instead of inline + environment caption.
%\SetAlgoCaptionSeparator{.}
%\SetAlgorithmName{Algorithm}{} %last arg is the title of listing table


% Settings for lines numbers
\SetNlSkip{0em}							% sets the value of the space between the line numbers and the text, by default 1em.
\SetNlSty{normalsize}{\hphantom{0}}{.}%defines how to print line numbers
%\hspace*{5mm} does not work 
\SetAlgoNlRelativeSize{-1}	% sets the relative size of line numbers. By default, line numbers are two size smaller than algorithm text

% How to ignore line nuber and to wrap
%http://tex.stackexchange.com/questions/153646/algorithm2e-disabling-line-numbers-for-specific-lines
%http://tex.stackexchange.com/questions/86580/how-to-adjust-line-numbers-of-algorithm2e-package
\makeatletter
%\newcommand{\nosemic}{\renewcommand{\@endalgocfline}{\relax}}% Drop semi-colon ;
%\newcommand{\dosemic}{\renewcommand{\@endalgocfline}{\algocf@endline}}% Reinstate semi-colon ;
%\newcommand{\pushline}{\Indp}% Indent
%\newcommand{\popline}{\Indm\dosemic}% Undent
\let\oldnl\nl% Store \nl in \oldnl
\newcommand{\nonl}{\renewcommand{\nl}{\let\nl\oldnl}}% Remove line number for one line
\makeatother


% Settings for vlines 			
%\SetInd{0.3em}{0.5em}			%default and spaces before and after are 0.5em and 1em
%\SetVlineSkip{5em}					% Sets the value of the vertical space after the little horizontal line which closes a block in vlined mode

%% User abbreviations for ASTRA %%
\SetKwInput{KwInput}{Input}
\SetKwInput{KwOutput}{Output}
%% See also %%
%http://tex.stackexchange.com/questions/145114/caption-below-boxed-algorithm2e-when-used-as-a-figure
%http://tex.stackexchange.com/questions/83536/align-comments-in-algorithm-with-package-algorithm2e



           
% для вертикального центрирования ячеек в tabulary
\def\zz{\ifx\[$\else\aftergroup\zzz\fi}
%$ \] % <-- чиним подсветку синтаксиса в некоторых редакторах
\def\zzz{\setbox0\lastbox
\dimen0\dimexpr\extrarowheight + \ht0-\dp0\relax
\setbox0\hbox{\raise-.5\dimen0\box0}%
\ht0=\dimexpr\ht0+\extrarowheight\relax
\dp0=\dimexpr\dp0+\extrarowheight\relax 
\box0
}



\lstdefinelanguage{Renhanced}%
{keywords={abbreviate,abline,abs,acos,acosh,action,add1,add,%
        aggregate,alias,Alias,alist,all,anova,any,aov,aperm,append,apply,%
        approx,approxfun,apropos,Arg,args,array,arrows,as,asin,asinh,%
        atan,atan2,atanh,attach,attr,attributes,autoload,autoloader,ave,%
        axis,backsolve,barplot,basename,besselI,besselJ,besselK,besselY,%
        beta,binomial,body,box,boxplot,break,browser,bug,builtins,bxp,by,%
        c,C,call,Call,case,cat,category,cbind,ceiling,character,char,%
        charmatch,check,chol,chol2inv,choose,chull,class,close,cm,codes,%
        coef,coefficients,co,col,colnames,colors,colours,commandArgs,%
        comment,complete,complex,conflicts,Conj,contents,contour,%
        contrasts,contr,control,helmert,contrib,convolve,cooks,coords,%
        distance,coplot,cor,cos,cosh,count,fields,cov,covratio,wt,CRAN,%
        create,crossprod,cummax,cummin,cumprod,cumsum,curve,cut,cycle,D,%
        data,dataentry,date,dbeta,dbinom,dcauchy,dchisq,de,debug,%
        debugger,Defunct,default,delay,delete,deltat,demo,de,density,%
        deparse,dependencies,Deprecated,deriv,description,detach,%
        dev2bitmap,dev,cur,deviance,off,prev,,dexp,df,dfbetas,dffits,%
        dgamma,dgeom,dget,dhyper,diag,diff,digamma,dim,dimnames,dir,%
        dirname,dlnorm,dlogis,dnbinom,dnchisq,dnorm,do,dotplot,double,%
        download,dpois,dput,drop,drop1,dsignrank,dt,dummy,dump,dunif,%
        duplicated,dweibull,dwilcox,dyn,edit,eff,effects,eigen,else,%
        emacs,end,environment,env,erase,eval,equal,evalq,example,exists,%
        exit,exp,expand,expression,External,extract,extractAIC,factor,%
        fail,family,fft,file,filled,find,fitted,fivenum,fix,floor,for,%
        For,formals,format,formatC,formula,Fortran,forwardsolve,frame,%
        frequency,ftable,ftable2table,function,gamma,Gamma,gammaCody,%
        gaussian,gc,gcinfo,gctorture,get,getenv,geterrmessage,getOption,%
        getwd,gl,glm,globalenv,gnome,GNOME,graphics,gray,grep,grey,grid,%
        gsub,hasTsp,hat,heat,help,hist,home,hsv,httpclient,I,identify,if,%
        ifelse,Im,image,\%in\%,index,influence,measures,inherits,install,%
        installed,integer,interaction,interactive,Internal,intersect,%
        inverse,invisible,IQR,is,jitter,kappa,kronecker,labels,lapply,%
        layout,lbeta,lchoose,lcm,legend,length,levels,lgamma,library,%
        licence,license,lines,list,lm,load,local,locator,log,log10,log1p,%
        log2,logical,loglin,lower,lowess,ls,lsfit,lsf,ls,machine,Machine,%
        mad,mahalanobis,make,link,margin,match,Math,matlines,mat,matplot,%
        matpoints,matrix,max,mean,median,memory,menu,merge,methods,min,%
        missing,Mod,mode,model,response,mosaicplot,mtext,mvfft,na,nan,%
        names,omit,nargs,nchar,ncol,NCOL,new,next,NextMethod,nextn,%
        nlevels,nlm,noquote,NotYetImplemented,NotYetUsed,nrow,NROW,null,%
        numeric,\%o\%,objects,offset,old,on,Ops,optim,optimise,optimize,%
        options,or,order,ordered,outer,package,packages,page,pairlist,%
        pairs,palette,panel,par,parent,parse,paste,path,pbeta,pbinom,%
        pcauchy,pchisq,pentagamma,persp,pexp,pf,pgamma,pgeom,phyper,pico,%
        pictex,piechart,Platform,plnorm,plogis,plot,pmatch,pmax,pmin,%
        pnbinom,pnchisq,pnorm,points,poisson,poly,polygon,polyroot,pos,%
        postscript,power,ppoints,ppois,predict,preplot,pretty,Primitive,%
        print,prmatrix,proc,prod,profile,proj,prompt,prop,provide,%
        psignrank,ps,pt,ptukey,punif,pweibull,pwilcox,q,qbeta,qbinom,%
        qcauchy,qchisq,qexp,qf,qgamma,qgeom,qhyper,qlnorm,qlogis,qnbinom,%
        qnchisq,qnorm,qpois,qqline,qqnorm,qqplot,qr,Q,qty,qy,qsignrank,%
        qt,qtukey,quantile,quasi,quit,qunif,quote,qweibull,qwilcox,%
        rainbow,range,rank,rbeta,rbind,rbinom,rcauchy,rchisq,Re,read,csv,%
        csv2,fwf,readline,socket,real,Recall,rect,reformulate,regexpr,%
        relevel,remove,rep,repeat,replace,replications,report,require,%
        resid,residuals,restart,return,rev,rexp,rf,rgamma,rgb,rgeom,R,%
        rhyper,rle,rlnorm,rlogis,rm,rnbinom,RNGkind,rnorm,round,row,%
        rownames,rowsum,rpois,rsignrank,rstandard,rstudent,rt,rug,runif,%
        rweibull,rwilcox,sample,sapply,save,scale,scan,scan,screen,sd,se,%
        search,searchpaths,segments,seq,sequence,setdiff,setequal,set,%
        setwd,show,sign,signif,sin,single,sinh,sink,solve,sort,source,%
        spline,splinefun,split,sqrt,stars,start,stat,stem,step,stop,%
        storage,strstrheight,stripplot,strsplit,structure,strwidth,sub,%
        subset,substitute,substr,substring,sum,summary,sunflowerplot,svd,%
        sweep,switch,symbol,symbols,symnum,sys,status,system,t,table,%
        tabulate,tan,tanh,tapply,tempfile,terms,terrain,tetragamma,text,%
        time,title,topo,trace,traceback,transform,tri,trigamma,trunc,try,%
        ts,tsp,typeof,unclass,undebug,undoc,union,unique,uniroot,unix,%
        unlink,unlist,unname,untrace,update,upper,url,UseMethod,var,%
        variable,vector,Version,vi,warning,warnings,weighted,weights,%
        which,while,window,write,\%x\%,x11,X11,xedit,xemacs,xinch,xor,%
        xpdrows,xy,xyinch,yinch,zapsmall,zip},%
    otherkeywords={!,!=,~,$,*,\%,\&,\%/\%,\%*\%,\%\%,<-,<<-},%$
    alsoother={._$},%$
    sensitive,%
    morecomment=[l]\#,%
    morestring=[d]",%
    morestring=[d]'% 2001 Robert Denham
}%

%решаем проблему с кириллицей в комментариях (в pdflatex) https://tex.stackexchange.com/a/103712/79756
\lstset{extendedchars=true,literate={Ö}{{\"O}}1
    {Ä}{{\"A}}1
    {Ü}{{\"U}}1
    {ß}{{\ss}}1
    {ü}{{\"u}}1
    {ä}{{\"a}}1
    {ö}{{\"o}}1
    {~}{{\textasciitilde}}1
    {а}{{\selectfont\char224}}1
    {б}{{\selectfont\char225}}1
    {в}{{\selectfont\char226}}1
    {г}{{\selectfont\char227}}1
    {д}{{\selectfont\char228}}1
    {е}{{\selectfont\char229}}1
    {ё}{{\"e}}1
    {ж}{{\selectfont\char230}}1
    {з}{{\selectfont\char231}}1
    {и}{{\selectfont\char232}}1
    {й}{{\selectfont\char233}}1
    {к}{{\selectfont\char234}}1
    {л}{{\selectfont\char235}}1
    {м}{{\selectfont\char236}}1
    {н}{{\selectfont\char237}}1
    {о}{{\selectfont\char238}}1
    {п}{{\selectfont\char239}}1
    {р}{{\selectfont\char240}}1
    {с}{{\selectfont\char241}}1
    {т}{{\selectfont\char242}}1
    {у}{{\selectfont\char243}}1
    {ф}{{\selectfont\char244}}1
    {х}{{\selectfont\char245}}1
    {ц}{{\selectfont\char246}}1
    {ч}{{\selectfont\char247}}1
    {ш}{{\selectfont\char248}}1
    {щ}{{\selectfont\char249}}1
    {ъ}{{\selectfont\char250}}1
    {ы}{{\selectfont\char251}}1
    {ь}{{\selectfont\char252}}1
    {э}{{\selectfont\char253}}1
    {ю}{{\selectfont\char254}}1
    {я}{{\selectfont\char255}}1
    {А}{{\selectfont\char192}}1
    {Б}{{\selectfont\char193}}1
    {В}{{\selectfont\char194}}1
    {Г}{{\selectfont\char195}}1
    {Д}{{\selectfont\char196}}1
    {Е}{{\selectfont\char197}}1
    {Ё}{{\"E}}1
    {Ж}{{\selectfont\char198}}1
    {З}{{\selectfont\char199}}1
    {И}{{\selectfont\char200}}1
    {Й}{{\selectfont\char201}}1
    {К}{{\selectfont\char202}}1
    {Л}{{\selectfont\char203}}1
    {М}{{\selectfont\char204}}1
    {Н}{{\selectfont\char205}}1
    {О}{{\selectfont\char206}}1
    {П}{{\selectfont\char207}}1
    {Р}{{\selectfont\char208}}1
    {С}{{\selectfont\char209}}1
    {Т}{{\selectfont\char210}}1
    {У}{{\selectfont\char211}}1
    {Ф}{{\selectfont\char212}}1
    {Х}{{\selectfont\char213}}1
    {Ц}{{\selectfont\char214}}1
    {Ч}{{\selectfont\char215}}1
    {Ш}{{\selectfont\char216}}1
    {Щ}{{\selectfont\char217}}1
    {Ъ}{{\selectfont\char218}}1
    {Ы}{{\selectfont\char219}}1
    {Ь}{{\selectfont\char220}}1
    {Э}{{\selectfont\char221}}1
    {Ю}{{\selectfont\char222}}1
    {Я}{{\selectfont\char223}}1
    {і}{{\selectfont\char105}}1
    {ї}{{\selectfont\char168}}1
    {є}{{\selectfont\char185}}1
    {ґ}{{\selectfont\char160}}1
    {І}{{\selectfont\char73}}1
    {Ї}{{\selectfont\char136}}1
    {Є}{{\selectfont\char153}}1
    {Ґ}{{\selectfont\char128}}1
}

% Ширина текста минус ширина надписи 999
\newlength{\twless}
\newlength{\lmarg}
\setlength{\lmarg}{\widthof{999}}   % ширина надписи 999
\setlength{\twless}{\textwidth-\lmarg}


\lstset{ %
%    language=R,                     %  Язык указать здесь, если во всех листингах преимущественно один язык, в результате часть настроек может пойти только для этого языка
    numbers=left,                   % where to put the line-numbers
    numberstyle=\fontsize{12pt}{14pt}\selectfont\color{Gray},  % the style that is used for the line-numbers
    firstnumber=2,                  % в этой и следующей строках задаётся поведение нумерации 5, 10, 15...
    stepnumber=5,                   % the step between two line-numbers. If it's 1, each line will be numbered
    numbersep=5pt,                  % how far the line-numbers are from the code
    backgroundcolor=\color{white},  % choose the background color. You must add \usepackage{color}
    showspaces=false,               % show spaces adding particular underscores
    showstringspaces=false,         % underline spaces within strings
    showtabs=false,                 % show tabs within strings adding particular underscores
    frame=leftline,                 % adds a frame of different types around the code
    rulecolor=\color{black},        % if not set, the frame-color may be changed on line-breaks within not-black text (e.g. commens (green here))
    tabsize=2,                      % sets default tabsize to 2 spaces
    captionpos=t,                   % sets the caption-position to top
    breaklines=true,                % sets automatic line breaking
    breakatwhitespace=false,        % sets if automatic breaks should only happen at whitespace
%    title=\lstname,                 % show the filename of files included with \lstinputlisting;
    % also try caption instead of title
    basicstyle=\fontsize{12pt}{14pt}\selectfont\ttfamily,% the size of the fonts that are used for the code
%    keywordstyle=\color{blue},      % keyword style
    commentstyle=\color{ForestGreen}\emph,% comment style
    stringstyle=\color{Mahogany},   % string literal style
    escapeinside={\%*}{*)},         % if you want to add a comment within your code
    morekeywords={*,...},           % if you want to add more keywords to the set
    inputencoding=utf8,             % кодировка кода
    xleftmargin={\lmarg},           % Чтобы весь код и полоска с номерами строк была смещена влево, так чтобы цифры не вылезали за пределы текста слева
} 

%http://tex.stackexchange.com/questions/26872/smaller-frame-with-listings
% Окружение, чтобы листинг был компактнее обведен рамкой, если она задается, а не на всю ширину текста
\makeatletter
\newenvironment{SmallListing}[1][]
{\lstset{#1}\VerbatimEnvironment\begin{VerbatimOut}{VerbEnv.tmp}}
{\end{VerbatimOut}\settowidth\@tempdima{%
        \lstinputlisting{VerbEnv.tmp}}
    \minipage{\@tempdima}\lstinputlisting{VerbEnv.tmp}\endminipage}    
\makeatother


\DefineVerbatimEnvironment% с шрифтом 12 пт
{Verb}{Verbatim}
{fontsize=\fontsize{12pt}{14pt}\selectfont}

\newfloat[chapter]{ListingEnv}{lol}{Листинг}

\captionsetup[ListingEnv]{
    format=tablecaption,
    labelsep=space,                 % Точка после номера листинга задается значением period
    singlelinecheck=off,
    position=top
}

\captionsetup[lstlisting]{
    format=tablecaption,
    labelsep=space,                 % Точка после номера листинга задается значением period
    singlelinecheck=off,
    position=top
}

\renewcommand{\lstlistingname}{Листинг}

%Общие счётчики окружений листингов
%http://tex.stackexchange.com/questions/145546/how-to-make-figure-and-listing-share-their-counter
% Если смешивать плавающие и не плавающие окружения, то могут быть проблемы с нумерацией
\makeatletter
\AtBeginDocument{%
    \let\c@ListingEnv\c@lstlisting
    \let\theListingEnv\thelstlisting
    \let\ftype@lstlisting\ftype@ListingEnv % give the floats the same precedence
}
\makeatother

% значок С++ — используйте команду \cpp
\newcommand{\cpp}{%
    C\nolinebreak\hspace{-.05em}%
    \raisebox{.2ex}{+}\nolinebreak\hspace{-.10em}%
    \raisebox{.2ex}{+}%
}

%%%  Чересстрочное форматирование таблиц
%% http://tex.stackexchange.com/questions/278362/apply-italic-formatting-to-every-other-row
\newcounter{rowcnt}
\newcommand\altshape{\ifnumodd{\value{rowcnt}}{\color{red}}{\vspace*{-1ex}\itshape}}
% \AtBeginEnvironment{tabular}{\setcounter{rowcnt}{1}}
% \AtEndEnvironment{tabular}{\setcounter{rowcnt}{0}}

%%% Ради примера во второй главе
\let\originalepsilon\epsilon
\let\originalphi\phi
\let\originalkappa\kappa
\let\originalle\le
\let\originalleq\leq
\let\originalge\ge
\let\originalgeq\geq
\let\originalemptyset\emptyset
\let\originaltan\tan
\let\originalcot\cot
\let\originalcsc\csc

%%% Русская традиция начертания математических знаков
\renewcommand{\le}{\ensuremath{\leqslant}}
\renewcommand{\leq}{\ensuremath{\leqslant}}
\renewcommand{\ge}{\ensuremath{\geqslant}}
\renewcommand{\geq}{\ensuremath{\geqslant}}
\renewcommand{\emptyset}{\varnothing}

%%% Русская традиция начертания математических функций (на случай копирования из зарубежных источников)
\renewcommand{\tan}{\operatorname{tg}}
\renewcommand{\cot}{\operatorname{ctg}}
\renewcommand{\csc}{\operatorname{cosec}}

%%% Русская традиция начертания греческих букв (греческие буквы вертикальные, через пакет upgreek)
\renewcommand{\epsilon}{\ensuremath{\upvarepsilon}}   %  русская традиция записи
\renewcommand{\phi}{\ensuremath{\upvarphi}}
%\renewcommand{\kappa}{\ensuremath{\varkappa}}
\renewcommand{\alpha}{\upalpha}
\renewcommand{\beta}{\upbeta}
\renewcommand{\gamma}{\upgamma}
\renewcommand{\delta}{\updelta}
\renewcommand{\varepsilon}{\upvarepsilon}
\renewcommand{\zeta}{\upzeta}
\renewcommand{\eta}{\upeta}
\renewcommand{\theta}{\uptheta}
\renewcommand{\vartheta}{\upvartheta}
\renewcommand{\iota}{\upiota}
\renewcommand{\kappa}{\upkappa}
\renewcommand{\lambda}{\uplambda}
\renewcommand{\mu}{\upmu}
\renewcommand{\nu}{\upnu}
\renewcommand{\xi}{\upxi}
\renewcommand{\pi}{\uppi}
\renewcommand{\varpi}{\upvarpi}
\renewcommand{\rho}{\uprho}
%\renewcommand{\varrho}{\upvarrho}
\renewcommand{\sigma}{\upsigma}
%\renewcommand{\varsigma}{\upvarsigma}
\renewcommand{\tau}{\uptau}
\renewcommand{\upsilon}{\upupsilon}
\renewcommand{\varphi}{\upvarphi}
\renewcommand{\chi}{\upchi}
\renewcommand{\psi}{\uppsi}
\renewcommand{\omega}{\upomega}


          
%%% Библиография. Общие настройки для двух способов её подключения %%%


%%% Выбор реализации %%%
\ifnumequal{\value{bibliosel}}{0}{%
    %%% Реализация библиографии встроенными средствами посредством движка bibtex8 %%%

%%% Пакеты %%%
\usepackage{cite}                                   % Красивые ссылки на литературу


%%% Стили %%%
%\bibliographystyle{BibTeX-Styles/utf8gost71u}    % Оформляем библиографию по ГОСТ 7.1 (ГОСТ Р 7.0.11-2011, 5.6.7)
\bibliographystyle{BibTeX-Styles/ugost2008mod}    % Оформляем библиографию по ГОСТ 7.1 (ГОСТ Р 7.0.11-2011, 5.6.7)
%.bst

\makeatletter
\renewcommand{\@biblabel}[1]{#1.}   % Заменяем библиографию с квадратных скобок на точку
\makeatother
%% Управление отступами между записями
%% требует etoolbox 
%% http://tex.stackexchange.com/a/105642
%\patchcmd\thebibliography
% {\labelsep}
% {\labelsep\itemsep=5pt\parsep=0pt\relax}
% {}
% {\typeout{Couldn't patch the command}}

%%% Список литературы с красной строки (без висячего отступа) %%%
%\patchcmd{\thebibliography} %может потребовать включения пакета etoolbox
%  {\advance\leftmargin\labelsep}
%  {\leftmargin=0pt%
%   \setlength{\labelsep}{\widthof{\ }}% Управляет длиной отступа после точки
%   \itemindent=\parindent%
%   \addtolength{\itemindent}{\labelwidth}% Сдвигаем правее на величину номера с точкой
%   \advance\itemindent\labelsep%
%  }
%  {}{}

%%% Цитирование %%%
\renewcommand\citepunct{;\penalty\citepunctpenalty%
    \hskip.13emplus.1emminus.1em\relax}                % Разделение ; при перечислении ссылок (ГОСТ Р 7.0.5-2008)


%%% Создание команд для вывода списка литературы %%%
\newcommand*{\insertbibliofull}{
\bibliography{biblio/othercites,biblio/authorpapersVAK,biblio/authorpapers,biblio/authorconferences}         % Подключаем BibTeX-базы % После запятых не должно быть лишних пробелов — он "думает", что это тоже имя пути
}

\newcommand*{\insertbiblioauthor}{
\bibliography{biblio/authorpapersVAK,biblio/authorpapers,biblio/authorconferences}         % Подключаем BibTeX-базы % После запятых не должно быть лишних пробелов — он "думает", что это тоже имя пути
}

\newcommand*{\insertbiblioother}{
\bibliography{biblio/othercites}         % Подключаем BibTeX-базы
}


%% Счётчик использованных ссылок на литературу, обрабатывающий с учётом неоднократных ссылок
%% Требуется дважды компилировать, поскольку ему нужно считать актуальный внешний файл со списком литературы
\newtotcounter{citenum}
\def\oldcite{}
\let\oldcite=\bibcite
\def\bibcite{\stepcounter{citenum}\oldcite}
  % Встроенная реализация с загрузкой файла через движок bibtex8
}{
    %%% Реализация библиографии пакетами biblatex и biblatex-gost с использованием движка biber %%%

%\usepackage{csquotes} % biblatex рекомендует его подключать. Пакет для оформления сложных блоков цитирования.
%%% Загрузка пакета с основными настройками %%%
\ifnumequal{\value{draft}}{0}{% Чистовик
\usepackage[%
backend=biber,% движок
bibencoding=utf8,% кодировка bib файла
%sorting=none,% настройка сортировки списка литературы
style=gost-numeric,% стиль цитирования и библиографии (по ГОСТ)
%%style=gost-authoryear,% стиль цитирования и библиографии (по ГОСТ)
%%%% выставить следующую опцию <<babel>> и закомментировать <<language=english>> для достижения многоязычных ссылок
babel=other, %выставим для отображения разных языков
%%language=english,%только английский = \setlanguage{}; autobib получение языка из babel/polyglossia, default: autobib % если ставить autocite или auto, то цитаты в тексте с указанием страницы, получат указание страницы на языке оригинала
%%autolang=other,%other многоязычная библиография
%%clearlang=true,% внутренний сброс поля language, если он совпадает с языком из babel/polyglossia
defernumbers=true,% нумерация проставляется после двух компиляций, зато позволяет выцеплять библиографию по ключевым словам и нумеровать не из большего списка
sortcites=true,% сортировать номера затекстовых ссылок при цитировании (если в квадратных скобках несколько ссылок, то отображаться будут отсортированно, а не абы как)
doi=true,% Показывать или нет ссылки на DOI
isbn=false% Показывать или нет ISBN, ISSN, ISRN
]{biblatex}[2016/09/17]%
}{%Черновик
\usepackage[%
backend=biber,% движок
bibencoding=utf8,% кодировка bib файла
sorting=none,% настройка сортировки списка литературы
]{biblatex}[2016/09/17]%
}
%%TO-DO: продумать автозамену всех полей hyphenation на language



%%% Подключение файлов bib %%%
%\addbibresource[label=other]{biblio/othercites.bib}
%\addbibresource[label=vak]{biblio/authorpapersVAK.bib}
%\addbibresource[label=papers]{biblio/authorpapers.bib}
%\addbibresource[label=conf]{biblio/authorconferences.bib}



%http://tex.stackexchange.com/a/141831/79756
%There is a way to automatically map the language field to the langid field. The following lines in the preamble should be enough to do that.
%This command will copy the language field into the  field and will then delete the contents of the language field. The language field will only be deleted if it was successfully copied into the langid field.
\DeclareSourcemap{ %модификация bib файла перед тем, как им займётся biblatex 
    \maps{
    	%% SPbPU
    	%% https://tex.stackexchange.com/a/229505/44348
    	\map{% delete month
    		\step[fieldset=month, null]
    		\step[fieldsource=date,
    		match=\regexp{([0-9]{4})-[0-9]{2}(-[0-9]{2})*},
    		replace=\regexp{$1}$5] % <<$5>> only for syntax highlihgting in IDE
    	}
%    	\map{% set current urldate
%    	\step[fieldset=urldate, null]	
%		\step[fieldset=urldate,fieldvalue={\the\year-\the\month-\the\day}]
%    	} 
%		\map{% не отображаем поле ``Глава''
%			\step[fieldset=chapter, null]
%			\step[fieldset=editor, null]
%		}
%    	} 
		\map{% перекидываем значения полей hyphenation в поля langid, которыми пользуется biblatex
			\step[fieldsource=hyphenation, fieldset=langid, origfieldval, final]
		}
        \map{% перекидываем значения полей language в поля langid, которыми пользуется biblatex
            \step[fieldsource=language, fieldset=langid, origfieldval, final]
            \step[fieldset=language, null]
        }
%        \map[overwrite, refsection=0]{% стираем значения всех полей addendum
%            \perdatasource{biblio/authorpapersVAK.bib}
%            \perdatasource{biblio/authorpapers.bib}
%            \perdatasource{biblio/authorconferences.bib}
%            \step[fieldsource=addendum, final]
%            \step[fieldset=addendum, null] %чтобы избавиться от информации об объёме авторских статей, в отличие от автореферата
%        }
        \map{% перекидываем значения полей numpages в поля pagetotal, которыми пользуется biblatex
            \step[fieldsource=numpages, fieldset=pagetotal, origfieldval, final]
            \step[fieldset=pagestotal, null]
        }
        \map{% если в поле medium написано "Электронный ресурс", то устанавливаем поле media, которым пользуется biblatex, в значение eresource.
            \step[fieldsource=medium,
            match=\regexp{Электронный\s+ресурс},
            final]
            \step[fieldset=media, fieldvalue=eresource]
        }
        \map[overwrite]{% стираем значения всех полей issn
            \step[fieldset=issn, null]
        }
        \map[overwrite]{% стираем значения всех полей abstract, поскольку ими не пользуемся, а там бывают "неприятные" латеху символы
            \step[fieldsource=abstract]
            \step[fieldset=abstract,null]
        }
        \map[overwrite]{ % переделка формата записи даты
            \step[fieldsource=urldate,
            match=\regexp{([0-9]{2})\.([0-9]{2})\.([0-9]{4})},
            replace={$3-$2-$1$4}, %, % $4 вставлен исключительно ради нормальной работы программ подсветки синтаксиса, которые некорректно обрабатывают $ в таких конструкциях
            final]
        }
%        \map[overwrite]{ % добавляем ключевые слова, чтобы различать источники
%            \perdatasource{biblio/othercites.bib}
%            \step[fieldset=keywords, fieldvalue={biblioother,bibliofull}]
%        }
%        \map[overwrite]{ % добавляем ключевые слова, чтобы различать источники
%            \perdatasource{biblio/authorpapersVAK.bib}
%            \step[fieldset=keywords, fieldvalue={biblioauthorvak,biblioauthor,bibliofull}]
%        }
%        \map[overwrite]{ % добавляем ключевые слова, чтобы различать источники
%            \perdatasource{biblio/authorpapers.bib}
%            \step[fieldset=keywords, fieldvalue={biblioauthornotvak,biblioauthor,bibliofull}]
%        }
%        \map[overwrite]{ % добавляем ключевые слова, чтобы различать источники
%            \perdatasource{biblio/authorconferences.bib}
%            \step[fieldset=keywords, fieldvalue={biblioauthorconf,biblioauthor,bibliofull}]
%        }
%        \map[overwrite]{% стираем значения всех полей series
%            \step[fieldset=series, null]
%        }
        \map[overwrite]{% перекидываем значения полей howpublished в поля organization для типа online
            \step[typesource=online, typetarget=online, final]
            \step[fieldsource=howpublished, fieldset=organization, origfieldval]
            \step[fieldset=howpublished, null]
        }
        % Так отключаем [Электронный ресурс]
%        \map[overwrite]{% стираем значения всех полей media=eresource
%            \step[fieldsource=media,
%            match={eresource},
%            final]
%            \step[fieldset=media, null]
%        }
		\map{
			\step[fieldsource=langid, match=russian, final]
			\step[fieldset=presort, fieldvalue={a}]
		}
		\map{
			\step[fieldsource=langid, notmatch=russian, final]
			\step[fieldset=presort, fieldvalue={z}]
		}%    
	}
}


%\DeclareSourcemap{
%	\maps[datatype=bibtex]{
%		\map{
%			\step[fieldsource=langid, match=russian, final]
%			\step[fieldset=presort, fieldvalue={a}]
%		}
%		\map{
%			\step[fieldsource=langid, notmatch=russian, final]
%			\step[fieldset=presort, fieldvalue={z}]
%		}
%	}
%}


%%% Убираем неразрывные пробелы перед двоеточием и точкой с запятой %%%
%\makeatletter
%\ifnumequal{\value{draft}}{0}{% Чистовик
%    \renewcommand*{\addcolondelim}{%
%      \begingroup%
%      \def\abx@colon{%
%        \ifdim\lastkern>\z@\unkern\fi%
%        \abx@puncthook{:}\space}%
%      \addcolon%
%      \endgroup}
%
%    \renewcommand*{\addsemicolondelim}{%
%      \begingroup%
%      \def\abx@semicolon{%
%        \ifdim\lastkern>\z@\unkern\fi%
%        \abx@puncthook{;}\space}%
%      \addsemicolon%
%      \endgroup}
%}{}
%\makeatother

%%% Правка записей типа thesis, чтобы дважды не писался автор
%\ifnumequal{\value{draft}}{0}{% Чистовик
%\DeclareBibliographyDriver{thesis}{%
%  \usebibmacro{bibindex}%
%  \usebibmacro{begentry}%
%  \usebibmacro{heading}%
%  \newunit
%  \usebibmacro{author}%
%  \setunit*{\labelnamepunct}%
%  \usebibmacro{thesistitle}%
%  \setunit{\respdelim}%
%  %\printnames[last-first:full]{author}%Вот эту строчку нужно убрать, чтобы автор диссертации не дублировался
%  \newunit\newblock
%  \printlist[semicolondelim]{specdata}%
%  \newunit
%  \usebibmacro{institution+location+date}%
%  \newunit\newblock
%  \usebibmacro{chapter+pages}%
%  \newunit
%  \printfield{pagetotal}%
%  \newunit\newblock
%  \usebibmacro{doi+eprint+url+note}%
%  \newunit\newblock
%  \usebibmacro{addendum+pubstate}%
%  \setunit{\bibpagerefpunct}\newblock
%  \usebibmacro{pageref}%
%  \newunit\newblock
%  \usebibmacro{related:init}%
%  \usebibmacro{related}%
%  \usebibmacro{finentry}}
%}{}

%\newbibmacro{string+doi}[1]{% новая макрокоманда на простановку ссылки на doi
%    \iffieldundef{doi}{#1}{\href{http://dx.doi.org/\thefield{doi}}{#1}}}

%\ifnumequal{\value{draft}}{0}{% Чистовик
%\renewcommand*{\mkgostheading}[1]{\usebibmacro{string+doi}{#1}} % ссылка на doi с авторов. стоящих впереди записи
%\renewcommand*{\mkgostheading}[1]{#1} % только лишь убираем курсив с авторов
%}{}
%\DeclareFieldFormat{title}{\usebibmacro{string+doi}{#1}} % ссылка на doi с названия работы
%\DeclareFieldFormat{journaltitle}{\usebibmacro{string+doi}{#1}} % ссылка на doi с названия журнала
%%% Убрать тире из разделителей элементов в библиографии:
%\renewcommand*{\newblockpunct}{%
%    \addperiod\space\bibsentence}%block punct.,\bibsentence is for vol,etc.

%%% Возвращаем запись «Режим доступа» %%%
%\DefineBibliographyStrings{english}{%
%    urlfrom = {Mode of access}
%}
%\DeclareFieldFormat{url}{\bibstring{urlfrom}\addcolon\space\url{#1}}

%%%% В списке литературы обозначение одной буквой диапазона страниц англоязычного источника %%%
\DefineBibliographyStrings{english}{%
    pages = {P\adddot} %заглавность буквы затем по месту определяется работой самого biblatex
}

%%% В ссылке на источник в основном тексте с указанием конкретной страницы обозначение одной большой буквой %%%
%\DefineBibliographyStrings{russian}{%
%    page = {C\adddot}
%}

%%% Исправление длины тире в диапазонах %%%
%\DefineBibliographyExtras{russian}{%
%  \protected\def\bibrangedash{%
%    \textendash\penalty\value{abbrvpenalty}}% almost unbreakable dash
%  \protected\def\bibdaterangesep{\bibrangedash}%тире для дат
%}

%Set higher penalty for breaking in number, dates and pages ranges
\setcounter{abbrvpenalty}{10000} % default is \hyphenpenalty which is 12

%Set higher penalty for breaking in names
\setcounter{highnamepenalty}{10000} % If you prefer the traditional BibTeX behavior (no linebreaks at highnamepenalty breakpoints), set it to ‘infinite’ (10 000 or higher).
\setcounter{lownamepenalty}{10000}

%%% Set low penalties for breaks at uppercase letters and lowercase letters
%\setcounter{biburllcpenalty}{500} %управляет разрывами ссылок после маленьких букв RTFM biburllcpenalty
%\setcounter{biburlucpenalty}{3000} %управляет разрывами ссылок после больших букв, RTFM biburlucpenalty

%% Список литературы с красной строки (без висячего отступа) %%%
%\printfield  ----  This command prints a hfieldi using the formatting directive hformati, as defined
%with \DeclareFieldFormat. 
%\printtext   ----  This command prints htexti, which may be printable text or arbitrary code generating
%printable text. It clears the punctuation buffer before inserting htexti and
%informs biblatex that printable text has been inserted.
%https://github.com/odomanov/biblatex-gost/blob/master/tex/latex/biblatex-gost/bbx/gost-numeric.bbx
\defbibenvironment{SSTfirst} % Примерно соответствует 1 варианту оформления списка использованных источников
  {\list
     {
    \printtext[labelnumberwidth]{%
	\printfield{labelprefix}%not numberprefix !
	\printfield{labelnumber}}
	}%
     {%
      \setlength{\labelwidth}{\labelnumberwidth}% This length register indicates the width of the widest labelnumber
      \addtolength{\labelwidth}{6pt}% \labelwidth Сдвигаем label, чтобы визуально сравнять с enumitem
      \setlength{\leftmargin}{\labelwidth}% default is \labelwidth используют также \parindent в enumerations
      \setlength{\labelsep}{\widthof{\  }}% Управляет длиной отступа после точки % default is \biblabelsep
      \setlength{\leftskip}{-2.2em}
      \addtolength{\leftmargin}{\leftskip}%<----- here
      \setlength{\itemsep}{0pt}% Управление дополнительным вертикальным разрывом между записями. \bibitemsep по умолчанию соответствует \itemsep списков в документе.
      \setlength{\itemindent}{\bibhang}% Пользуемся тем, что \bibhang по умолчанию принимает значение \parindent (абзацного отступа), который переназначен в styles.tex
      \addtolength{\itemindent}{\labelwidth}% \labelwidth Сдвигаем правее на величину номера с точкой
      \addtolength{\itemindent}{\labelsep}% \labelsep Сдвигаем ещё правее на отступ после точки
      \setlength{\parsep}{\bibparsep}% расстояние между параграфами 
     }%
      \renewcommand*{\makelabel}[1]{\hss##1} % выравнивание по labelnumberwidth >= 2 строки item
  }
  {\endlist}
  {\item}
% % % %
\defbibenvironment{tsk} % переопределяем окружение библиографии из gost-numeric.bbx пакета biblatex-gost
  {\list
	{
		\printtext[labelnumberwidth]{%
			\printfield{labelprefix}%not numberprefix !
			\printfield{labelnumber}}
	}%
	{%
		\setlength{\labelwidth}{\labelnumberwidth}% This length register indicates the width of the widest labelnumber
		%      \addtolength{\labelwidth}{-3pt}% \labelwidth Сдвигаем label, чтобы визуально сравнять с enumitem
		\setlength{\leftmargin}{\labelwidth}% default is \labelwidth используют также \parindent в enumerations
		\setlength{\labelsep}{\widthof{\  }}% Управляет длиной отступа после точки % default is \biblabelsep
		\setlength{\itemsep}{0pt}% Управление дополнительным вертикальным разрывом между записями. \bibitemsep по умолчанию соответствует \itemsep списков в документе.
		\setlength{\itemindent}{0pt}% Пользуемся тем, что \bibhang по умолчанию принимает значение \parindent (абзацного отступа), который переназначен в styles.tex
		\addtolength{\itemindent}{0pt}% \labelwidth Сдвигаем правее на величину номера с точкой
		\addtolength{\itemindent}{0pt}% \labelsep Сдвигаем ещё правее на отступ после точки
		\setlength{\parsep}{\bibparsep}% расстояние между параграфами 
	}%
	\renewcommand*{\makelabel}[1]{\hss##1} % выравнивание по labelnumberwidth >= 2 строки item
}
{\endlist}
{\item}

%%% Счётчик использованных ссылок на литературу, обрабатывающий с учётом неоднократных ссылок
%%http://tex.stackexchange.com/a/66851/79756
%\newcounter{citenum}
%\newtotcounter{citenum}
%\makeatletter
%\defbibenvironment{counter} %Env of bibliography
%  {\setcounter{citenum}{0}%
%  \renewcommand{\blx@driver}[1]{}%
%  } %what is doing at the beginining of bibliography. In your case it's : a. Reset counter b. Say to print nothing when a entry is tested.
%  {} %Здесь то, что будет выводиться командой \printbibliography. \thecitenum сюда писать не надо
%  {\stepcounter{citenum}} %What is printing / executed at each entry.
%\makeatother
%\defbibheading{counter}{}

%
%
%\newtotcounter{citeauthorvak}
%\makeatletter
%\defbibenvironment{countauthorvak} %Env of bibliography
%{\setcounter{citeauthorvak}{0}%
%    \renewcommand{\blx@driver}[1]{}%
%} %what is doing at the beginining of bibliography. In your case it's : a. Reset counter b. Say to print nothing when a entry is tested.
%{} %Здесь то, что будет выводиться командой \printbibliography. Обойдёмся без \theciteauthorvak в нашей реализации
%{\stepcounter{citeauthorvak}} %What is printing / executed at each entry.
%\makeatother
%\defbibheading{countauthorvak}{}
%
%\newtotcounter{citeauthornotvak}
%\makeatletter
%\defbibenvironment{countauthornotvak} %Env of bibliography
%{\setcounter{citeauthornotvak}{0}%
%    \renewcommand{\blx@driver}[1]{}%
%} %what is doing at the beginining of bibliography. In your case it's : a. Reset counter b. Say to print nothing when a entry is tested.
%{} %Здесь то, что будет выводиться командой \printbibliography. Обойдёмся без \theciteauthornotvak в нашей реализации
%{\stepcounter{citeauthornotvak}} %What is printing / executed at each entry.
%\makeatother
%\defbibheading{countauthornotvak}{}
%
%\newtotcounter{citeauthorconf}
%\makeatletter
%\defbibenvironment{countauthorconf} %Env of bibliography
%{\setcounter{citeauthorconf}{0}%
%    \renewcommand{\blx@driver}[1]{}%
%} %what is doing at the beginining of bibliography. In your case it's : a. Reset counter b. Say to print nothing when a entry is tested.
%{} %Здесь то, что будет выводиться командой \printbibliography. Обойдёмся без \theciteauthorconf в нашей реализации
%{\stepcounter{citeauthorconf}} %What is printing / executed at each entry.
%\makeatother
%\defbibheading{countauthorconf}{}
%
%\newtotcounter{citeauthor}
%\makeatletter
%\defbibenvironment{countauthor} %Env of bibliography
%{\setcounter{citeauthor}{0}%
%    \renewcommand{\blx@driver}[1]{}%
%} %what is doing at the beginining of bibliography. In your case it's : a. Reset counter b. Say to print nothing when a entry is tested.
%{} %Здесь то, что будет выводиться командой \printbibliography. Обойдёмся без \theciteauthor в нашей реализации
%{\stepcounter{citeauthor}} %What is printing / executed at each entry.
%\makeatother
%\defbibheading{countauthor}{}
%
%\defbibheading{authorpublications}[\authorbibtitle]{\section*{#1}}
%\defbibheading{pubsubgroup}{\noindent\textbf{#1}}
%\defbibheading{otherpublications}{\section*{#1}}


%%%% Создание команд для вывода списка литературы %%%
%\newcommand*{\insertbibliofull}{
%\printbibliography[keyword=bibliofull,section=0]
%\printbibliography[heading=counter,env=counter,keyword=bibliofull,section=0]
%}
%
%\newcommand*{\insertbiblioauthor}{
%\printbibliography[heading=authorpublications,keyword=biblioauthor,section=1,title=\authorbibtitle]
%}
%\newcommand*{\insertbiblioauthorimportant}{
%\printbibliography[heading=authorpublications,keyword=biblioauthor,section=2,title={Наиболее значимые \MakeLowercase{\protect\authorbibtitle{}}}]
%}
%\newcommand*{\insertbiblioauthorgrouped}{% Заготовка для вывода сгруппированных печатных работ автора. Порядок нумерации определяется в соответствующих счетчиках внутри окружения refsection в файле common/characteristic.tex
%\section*{\authorbibtitle}
%\printbibliography[heading=pubsubgroup, keyword=biblioauthorvak, section=1,title=\vakbibtitle]%
%\printbibliography[heading=pubsubgroup, keyword=biblioauthorconf, section=1,title=\confbibtitle]%
%\printbibliography[heading=pubsubgroup, keyword=biblioauthornotvak, section=1,title=\notvakbibtitle]%
%}
%
%\newcommand*{\insertbiblioother}{
%\printbibliography[heading=otherpublications,keyword=biblioother]
%}




%Bibliography update
%TO delete predefined description of References
\defbibheading{bibliography}[\bibname]{%
	\markboth{#1}{#1}}
%https://tex.stackexchange.com/a/307757/44348
\newcommand{\fullbibtitle}{Библиографический список} % (ГОСТ Р 7.0.11-2011, 4)


%%%delete / in urldate (works together with \map)
\NewBibliographyString{urldateprefix}
%
\DefineBibliographyStrings{russian}{%
	urldateprefix = {дата обращения\addcolon}}
\DefineBibliographyStrings{english}{%
	urldateprefix = {visited on}}
\DeclareFieldFormat{urldate}{(\bibstring{urldateprefix}\addspace\mkdayzeros{\thefield{urlday}}\adddot\mkmonthzeros{\thefield{urlmonth}}\adddot\mkyearzeros{\thefield{urlyear}})}




%% small ``p'' before total page number of books can be made automatically only by 
%%


%renew (chapter+pages) to series format
%\newbibmacro*{chapter+pages}{%
%	\iffieldundef{chapter}
%	{}%
%	{\printfield{chapter}\space---\space}% 
%%	\setunit*{\bibpagespunct}%
%	\printfield{pages}%
%	\newunit}
\newbibmacro*{chapter+pages}{%
	\iffieldundef{chapter}
	{}%
	{\printfield{chapter}%
		\setunit*{\addspace---\space}}%
	\printfield{pages}%
	\newunit}



%delete / from date format - REPLACED BY \map to delete totally
%\DeclareFieldFormat{date}{\thefield{year}}
%TO-DO: add month with condition if it does not exist than do not write the dot
%\thefield{month}\nobreakdash\adddot

%delete : from DOI format 
\DeclareFieldFormat{doi}{%
	\mkbibacro{DOI}\space
	\ifhyperref
	{\href{https://doi.org/#1}{\nolinkurl{#1}}}
	{\nolinkurl{#1}}\isdot} 


%%%add Ser.: to series format
%% First approach
\NewBibliographyString{seriesprefix}
%
\DefineBibliographyStrings{english}{%
	seriesprefix = {Ser}}
\DefineBibliographyStrings{russian}{%
	seriesprefix = {Сер}}
\DeclareFieldFormat{series}{\bibstring{seriesprefix}\adddot\addcolon\space{#1}\isdot}

%% Second approach % nested conditions
%\DeclareFieldFormat{series}{\iffieldequalstr{langid}{russian}{Сер}{\iffieldequalstr{langid}{english}{Bingo}{NotSpecified}}\adddot\addcolon\space{#1}\isdot} %


%add brackets to note format
\DeclareFieldFormat{note}{\mkbibparens{{#1}}\isdot} 

%delete spaces before ; and :
\renewcommand*{\addcolondelim}{\addcolon\space}
\renewcommand*{\addsemicolondelim}{\addsemicolon\space}

%modify PhD theis, for candidate one can use words CANDthesis
\DefineBibliographyStrings{english}{phdthesis = {diss\adddot\space\ldots}}
\renewcommand*{\gostmedialanguage}{}    % Реализация пакетом biblatex через движок biber
}

\AtEveryBibitem{\stepcounter{citenum}}%подсчет ссылок
%%% Управление компиляцией отдельных частей диссертации %%%
% Необходимо сначала иметь полностью скомпилированный документ, чтобы все
% промежуточные файлы были в наличии
% Затем, для вывода отдельных частей можно воспользоваться командой \includeonly
% Ниже примеры использования команды:
%
%\includeonly{Dissertation/part2}
%\includeonly{Dissertation/contents,Dissertation/appendix,Dissertation/conclusion}
%
% Если все команды закомментированы, то документ будет выведен в PDF файл полностью

%TO-DO: 

% формат А5 

% масштабирование отступов и интервалов на основе параметров, зависимых от шрифтов (em, ex) 

%TO-DO warnings in draft option:
% во введении больше 4 страниц
% в заключении меньше 2 страниц
% в заключении больше 5 страниц
% ключевых слов больше 15
% ключевых слов/словосочетаний больше 5
% ключевых слов меньше 5
% ключевых слов/словосочетаний меньше 3
% в реферате больше 600 печатных знаков
% в конце названия главы/параграфа/подпараграфа отсутствуют точки
% при наличии более 1 строки в названии главы/параграфа/подпараграфа: в конце строки отсутствуют предлоги или союзы (проверка на ~) 
% в задании контрольные даты до защиты
% в библиографии дата обращения не раньше 1 дня преддипломной практики и не позднее даты загрузки ВКР

%TO_DO расширение примеров
% добавление из Положения разнообразных примеров по оформлению таблиц
% все изображения сделать более лёгкими (без расплывчатости) -> шаблон будет меньше весить
% в качестве использования цитат привести примеры на известных политехников (не современников)

%TO-DO улучшение сопутствующего ПО
% в TexStudio задать автоподстановку label в \firef{}, \taref{}.

%TO-DO синхронизация с шаблонами кандидатских и докторских диссертаций А.Акиньшина
% перенос лучшего функционала
% автоматизированная подача данных в http://vkr.spbstu.ru

%TO-DO переработка текущего функционала
%
% на титульной странице в таблице с подписями не должно быть отступов ~1мм слева и справа.
% 
% оформление приложений сейчас реализовано через <<взлом>> memoir-classa. Лучше использовать встроенный функционал, а именно определить дополнительный стиль оформления глав.

% устранить команды \NewPage: \newpage\leavevmode\thispagestyle{empty}\newpage после приложения %начать новое приложение с новой страницы % временное решение, т.к. не корректно работает \ContinueChapterPagesEnd. Пояснение:
%https://tex.stackexchange.com/questions/2958/why-is-newpage-ignored-sometimes

% проверка сортировки списка литературы (А-Я, A-Z).

% Оставить обратную связь, благодарности предложения:
% Google форма

% Внести изменение в шаблон для всех:
% pull-request

% Обсуждения по запуску шаблона, см. кандидатские и докторские диссертации

% Обсуждения по совершентствованию шаблона ВКР:
% gitter-канал



%% Список использованных источников
% текущая реализация - быстрое приближение к требованиям

%1) в действующем варианте env=SSTfirst необходимо выполнить точное выравнивание по абзацному отступу. Сейчас оформление :
%1.1) единиц 1-9 немного выходят за рамки отступа
%1.2) десяток 10-99 немного не добирает до абзацного отступа
%1.3) если будут сотни, то проблема усугубится
% Скороее всего, необходимо сделать выравнивание по левому краю 

%2) необходимо реализовать второй вариант вывода библиографии




%% Экспорт - импорт данных
% 
% 1) Формирование файла renames.tex на основе данных из личного кабинета 
% 2) Экспорт мета-данных на vkr.spbstu.ru



%% Создание сопутствующих файлов
%	\item Файлы \verb|task.pdf|(\verb|.tex|) --- задание;
%	\item Файлы \verb|annotation.pdf|(\verb|.tex|) --- аннотация;
%	\item Файлы \verb|slides.pdf|(\verb|.tex|) --- слайды;
%	\item Файлы \verb|poster.pdf|(\verb|.tex|) --- постер;
%	\item Файлы \verb|advisor_review.pdf|(\verb|.tex|) --- отзыв;
%	\item Файлы \verb|external_review.pdf|(\verb|.tex|) --- рецензия;
%%
%%%% Preamble end %%%%  % лучше не редактировать / please, keep unmodified

\setcounter{docType}{1} % лучше не редактировать / please, keep unmodified

%%%% Настройки автора / Author settings
%% 
%%%% Настройки автора 
%% 
%% 	 Пожалуйста, ознакомьтесь с функционалом шаблона из [1,2], а также с пакетами, подключенными в ch_preamble.
%% 
%%   Новым командам лучше присваивать уникальные имена.
%% 
%%%% Author settings
%% 
%%   Please, see all possible packages using the search in files of ch_preamble. 
%%   
%%   Please, for user-defined commands write only unique command titles.
%%


%%%% Подключение библиографии / Upload bibliography
%% 
%% 

\addbibresource{my_folder/my_biblio.bib} 



%%%% Полезные настройки / Usefull settings
%% 
%% Раскомментируйте, чтобы
%%
%% pdf при открытии выравнивался по окну
%% pdf fit screen window
\hypersetup{
pdfstartview={FitBH}
}
%% перенумеровать все строки pdf
%% enumerate all lines in pdf 
%\usepackage{lineno}
%\linenumbers
%%
%% установить дату после названия ВКР - расскоментируйте код в title.tex
%% set data after the thesis title - uncomment code in title.tex
\let\ordinal\relax %avoid extra warning
\usepackage{datetime}



%% In case of deleting the following info, please, delete the examples in the chapter body.

%% В случае комментирования (удаления) следующего кода могут появиться ошибки при компиляции примеров, т.е. необходимо будет удалить и примеры в теле главы.

\newcommand{\overbar}[1]{\mkern 1.5mu\overline{\mkern-1.5mu#1\mkern-1.5mu}\mkern 1.5mu}

%http://tex.stackexchange.com/questions/16645/blackboard-italic-font
% for itallic sign of context K to be a parametr
\DeclareMathAlphabet{\mathbbmsl}{U}{bbm}{m}{sl}
\newcommand{\cont}[1][K]{\ensuremath{\mathbbmsl{#1}}}

%%ARROWS

%mu = math unit = 1em
%\mkern-18mu
%"minus quad"

%https://tex.stackexchange.com/a/389805/44348
\newcommand{\fcaarrow}[1]{%
	{}^{\scriptscriptstyle\bm{#1}}
}
%%%%%%%%%%%%%%%%%%%%%%% ARROWS from Formal Concept Analysis
% small and bold \uparrow
\newcommand{\uA}{\fcaarrow{{\uparrow\mkern-12mu}}}
% small and bold \downarrow
\newcommand{\dA}{\fcaarrow{\downarrow\mkern-2mu}}
% small and bold \uparrow+\downarrow
\newcommand{\ud}{\fcaarrow{\uparrow\mkern-12mu}\fcaarrow{\downarrow\mkern-2mu}}
% small and bold \downarrow+\uparrow
\newcommand{\du}{\fcaarrow{\downarrow\mkern-2mu}\fcaarrow{\uparrow\mkern-12mu}}


%http://tex.stackexchange.com/questions/74125/how-do-i-put-text-over-symbols
\newcommand\eqdef{\mathrel{\overset{\makebox[0pt]{\mbox{\normalfont\tiny def}}}{=}}} %\sffamily



%%% Правила задания нового окружения

\theoremstyle{myplain} % первая команда для ввода доказательств
\newtheorem{m-new-env-first}{Название\_окружения}[chapter] 
% вместо m-new-env-first необходимо подставить название нового окружения;
% вместо Название\_окружения необходимо подставить название окружения, выводящееся в pdf;
% последний параметр обеспечивает нумерацию в пределах главы не меняется


\theoremstyle{mydefinition} % первая команда для ввода окружений, не связанных с доказательствами
\newtheorem{m-new-env-second}{Название\_окружения}[chapter] 
% вместо m-new-env-second необходимо подставить название нового окружения;
% вместо Название\_окружения необходимо подставить название окружения, выводящееся в pdf;
% последний параметр обеспечивает нумерацию в пределах главы не меняется % добавляем свои команды / update your commands

\begin{document} % начало документа


%%% Внесите свои данные - Input your data
%%
%%
\newcommand{\Author}{И.О.\,Фамилия} % И.О. Фамилия автора 
\newcommand{\AuthorFull}{Фамилия Имя Отчество} % Фамилия Имя Отчество автора
\newcommand{\AuthorFullDat}{Фамилия Имя Отчество} % Фамилия Имя Отчество автора в дательном падеже (Кому? Студенту...)
\newcommand{\AuthorFullVin}{Фамилия Имя Отчество} % в винительном падеже (Кого? что?  Програмиста ...)
\newcommand{\AuthorPhone}{+7-9XX-XXX-XX-XX} % номер телефорна автора для оперативной связи  
\newcommand{\Supervisor}{И.О.\,Фамилия} % И. О. Фамилия научного руководителя
\newcommand{\SupervisorFull}{Фамилия Имя Отчество} % Фамилия Имя Отчество научного руководителя
\newcommand{\SupervisorVin}{И.О.\,Фамилию} % И. О. Фамилия научного руководителя  в винительном падеже (Кого? что? Руководителя ...)
\newcommand{\SupervisorJob}{должность} %
\newcommand{\SupervisorJobVin}{должность} % в винительном падеже (Кого? что?  Програмиста ...)
\newcommand{\SupervisorDegree}{степень} %
\newcommand{\SupervisorTitle}{звание} % 
%%
%%
%Руководитель, утверждающий задание
\newcommand{\Head}{И.О.\,Фамилия} % И. О. Фамилия руководителя подразделения (руководителя ОП)
\newcommand{\HeadDegree}{Должность руководителя}% Только должность:   
%Руководитель %ОП 
%Заведующий % кафедрой
%Директор % Высшей школы
%Зам. директора
\newcommand{\HeadDep}{M} % заменить на краткую аббревиатуру подразделения или оставить пустым, если утверждает руководитель ОП

%%% Руководитель, принимающий заявление
\newcommand{\HeadAp}{И.О.\,Фамилия} % И. О. Фамилия руководителя подразделения (руководителя ОП)
\newcommand{\HeadApDegree}{Должность руководителя}% Только должность:   
%Руководитель ОП 
%Заведующий кафедрой
%Директор Высшей школы
\newcommand{\HeadApDep}{O} % заменить на краткую аббревиатуру подразделения или оставить пустым, если утверждает руководитель ОП
%%% Консультант по нормоконтролю
\newcommand{\ConsultantNorm}{И.О.\,Фамилия} % И. О. Фамилия консультанта по нормоконтролю. ТОЛЬКО из числа ППС!
\newcommand{\ConsultantNormDegree}{должность, степень} %   
%%% Первый консультант
\newcommand{\ConsultantExtraFull}{Фамилия Имя Отчетство} % Фамилия Имя Отчетство дополнительного консультанта 
\newcommand{\ConsultantExtra}{И.О.\,Фамилия} % И. О. Фамилия дополнительного консультанта 
\newcommand{\ConsultantExtraDegree}{должность, степень} % 
\newcommand{\ConsultantExtraVin}{И.О.\,Фамилию} % И. О. Фамилия дополнительного консультанта в винительном падеже (Кого? что? Руководителя ...)
\newcommand{\ConsultantExtraDegreeVin}{должность, степень} %  в винительном падеже (Кого? что? Руководителя ...)
%%% Второй консультант
\newcommand{\ConsultantExtraTwoFull}{Фамилия Имя Отчетство} % Фамилия Имя Отчетство дополнительного консультанта 
\newcommand{\ConsultantExtraTwo}{И.О.\,Фамилия} % И. О. Фамилия дополнительного консультанта 
\newcommand{\ConsultantExtraTwoDegree}{должность, степень} % 
\newcommand{\ConsultantExtraTwoVin}{И.О.\,Фамилию} % И. О. Фамилия дополнительного консультанта в винительном падеже (Кого? что? Руководителя ...)
\newcommand{\ConsultantExtraTwoDegreeVin}{должность, степень} %  в винительном падеже (Кого? что? Руководителя ...)
\newcommand{\Reviewer}{И.О.\,Фамилия} % И. О. Фамилия резензента. Обязателен только для магистров.
\newcommand{\ReviewerDegree}{должность, степень} % 
%%
%%
\renewcommand{\thesisTitle}{Тема выпускной квалификационной работы}
\newcommand{\thesisDegree}{работа бакалавра}% дипломный проект, дипломная работа, магистерская диссертация %c 2020
\newcommand{\thesisTitleEn}{Title of the thesis} %2020
\newcommand{\thesisDeadline}{дд.мм.202X}
\newcommand{\thesisStartDate}{дд.мм.202X}
\newcommand{\thesisYear}{202X}
%%
%%
\newcommand{\group}{N} % заменить вместо N номер группы
\newcommand{\thesisSpecialtyCode}{ХХ.ХХ.ХХ}% код направления подготовки
\newcommand{\thesisSpecialtyTitle}{Наименование направления подготовки} % наименование направления/специальности
\newcommand{\thesisOPPostfix}{YY} % последние цифры кода образовательной программы (после <<_>>)
\newcommand{\thesisOPTitle}{Наименование направленности (профиля) образовательной программы}% наименование образовательной программы
%%
%%
\newcommand{\institute}{
Название института
%Институт компьютерных наук и~технологий
%Гуманитарный институт
%Инженерно-строительный институт
%Институт биомедицинских систем и технологий
%Институт металлургии, машиностроения и транспорта
%Институт передовых производственных технологий
%Институт прикладной математики и механики
%Институт физики, нанотехнологий и телекоммуникаций
%Институт физической культуры, спорта и туризма
%Институт энергетики и транспортных систем
%Институт промышленного менеджмента, экономики и торговли
}%
%%
%%




%%% Задание ключевых слов и аннотации
%%
%%
%% Ключевых слов от 3 до 5 слов или словосочетаний в именительном падеже именительном падеже множественного числа (или в единственном числе, если нет другой формы) по правилам русского языка!!!
%%
%%
\newcommand{\keywordsRu}{Стилевое оформление сайта, управление контентом, php, MySQL, архитектура системы} % ВВЕДИТЕ ключевые слова по-русски
%%
%%
\newcommand{\keywordsEn}{Style registration, content management, php, MySQL, system architecture} % ВВЕДИТЕ ключевые слова по-английски
%%
%%
%% Реферат ОТ 1000 ДО 1500 знаков на русский или английский текст
%%
%Реферат должен содержать:
%- предмет, тему, цель ВКР;
%- метод или методологию проведения ВКР:
%- результаты ВКР:
%- область применения результатов ВКР;
%- выводы.

\newcommand{\abstractRu}{В данной работе изложена сущность подхода к созданию динамического информационного портала на основе использования открытых технологий Apache, MySQL и PHP. Даны общие понятия и классификация IT-систем такого класса. Проведен анализ систем-прототипов. Изучена технология создания указанного класса информационных систем. Разработана конкретная программная реализация динамического информационного портала на примере портала выбранной тематики...} % ВВЕДИТЕ текст аннотации по-русски
%%
%%
\newcommand{\abstractEn}{In the given work the essence of the approach to creation of a dynamic information portal on the basis of use of open technologies Apache, MySQL and PHP is stated. The general concepts and classification of IT-systems of such class are given. The analysis of systems-prototypes is lead. The technology of creation of the specified class of information systems is investigated. Concrete program realization of a dynamic information portal on an example of a portal of the chosen subjects is developed...} % ВВЕДИТЕ текст аннотации по-английски


%%% РАЗДЕЛ ДЛЯ ОФОРМЛЕНИЯ ПРАКТИКИ
%Место прохождения практики
\newcommand{\PracticeType}{Отчет о прохождении % 
	%стационарной производственной (технологической (проектно-технологической)) %
	такой-то % тип и вид ЗАМЕНИТЬ
	практики}

\newcommand{\Workplace}{СПбПУ, ИКНТ, ВШИСиСТ} % TODO Rename this variable

% Даты начала/окончания
\newcommand{\PracticeStartDate}{%
дд.мм.гггг%
%	22.06.2020
}%
\newcommand{\PracticeEndDate}{%
	дд.мм.гггг%
%	18.07.2020%
}%
%%

\newcommand{\School}{
	Название высшей школы
%	Высшая школа интеллектуальных систем и~суперкомпьютерных~технологий 
}
\newcommand{\practiceTitle}{Тема практики}


%% ВНИМАНИЕ! Необходимо либо заменить текст аннотации (ключевых слов) на русском и английском, либо удалить там весь текст, иначе в свойства pdf-отчета по практике пойдет шаблонный текст.

%%% Не меняем дальнейшую часть - Do not modify the rest part
%%
%%
%%
%%
\ifnumequal{\value{docType}}{1}{% Если ВКР, то...
	\newcommand{\DocType}{Выпускная квалификационная работа}
	\newcommand{\pdfDocType}{\DocType~(\thesisDegree)} %задаём метаданные pdf файла
	\newcommand{\pdfTitle}{\thesisTitle}
}{% Иначе 
	\newcommand{\DocType}{\PracticeType}
	\newcommand{\pdfDocType}{\DocType} %задаём метаданные pdf файла
	\newcommand{\pdfTitle}{\practiceTitle}
}%
\newcommand{\HeadTitle}{\HeadDegree~\HeadDep}
\newcommand{\HeadApTitle}{\HeadApDegree~\HeadApDep}
\newcommand{\thesisOPCode}{\thesisSpecialtyCode\_\thesisOPPostfix}% код образовательной программы
\newcommand{\thesisSpecialtyCodeAndTitle}{\thesisSpecialtyCode~\thesisSpecialtyTitle}% Код и наименование направления/специальности
\newcommand{\thesisOPCodeAndTitle}{\thesisOPCode~\thesisOPTitle} % код и наименование образовательной программы
%%
%%
\hypersetup{%часть болка hypesetup в style
		pdftitle={\pdfTitle},    % Заголовок pdf-файла
		pdfauthor={\AuthorFull},    % Автор
		pdfsubject={\pdfDocType. Шифр и наименование направления подготовки: \thesisSpecialtyCodeAndTitle. \abstractRu},      % Тема
		pdfcreator={LaTeX, SPbPU-student-thesis-template},     % Приложение-создатель
%		pdfproducer={},  % Производитель, Производитель PDF % будет выставлена автоматически
		pdfkeywords={\keywordsRu}
}
%%
%%
%% вспомогательные команды
\newcommand{\firef}[1]{рис.\ref{#1}} %figure reference
\newcommand{\taref}[1]{табл.\ref{#1}}	%table reference
%%
%%
%% Архивный вариант задания ключевых слов, аннотации и благодарностей 
% Too hard to export data from the environment to pdf-info
% https://tex.stackexchange.com/questions/184503/collecting-contents-of-environment-and-store-them-for-later-retrieval
%заменить NewEnviron на newenvironment для распознавания команды в TexStudio
%\NewEnviron{keywordsRu}{\noindent\MakeUppercase{\BODY}}
%\NewEnviron{keywordsEn}{\noindent\MakeUppercase{\BODY}}
%\newenvironment{abstractRu}{}{}
%\newenvironment{abstractEn}{}{}
%\newenvironment{acknowledgementsRu}{\par{\normalfont \acknowledgements.}}{}
%\newenvironment{acknowledgementsEn}{\par{\normalfont \acknowledgementsENG.}}{}


%%% Переопределение именований %%% Не меняем - Do not modify
%\newcommand{\Ministry}{Минобрнауки России} 
\newcommand{\Ministry}{Министерство науки и высшего образования Российской~Федерации} %с 2020
\newcommand{\SPbPU}{Санкт-Петербургский политехнический университет Петра~Великого}
\newcommand{\SPbPUOfficialPrefix}{Федеральное государственное автономное образовательное учреждение высшего образования}
\newcommand{\SPbPUOfficialShort}{ФГАОУ~ВО~<<СПбПУ>>}
%% Пробел между И. О. не допускается.
\renewcommand{\alsoname}{см. также}
\renewcommand{\seename}{см.}
\renewcommand{\headtoname}{вх.}
\renewcommand{\ccname}{исх.}
\renewcommand{\enclname}{вкл.}
\renewcommand{\pagename}{Pages}
\renewcommand{\partname}{Часть}
\renewcommand{\abstractname}{\textbf{Аннотация}}
\newcommand{\abstractnameENG}{\textbf{Annotation}}
\newcommand{\keywords}{\textbf{Ключевые слова}}
\newcommand{\keywordsENG}{\textbf{Keywords}}
\newcommand{\acknowledgements}{\textbf{Благодарности}}
\newcommand{\acknowledgementsENG}{\textbf{Acknowledgements}}
\renewcommand{\contentsname}{Content} % 
%\renewcommand{\contentsname}{Содержание} % (ГОСТ Р 7.0.11-2011, 4)
%\renewcommand{\contentsname}{Оглавление} % (ГОСТ Р 7.0.11-2011, 4)
\renewcommand{\figurename}{Рис.} % Стиль СПбПУ
%\renewcommand{\figurename}{Рисунок} % (ГОСТ Р 7.0.11-2011, 5.3.9)
\renewcommand{\tablename}{Таблица} % (ГОСТ Р 7.0.11-2011, 5.3.10)
%\renewcommand{\indexname}{Предметный указатель}
\renewcommand{\listfigurename}{Список рисунков}
\renewcommand{\listtablename}{Список таблиц}
\renewcommand{\refname}{\fullbibtitle}
\renewcommand{\bibname}{\fullbibtitle}

\newcommand{\chapterEnTitle}{Сhapter title} % <- input the English title here (only once!) 
\newcommand{\chapterRuTitle}{Название главы}          % <- введите 
\newcommand{\sectionEnTitle}{Section title} %<- input subparagraph title in english
\newcommand{\sectionRuTitle}{Название подраздела} % <- введите название подраздела по-русски
\newcommand{\subsectionEnTitle}{Subsection title} % - input subsection title in english
\newcommand{\subsectionRuTitle}{Название параграфа} % <- введите название параграфа по-русски
\newcommand{\subsubsectionEnTitle}{Subsubsection title} % <- input subparagraph title in english
\newcommand{\subsubsectionRuTitle}{Название подпараграфа} % <- введите название подпараграфа по-русски % Заполнить сведения, 
										 % в т.ч. ключевые слова и аннотацию.

%%%% Титульник ВКР / Thesis title 
%%
%% добавить лист в pdf-навигацию 
%% add to pdf navigation menu
%%
\pdfbookmark[-1]{\pdfTitle}{tit}
%%
\thispagestyle{empty}%
\makeatletter
\newgeometry{top=2cm,bottom=2cm,left=3cm,right=1cm,headsep=0cm,footskip=0cm}
\savegeometry{NoFoot}%
\makeatother


%%% Распечатать версию документа / Print document version
%%
\begin{flushright}
%	\vspace{0pt plus0.1fill}
	\boxed{\small
		\begin{tabular}{r} 
			\textbf{Пример ВКР <<SPbPU-student-thesis-template>>.} %\\ % перенос на новую строку
			\textbf{Версия от \today % \; время:  \currenttime. % время версии
			}
		\end{tabular}
	} %end boxed
%	\vspace*{-5pt} % раскомментировать, если не хватает места
	\vspace{0pt plus0.1fill} % раскоментировать, если хватает места
\end{flushright}

{\centering%
	\Ministry\\
	\SPbPU\\
	{%\bfseries %2020 - указание на изменения, которые могут быть введены в 2020 году
		\institute}
\par}%


\vspace{0pt plus1fill} %число перед fill = кратность относительно некоторого расстояния fill, кусками которого заполнены пустые места


\noindent
\begin{minipage}{\linewidth}
	\vspace{\mfloatsep} % интервал 
	\begin{tabularx}{\linewidth}{Xl}
	&Работа допущена к защите     \\
	&\HeadTitle     \\			
	&\underline{\hspace*{0.1\textheight}} \Head     \\
	&<<\underline{\hspace*{0.05\textheight}}>> \underline{\hspace*{0.1\textheight}} \thesisYear~г.  \\ 
	\end{tabularx}
	\vspace{\mfloatsep} % интервал 	
\end{minipage}


\vspace{0pt plus2fill} %


{\centering%
	
	\MakeUppercase{\bfseries{}\DocType} \\ 
	\MakeUppercase{\thesisDegree}%


%\intervalS% %ОБЯЗАТЕЛЬНО ДОБАВИТЬ ОТСТУП, ЕСЛИ ХВАТАЕТ МЕСТА
{\centering%
	\MakeUppercase{\bfseries{\thesisTitle}}}%

}\par%

%\intervalS% %ОБЯЗАТЕЛЬНО ДОБАВИТЬ ОТСТУП, ЕСЛИ ХВАТАЕТ МЕСТА
%по специальности % для специалистов
\noindent	по направлению подготовки \thesisSpecialtyCodeAndTitle{}\\% для бакалавров и магистров 
%\noindent Направленность  % для специалистов
\noindent	Направленность (профиль)	\thesisOPCodeAndTitle % для бакалавров и магистров
% Лучше по~профилю, но что делать, так составили Положение
\par%





\vspace{4mm plus2fill}%

\noindent
\begin{tabularx}{\linewidth}{lXl}
	Выполнил              &	   &             \\
	студент гр.~\group     &    & \Author     \\[\mfloatsep]

	Руководитель 		  &    &             \\
	\SupervisorJob,		  &    &             \\
	\SupervisorDegree, \SupervisorTitle\footnote{Должность указывают сокращенно, учёную степень и звание ---~при наличии, а~подразделения ---~аббревиатурами. <<СПбПУ>> и~аббревиатуры институтов не добавляют.} 	  &    & \Supervisor \\[\mfloatsep]
	
	Консультант\footnote{Оформляется по решению руководителя ОП или подразделения. Только 1 категория: <<Консультант>>. В исключительных случаях можно указать <<Научный консультант>> (должен иметь степень). Без печати и заверения подписи.}		  &    & 			 \\
	\ConsultantExtraDegree 	  &    & \ConsultantExtra\\[\mfloatsep]
	
	Консультант  &    &  \\   	
	по нормоконтролю\footnote{Обязателен, из числа ППС по решению руководителя ОП или подразделения. Должность и степень не указываются. Сведения помещаются в последнюю строчку по порядку. Рецензенты не указываются.}  		 	  &    & \ConsultantNorm  % обязателен
\end{tabularx} %


%
\vspace{0pt plus4fill}% 


\begin{center}%
Санкт-Петербург\\
\thesisYear
\end{center}%
\restoregeometry
\newpage					 % Титульный лист
										 % Убираем footnotes, консультанта, если нет

%\input{/My_task}					 % Задание 
										 % Для сдачи в высшую школу компилируем двухсторонний My_task.tex 
										 % После подписания задания изменение его содержания и оформления запрещено

%%% Не менять - Do not modify
%%\thispagestyle{empty}%
%\setcounter{sumPageFirst}{\value{page}} %сохранили номер первой страницы Реферата
\ifnumequal{\value{sumPrint}}{1}{% если двухсторонняя печать Задания, то...
	\newgeometry{twoside,top=2cm,bottom=2cm,left=3cm,right=1cm,headsep=0cm,footskip=0cm}
	\savegeometry{MyTask} %save settings
	\makeatletter % задаём оформление второй страницы ВКР как нечетной, а третьей - как чётной
	\let\tmp\oddsidemargin
	\let\oddsidemargin\evensidemargin
	\let\evensidemargin\tmp
	\reversemarginpar
	\makeatother
}{} % 
\pagestyle{empty} % удаляем номер страницы на втором/третьем листе 
\chapter*[Count-me]{Реферат} % * - не нумеруем
\thispagestyle{empty}% удаляем параметры страницы
%\setcounter{sumPageFirst}{\value{page}}
%sumPageFirst \arabic{sumPageFirst}
%
%
%% Возможность проверить другие значения счетчиков - debugging
%\ref*{TotPages}~с.,
%\formbytotal{mytotalfigures}{рисун}{ок}{ка}{ков},
%\formbytotal{mytotaltables}{таблиц}{у}{ы}{},
%There are \TotalValue{mytotalfigures} figures in this document
%There are \TotalValue{mytotalfiguresInApp} figuresINAPP in this document
%There are \TotalValue{mytotaltables} tables in this document
%There are \TotalValue{mytotaltablesInApp} figuresINAPP in this document
%There are \TotalValue{myappendices} appendix chapters in this document
%\total{citenum}~библ. наименований.



%% Для того, чтобы значения счетчиков корректно отобразились, необходимо скомпилировать файл 2-3 раза
На \total{mypages}~c.,  
\formbytotal{myfigures}{рисун}{ок}{ка}{ков},
\formbytotal{mytables}{таблиц}{у}{ы}{},
\formbytotal{myappendices}{приложен}{ие}{ия}{ий}.  

%\noindent
{\MakeUppercase{Ключевые слова: \keywordsRu}.}\footnote{Всего \textbf{слов}: от 3 до 15. Всего \textbf{слов и словосочетаний}: от 3 до 5. Оформляются в именительном падеже множественного числа (или в единственном числе, если нет другой формы), оформленных по правилам русского языка. \textit{Внимание! Размещение сноски после точки является примером как запрещено оформлять сноски.}} % Ключевые слова из renames.tex

Тема выпускной квалификационной работы: <<\thesisTitle>>\footnote{Реферат \textbf{должен содержать}: предмет, тему, цель ВКР; метод или методологию проведения ВКР: результаты ВКР: область применения результатов ВКР; выводы.}.


\abstractRu\footnote{ОТ 1000 ДО 1500 печатных знаков (ГОСТ Р 7.0.99-2018 СИБИД) на русский или английский текст. Текст реферата повторён дважды на русском и английском языке для демонстрации подхода к нумерации страниц.} % Аннотация из renames.tex

\abstractRu % УДАЛИТЬ. Повтор иллюстрации переноса текста на вторую страницу



\printTheAbstract % не удалять


\total{mypages}~p., 
\total{myfigures}~figures, 
\total{mytables}~tables,
\total{myappendices}~appendices.

%\noindent
{\MakeUppercase{Keywords: \keywordsEn}.} % Ключевые слова из renames.tex 
	
The subject of the graduate qualification work is <<\thesisTitleEn>>.
	
	
\abstractEn % Аннотация из renames.tex

\abstractEn % УДАЛИТЬ. Повтор для иллюстрации переноса текста на вторую страницу
	


%% Не менять - Do not modify
\thispagestyle{empty}
%\setcounter{sumPageLast}{\value{page}} %сохранили номер последней страницы Задания
%\setcounter{sumPages}{\value{sumPageLast}-\value{sumPageFirst}}
%sumPageLast \arabic{sumPageLast}
%
%sumPages \arabic{sumPages}
%\restoregeometry % восстанавливаем настройки страницы
%\setcounter{sumPageLast}{\value{page}} %сохранили номер последней страницы Задания
\setcounter{sumPages}{\value{sumPageLast}-\value{sumPageFirst}}
\arabic{sumPageLast}
\arabic{sumPages}
\restoregeometry % восстанавливаем настройки страницы
\pagestyle{plain} % удаляем номер страницы на первой/второй странице Задания
%% Обязательно закомментировать, если получается один лист в задании:
\ifnumequal{\value{sumPages}}{0}{% Если 1 страница в Задании, то ничего не делать.
}{% Иначе
	\setcounter{page}{\value{page}-\value{sumPages}} 	% вычесть значение sumPages при печати более 1 страницы страниц
}%	% настройки - конец			 	 % Реферат 
										 % Убираем footnotes, дубли команд \abstractEn и \abstractRu 
										
\setcounter{page}{2}
%%% Не мянять - Do not modify !
%%
%%
%% Оглавление (ГОСТ Р 7.0.11-2011, 5.2)
%\ifdefmacro{\microtypesetup}{\microtypesetup{protrusion=false}}{} % не рекомендуется применять пакет микротипографики к автоматически генерируемому оглавлению
%\tableofcontents*
%\addtocontents{ptc}{\protect\tocheader}
%\endTOCtrue
%\ifdefmacro{\microtypesetup}{\microtypesetup{protrusion=true}}{}
%https://tex.stackexchange.com/questions/170912/contents-page-in-two-different-languages


\setlength{\parskip}{0.35\onelineskip} % интервал между элементов - полуторный
\begin{Spacing}{\Single} %интервал внутри элемента - одинарный
\tableofcontents
 \end{Spacing}
\setlength{\parskip}{0pt} % интервал между элементов - полуторный
\OnehalfSpacing*    % Полуторный интервал % * to force it in the floats
\newpage  	         % Оглавление


\chapter*{Введение} % * не проставляет номер
\addcontentsline{toc}{chapter}{Введение} % вносим в содержание

Работа обычного учителя включает в себя не только составление программы для детей и их непосредственное обучение, но и проверку всех написанных ими работ в том числе. Как правило, этот процесс происходит независимо от рабочего времени. Что заставляет тратить несколько часов личного свободного времени.

В современном мире стало довольно популярно оцифровывать механизмы обучения. Сейчас вы можете встретить такие инструменты, как автоматическая проверка тестовой части единого государственного экзамена, онлайн тесты на дистанционных ресурсах. Но по-прежнему процесс автоматической проверки рукописных текстов не является частью школьного и дошкольного образований.

А сейчас появляются тенденции проведения дистанционного обучения, где каждая работа оцифровывается перед тем, как попасть в руки преподавателя. Соответственно, такую работу сложнее проверять и анализировать, так как делать пометки в таком формате работы становится затруднительным. И это помимо того, что проверка уже занимает много личного времени.

Выходом из этой ситуации является программа автоматической проверки рукописных работ за счет анализа цифровых изображений методом нейронных сетей.

Очевидно, что такой программный продукт является актуальным в наше время, поскольку значительно упрощает ручную проверку работ.

\textbf{Целью данной работы} является исследование и анализ существующих алгоритмов, позволяющих считывать текст, написанный от руки, в формате цифрового изображения. И для достижения данной цели необходимо выполнить следующие \textbf{задачи}:
\begin{itemize}
	\item Исследовать определенные типы нейронных сетей;
	\item Исследовать способы разбиения текста на изображении;
	\item Исследовать готовые алгоритмы считывания текста;
	\item Вывод по анализу улучшения существующих алгоритмов.
\end{itemize}






%% Вспомогательные команды - Additional commands
%\newpage % принудительное начало с новой страницы, использовать только в конце раздела
%\clearpage % осуществляется пакетом <<placeins>> в пределах секций
%\newpage\leavevmode\thispagestyle{empty}\newpage % 100 % начало новой строки	    	 % Введение

%% Начало основной части
\chapter*{Предисловие} %\label{ch1}

% не рекомендуется использовать отдельную section <<введение>> после лета 2020 года
%\section{Введение. Сложносоставное название первого параграфа первой главы для~демонстрации переноса слов в содержании} \label{ch1:intro}
\addcontentsline{toc}{chapter}{Предисловие}

Все изображения, о которых далее идет речь, хранятся в виде растровых изображений на твердотельном накопителе, к примеру, диск компьютера. Такие файлы хранят информацию лишь о цвете каждого отдельного пикселя в матрице изображения. Процесс распознавания текста – это получение каждого отдельного символа в виде его кода в том или ином текстовом формате.

На данный момент существует множество решений для распознавания машинописных и рукопечатных текстов на изображениях. Есть даже готовые программные продукты, такие как FineReader, которые неплохо справляются со своей задачей. Для первого варианта задача является готовой, но для второго она до сих пор полностью не решена, так как является существенно затруднительной по сравнению с предыдущей.

Задача распознавания рукописного текста (handwriting recognition) имеет два типа подхода:
\begin{itemize}
	\item 	Онлайн распознавание – анализ текста сразу при его написании;
	\item Оффлайн распознавание - анализ уже готового написанного текста на изображении.
\end{itemize}

В первом случае программа должна считывать символ сразу при его написании, что гораздо упрощает работу алгоритма, поскольку не нужно распознавать целое изображение и искать в нем текст, выполнять дальнейшие шаги обработки. Текст распознается с большей точностью из-за особенности написания (символы обычно более строго отделяются друг от друга). И алгоритм достаточно прост. Сперва нужно распознать введенный на текущем этапе ввода символ. Сделать это можно, используя нейронные сети. Такая технология позволит получить вероятности того или иного класса среди всех возможных, после чего выбирается наиболее подходящая. Причем здесь можно учитывать введенные символы до текущего, чтобы, возможно, изменить выбор нейронной сети в пользу другого символа. 

Такая задача уже реализована, например ввод текста в клавиатуре смартфона, ввод текста в программных продуктах, таких как google переводчик, что позволяет более гибко печатать иероглифы.

Во втором случае мы имеем дело с отсканированными или сфотографированными изображениями, где текст уже полностью представлен в готовом виде. Это могут быть листы конспектов, работ по одному из учебных предметов, старые рукописи. При этом в зависимости от типа таких изображений определяется направление использования, к примеру, для учебных работ - их проверка, для старых рукописей - наиболее быстрый способ оцифрования таких документов, чем вручную перепечатывать каждую страницу. Но тут как раз добавляется ряд проблем, которые усложняют реализацию.

Первой проблемой является сам вид изображения. Оно может иметь определенные дефекты, такие как пятна, шум. В старых документах такие недостатки являются наиболее встречаемыми, текст и вовсе может быть расплывчатым. И его нужно сперва найти, а такие факторы существенно усложняют этот процесс. Далее идет проблема самого написания, потому что в случае машинописного текста линия текста прямая, параллельная остальным, нахождение строк не составляет трудностей. Человек при написании может изгибать строку, ширина интервалов строк может варьироваться в достаточно широких пределах. Могут быть случаи, когда, к примеру, в тех же старых рукописях из-за размытого текста совсем нет явной границы строк. В учебной работе из-за помарок и исправлений такие интервалы тоже могут быть стерты. А при онлайн распознавании и вовсе строк нет, не нужно определять их, алгоритм пропускает эту стадию и сразу считывает слова. Также текст может быть расположен в разных, независимых частях изображения, тогда решение по сегментации строк вообще становится нетипичным. 

И самая главная проблема состоит в начертаниях: строки, слова и отдельные символы текста могут накладываться друг на друга, пробелы между словами могут быть совершенно различными. В случае пересечения символов нужно сперва их разбить, то есть определить, по какой линии и как разделить сцепившиеся символы. Такой процесс разбиения затруднителен. Впрочем, кляксы, зачеркивания и исправления совсем портят возможность упростить алгоритм распознавания.\cite{li}

В общем случае процесс оффлайн распознавания рукописного текста состоит из следующих этапов:
\begin{itemize}
	\item Первоначальная обработка изображения;
	\item Поиск текста или сегментация строк;
	\item Сегментация слов;
	\item Распознавание слов;
	\item Возможное исправление первичных результатов.
\end{itemize}
	
Стоит отметить, что последний этап не является обязательным и вовсе не нужен в случаях, когда нужно извлечь лишь необходимую информацию. А четвертый этап может потребовать сегментации символов, если известно, что текст может содержать символы помимо букв алфавита. 


%% Вспомогательные команды - Additional commands
%
%\newpage % принудительное начало с новой страницы, использовать только в конце раздела
%\clearpage % осуществляется пакетом <<placeins>> в пределах секций
%\newpage\leavevmode\thispagestyle{empty}\newpage % 100 % начало новой страницы	         	 % Глава 1
%\ContinueChapterBegin % размещать главы <<подряд>>
\chapter{Нейронные сети} \label{ch4}

% не рекомендуется использовать отдельную section <<введение>> после лета 2020 года
%\section{Введение} \label{ch4:intro}

Нейронные сети активно развиваются в современном мире. В крупных проектах, где есть возможность делиться информацией, таких как Instagram, Facebook, Вконтакте давно уже начали использовать нейронные сети для распознавания образов. Такие типы сетей позволяют определить и квалифицировать объект, как правило, на изображении. Они позволяют своего рода придать зрение компьютеру или другому вычислительному устройству, чтобы тот смог увидеть и определить, что ему показывают. При распознавании текста на изображениях будет использоваться тот же механизм, ведь необходимо увидеть и прочитать текст, который представляет из себя пиксели на картинке. Из используемых далее в алгоритмах сетей выделяются многоуровневые персептроны и сверточные нейронные сети. \cite{neiron1}
	
\section{Персептрон} \label{ch4:sec1}
Такая нейронная сеть представляет из себя граф, где есть начальные узлы - входные данные, и есть конечные узлы - выходные данные (классы). Они соединены между собой ребрами, которые в свою очередь имеют веса. Эти веса изначально неизвестны. Это был одноуровневый персептрон, но в общем случае мы можем добавить между ними дополнительные уровни из узлов, так называемые скрытые слои. И каждый слой связан с предыдущим наборов ребер. Первая задача состоит в том, чтобы научить такую сеть классифицировать объект, причем необязательно как бы зрительно. То есть нужно определить изначальные веса для каждого ребра, чтобы подавая на входные узлы графа, получить узел, который будет показывать наиболее вероятный класс такого объекта. И нам нужно говорить, правильный был ответ сети, или же нет, указывая на верный вариант. Впоследствии она научится сама уже квалифицировать объекты. Такой метод еще называется методом обратного распространения ошибки.

В нашем случае, допустим, возьмем символ, который представлен матрицей пикселей, где ячейка, принадлежащая символу, отмечена единицей, или близким к ней значением, а фон нулем. Тогда подав на вход многоуровнего персептрона эту матрицу, мы должны, после обучения, разумеется, получить наиболее вероятный символ.  \cite{cit1}


\section{Сверточная нейронная сеть} \label{ch4:sec2}

Сверточная нейронная сеть устроена несколько сложнее, чем предыдущий вариант, но более тесно связана с распознаванием образов, так как обладает специальным механизмом, увеличивающим эффективность и масштабирование в отличие от персептрона. Задача стоит в том, чтобы от мелких деталей, которые могут содержаться на изображении, к примеру линии, кривые, переходить к более сложным, что позволит распознать достаточно сложный объект, сложнее, чем символ. Такая сеть проходит по изображению фильтрами, представляющие из себя матрицы, и которые будут сигнализировать о том, что в этом участке, где в данный момент находится фильтр, имеется та деталь, которую этот фильтр ищет, к примеру прямую под определенным углом. Затем они сворачиваются в матрицы, к которым снова применяются фильтры. Матрицы фильтров задаются посредством обучения сети. Затем на выходе мы получаем определенный класс принадлежности. \cite{neiron2}

%\FloatBarrier % заставить рисунки и другие подвижные (float) элементы остановиться

\section{Выводы} \label{ch4:conclusion}

Изучив эти два типа нейронных сетей, можно сделать вывод о том, что для более мелких и простых случаев лучше использовать персептрон, но если же нужно найти на изображении некоторый объект или определить более сложный предмет, то лучше использовать сверточную нейронную сеть. 

%% Вспомогательные команды - Additional commands
%
%\newpage % принудительное начало с новой страницы, использовать только в конце раздела
%\clearpage % осуществляется пакетом <<placeins>> в пределах секций
%\newpage\leavevmode\thispagestyle{empty}\newpage % 100 % начало новой страницы    
\chapter{Сегментация изображения на детали} %\label{ch2}
	
% не рекомендуется использовать отдельную section <<введение>> после лета 2020 года
%\section{Введение} \label{ch2:intro}

На входе алгоритма поступает массив изображений, каждое из которых содержит в себе рукописный текст. Обрабатывать можно параллельно несколько изображений, если архитектура устройства, на котором будет запущен алгоритм, позволяет выполнять инструкции и команды на нескольких ядрах.

Но не существует алгоритма, который смог бы обработать целиком всю страницу. Такую задачу следует декомпозировать на нескольно подзадач - этапы. И первые из них сперва произвести элементарные единицы текста, которые потом в последствии алгоритм распознавания сможет определить и выдать на выходе цифровой текст.

\section{Первичная обработка изображения}

Прежде чем сегментировать текст на изображении, его нужно сперва обработать, то есть сделать пригодным для удобного использования многими алгоритмами разбиения. Способов реализации для этой задачи существует несколько, но результат должен быть один – черно-белое изображение, где текст выделен определенным цветом, к примеру черным, а фон иным. Кроме этого, удалены шумы и прочие дефекты, хотя второе можно отбрасывать на этапе поиска текста.

Первым делом нужно повысить качество изображения. Чтобы это сделать, в самом начале работы можно убрать излишние шумы и провести сглаживание. Широко применяемый для шумоподавления фильтр Гаусса отлично подойдет и в нашем случае. Если же изображение изначально было размытно, то хорошо было бы повысить его четкость, чтобы далее было проще определять строки.

После нужно привести цветное изображение к черно-белому варианту. Это можно просто сделать, используя известную формулу для каждого пикселя: 
\begin{equation}% лучше не оставлять пропущенную строку (\par) перед окружениями для избежания лишних отсупов в pdf
	\label{eq:Pixel-ch1} % eq - equations, далее название, ch поставлено для избежания дублирования
	C = 0,2989 * R + 0,5870 * G + 0,1140 * B,
\end{equation}
где С – результирующее значение, R, G, B – красная, зеленая и синяя составляющие исходного цвета, каждый параметр в пределах [0; 1]. 

Когда изображение окажется в черно-белом формате, нужно провести пороговую бинаризацию, благодаря которой черный цвет изображения будет соответствовать тексту или «мусору» на данном этапе, а белый фону. То есть уберутся шероховатости, теневые дефекты изображения. Исходя из пиков черного и белого цвета, то есть областей с самым темным цветом и самым светлым цветом можно отбросить все средние значения, округлив их по определенным параметрам, которые предварительно подбираются наиболее оптимально исходя из карты порогов. Это можно осуществить, например, методом Оцу.\cite{otzu} Но, кроме этого, если заранее определить, какая составляющая цветого оттенка преобладает больше в изображении, можно изменить константы в выше приведенной формуле. В таком случае изображение сразу станет наиболее контрастным и пороговая бинаризация обеспечит лучший результат.

После такого рода обработки можно смело составлять компоненты связности и производить отброс уже ненужной информации, как например изображения, пятна или прочие лишние элементы. Это можно сделать, используя нейронные сети, которые будут классифицировать объекты, относящиеся к тексту и нет, однако это потребует сложных операций, которые, в свою очередь, добавят дополнительные нагрузки. Помимо такого способа можно обрабатывать геометрические свойства объектов и среднюю яркость малых участков относительно соседних.

\section{Сегментация строк}

На данном этапе нужно получить отдельные изображения, в каждом из которых будет содержаться строка текста. И чтобы реализовать данную задачу, существует два основных подхода обработки: снизу вверх (bottom-up approach) и сверху вниз (top-down approach).

Идея первого заключается в обработке «самых маленьких» компонент связности, к примеру тех, что образуются при анализе каждого пикселя и его соседних элементов: если пиксель черный, и у него есть соседний элемент, который обладает белым цветом, то мы заключаем его в компонент. Причем соседство смотрится по всем направлениям, то есть все восемь пикселей вокруг текущего. Далее они все связываются, и мы можем получить отдельные слова, буквы. Объединяя последние достигаем уровня строк. Получение крупных объектов идет из объединения более мелких.
 
Идея top-down подхода заключается в противоположной идее, когда сначала идет сегментация на строки, затем сегментация из этих строк на слова, а затем уже, если это требуется, на отдельные элементы, такие как буквы.

Способов получить отдельные строки существует на данный момент несколько, некоторые из них:
\begin{itemize}
	\item Метод горизонтальной проекции. Заключается он в нахождении горизонтального профиля изображения. Сначала считаются суммы пикселей вдоль горизонтального направления. Затем ищутся локальные минимумы у такого профиля. Они же будут соответствовать интервалу между строк. В случае, если изображение строго разделено на два цвета – белый и черный, то суммы в интервалах будут значительно отличаться от тех, что находились на строках. Но такой подход достаточно хорошо анализирует изображение, если текст близко к печатному, когда все строки параллельны и имеют большой интервал между собой, тогда такие суммы будут близки к нулю. Но в общем случае при рукописном тексте интервал может быть минимальным или вовсе строки накладываются друг на друга. Кроме того, он перестает быть эффективным при значительном наклоне текста, так как профиль приобретает более-менее равномерный вид. В таком случае можно использовать модификацию данного метода – локальный горизонтальный профиль. В таком случае применяются дополнительные средства, которые позволяют в изображении даже с градациями серого отметить строки, учитывая их наклон. Но если речь идет о документах, где текст находится в различных местах, то такие подходы становятся менее эффективными. 
	\item Если допустить, что буквы пишутся не слитно, а с определенным интервалом, то существует метод диаграмм Вороного, который позволяет сегментировать текст с использованием bottop-up подхода. Все буквы разделяются между собой на участки, после чего из таких участков можно получить слова, и затем строки. Но, очевидно, что такой способ наиболее часто не применим.\cite{Yosef2009LineSF}
	\item Существует предположение, что человек при письме пишет строки по воображаемым линиям, базовым линиям (based lines), и задача состоит в том, чтобы найти их. Для этого сначала определяются значимые компоненты связности. В них ищутся все локальные экстремумы (минимумы и максимумы), и средняя разница по каждому определяет высоту компонента. После определенных дополнительных операций можно получить направление линии для каждой строки. Но такой метод подразумевает, что строки будут идти прямо, а не извилисто, потому что во втором случае такой метод не сможет аппроксимировать наши получаемые значения для нахождения прямой. Однако такой способ отлично работает и за небольшую трату времени. И его можно улучшить, получая уже не прямую, а кривую при аппроксимации. Однако на это придется потратить ресурсы.\cite{kim}
	
\end{itemize}

\section{Сегментация слов}

Сегментация слов достаточно сложный этап. Входом в алгоритм служит изображение, представляющее строку, полученное на предыдущем этапе разделения строк. В машинном тексте задача не является сложной – интервалы между словами четко определены, поскольку имеют более-менее одинаковую длину, а также существует более четкая грань в разделении слова и самого интервала. В рукописном тексте могут быть удлиненные штрихи, которые не позволяют строго разделять интервал от слова, он получается более размытым. Кроме этого, пробелы между словами не всегда гораздо длиннее, чем пробелы в словах, а расстояние между слов может варьироваться в широких пределах. Квалифицировать данный процесс достаточно тяжело. Далее пойдет речь о двух подходах для реализации данной задачи.

Первый заключается в использовании нейронной сети. В заранее подготовленной строке нужно разобрать все пустоты по двум классам: пробел между слов и пробел в самом слове. Для этого можно использовать многоуровневый персептрон. По сути, такой способ хорошо подходит для поставленной задачи.

Второй способ заключается в использовании геометрических свойствах компонент связности. Можно анализировать расстояния между связными элементами. Причем тут есть различные разновидности того, как это осуществить: можно смотреть длину между двумя крайними точками, что позволит оценивать штрихи, а можно оценивать проекцию расстояния по горизонтали, как длину интервала или брать среднее значение.

При использовании диаграмм Вороного на предыдущем этапе, разделение на слова не составляет трудностей, ибо среди ячеек будут присутствовать явные пробелы, так как в определенных буквенных ячейках будет присутствовать пустота по х, большая, чем в других.

Можно использовать еще один способ, который отлично работает при печатных текстах. Если исходное изображение не было подвергнуто на фазе предварительной обработки пороговой бинаризации, то здесь самое время ее применить. После размыть Гауссовым методом изображение вдоль горизонтального направления. Тогда компоненты букв начнут более сильно пересекаться, что позволит объединить их в одну. Или еще лучше полностью размыть изображение, чтобы слова стали буквально единым целым, а интервалы между ними были бы с наименьшей яркостью. После чего можно повторно применить упрощенную версию пороговой бинаризации и использовать способ горизонтальной проекции, но вдоль вертикального направления. Таким образом, подсчитав суммы пикселей, можно будет найти локальные минимумы, которые будут соответствовать интервалам между слов.\cite{kak}

\section{Выводы} \label{ch2:conclusion}

Важным фактором для определения и разбиения текста на изображении является его предварительная обработка. Если изображение будет содержать повреждения, лишние детали, то это существенно будет усложнять алгоритмы разбиения текста на строки, и далее на слова. Наиболее часто для таких целей используют фильтры Гаусса и пороговой бинаризации, их достаточно, чтобы текст стал контрастным на большинстве типов изображений.

Наиболее предпочтительным способом сегментации все-таки является top-down подход, так как у него больше примеров практических реализации, к ним также прилагаются различные модификации. Выявив значительные преимущества каждого, можно составить свой вариант, который будет преобладать по точности и результативности.

%% Вспомогательные команды - Additional commands
%
%\newpage % принудительное начало с новой страницы, использовать только в конце раздела
%\clearpage % осуществляется пакетом <<placeins>> в пределах секций
%\newpage\leavevmode\thispagestyle{empty}\newpage % 100 % начало новой страницы	         	 % Глава 2
\chapter{Распознавание} \label{ch3}

% не рекомендуется использовать отдельную section <<введение>> после лета 2020 года
%\section{Введение} \label{ch3:intro}

Данный этап является самым сложным среди остальных, так как это и есть стадия распознавания рукописных символов. 
	
\section{Нормализация текста} \label{ch3:sec1}

На данном этапе на входе алгоритма организовываются отдельные изображения слов или объектов слов. И перед тем, как начинать распознавать эти изображения, сперва лучше их привести к виду, наиболее пригодному для этой самой обработки. Потому что из-за многочисленных вариантов написания, начертаний, слова могут быть написаны под различными углами, что затрудняет способы распознавания. И задача состоит в том, чтобы привести их к виду, более-менее параллельному горизонтальному направлению, причем не теряя данных, то есть изображение не должно исковеркаться или как-то сжаться после этапа нормализации.

Для реализации существует несколько подходов и многие из них ориентируются, собственно, на углы наклона текста относительно некоторых прямых, а именно угол наклона относительно горизонтальной прямой и угол наклона относительно вертикальной. Но второй вариант подходит для тех элементов, что должны быть вертикальными.

На самом деле задача состоит в выборе правильного наклона и, возможно, перспективы, и такие средства уже давно реализованы в той же программной среде Photoshop.

Основная проблема заключается в определении угла, при котором текст станет выровненным по горизонтали. Для ее решения можно использовать горизонтальные профили. В таком случае нужно искать максимум профиля при изменении угла, тогда слово максимально выравнено. Если оно каким-то образом искривлено или наклонено, тогда профиль будет распределен, что не является результативным свойством.

Но также важен факт того, что потеря информации должна быть сведена к минимуму. И, какой способ бы не использовался, на изображении могут «исчезнуть» некоторые пиксели, из-за чего компоненты связности могут нарушаться и сами буквы или целые слова затем труднее будет распознавать. Для исправления такой ситуации можно использовать анализ исходного варианта, и добавление новых пикселей, а затем сглаживание, к примеру билинейное. Или можно сразу применить сглаживание, но тогда результат все равно может быть несколько иным.


\section{Распознавание слов} \label{ch3:sec2}

Есть два основных подхода для реализации распознавания слов: сначала разбить на отдельные символы, распознать их, а затем склеить в слово, и анализировать слово целиком.

Первый и второй подходы могут работать вместе, когда сначала слово разбивается на символы поочередно, начиная с первой, затем с каждой последующей буквой формируется наиболее вероятное слово. В таком случае избегается возможность ошибки при распознавании. Но поговорим про каждый способ сначала в отдельности.

Задача разбить слово на символы является сама по себе сложной. Тем более если речь идет о рукописном тексте. И на данный момент нет эффективного решения. Интервалы между буквами не всегда очевидны, в особенности если они связаны между собой, а не пишутся раздельно. Для последнего случая, отметим, есть те же самые диаграммы Вороного. Но в общем случае разбить на символы может быть и вовсе невыполнимо, тогда можно разбивать на определенные участки. Можно применить слабо работающий, но в перспективе улучшенный способ в две стадии. На первой стадии сперва нужно разбить на вероятные интервалы. Для этого составляются вертикальные профили изображения и ищутся локальные минимумы. Тут во внимание нужно взять величину отношения высоты символа к его ширине. В среднем эта величина равна 0,3. То есть мы можем определить интервалы поиска локальных минимумов, посмотрев высоту слова, и подсчитав среднюю ширину символа для конкретного слова. Так же следует учитывать среднюю яркость целого участка. Вторая стадия исключает ложные разделения и определяет явные возможные варианты.

Также можно разбить на достаточно малые интервалы слово, и идти вдоль изображения, разбирая каждый участок. Составляя вероятные карты можно попробовать определить, какие символы встречаются, отсекать их, и искать дальше.

Хорошим вариантом является применение нейронной сети. Причем можно предварительно поделить изображение на участки равной длины исходя из соображения подсчитываемой ширины символа. И тогда использовать сеть по типу персептрона. Если вероятности для каждого класса примерно равные, значит сеть не смогла точно определить символы, можно снова сегментировать изображение. Либо использовать сверточную сеть, которую наиболее часто используют в таких задачах, потому что она как раз создана для распознавания и определения видимых признаков. Тогда так же проходить по изображению вдоль горизонтального направления и определять классы встречаемых участков. Или же делить на более мелкие участки, и использовать многоуровневую сверточную нейронную сеть.

Если говорить о подходе определения слова целиком, то сперва лучше определить примерный контекст текста, чтобы завести базу данных возможных слов. Или же использовать полную БД всех слов. Затем задействовать для алгоритма обученную нейронную сеть, которая будет получать возможные варианты слова. Сеть не обязана получить сразу верное значение. Сохраняются вероятности для каждого случая.

Алгоритм последовательного считывания символов предполагает, что каждый последующий будет определяться на основе составляющегося слова. Предварительно можно построить префиксные деревья. Тогда определив текущий символ, по дереву достаточно быстро и просто определяются возможные следующие варианты символов. Таким образом сразу исключаются возможные ошибки или же можно определить, является ли следующий символ ошибкой в исходном слове. Но тогда алгоритм ломается после такого случая. Если же не учитывать параллельную обработку ошибок, тогда следующий символ ищется заново с корректировками. 

\section{Улучшение результатов}  \label{ch3:sec3}

На этапе использования нейронной сети для определения слова целиком был получен массив вероятностей подходящих вариантов для каждого слова. Достаточно легко определить границы предложений по знакам, которые не так сложно ищутся. Тогда, используя семантику языка, возможно определить наиболее подходящий вариант следующего после текущего слова. И, разумеется, обрабатывая одновременно этот фактор и фактор массива вероятностей, получается точный результат. 

Если распознавание шло по сегментации слова на символы или последовательности символов, то нужно использовать словарь, именно печатный словарь слов. По нему искать наиболее подходящее слово и заменять результат на него. В случае нескольких возможных совпадений использовать предыдущий подход.








%% Вспомогательные команды - Additional commands
%
%\newpage % принудительное начало с новой страницы, использовать только в конце раздела
%\clearpage % осуществляется пакетом <<placeins>> в пределах секций
%\newpage\leavevmode\thispagestyle{empty}\newpage % 100 % начало новой страницы           	 % Глава 3
        	 % Глава 3
%\ContinueChapterEnd % завершить размещение глав <<подряд>>
%% Завершение основной части

\chapter*{Заключение} \label{ch-conclusion}
\addcontentsline{toc}{chapter}{Заключение}	% в оглавление 

Проведя исследование различных алгоритмов и их результатов, мы пришли к выводу, что на данный момент нет эффективного, как такового. Каждый алгоритм получает лишь примерный результат и точность может достигать до 85-90\%. 

Однако совмещенное использование каждого из вариантов для определенного этапа позволит повысить точность, влияя также на точность последующих этапов. 

Использование нейронных сетей также значительно повышают точность. Их развитие в современном мире позволит достичь лучших результатов распознавания, что повлечет за собой создание прикладных программ для проверки учебных работ.
        	 % Заключение

%% Наличие следующих перечней не исключает расшифровку сокращения и условного обозначения при первом упоминании в тексте!
%\chapter*{Список сокращений и условных обозначений}             % Заголовок
\addcontentsline{toc}{chapter}{Список сокращений и условных обозначений}  % Добавляем его в оглавление
\noindent
\addtocounter{table}{-1}% Нужно откатить на единицу счетчик номеров таблиц, так как следующая таблица сделана для удобства представления информации по ГОСТ
%\begin{longtabu} to \dimexpr \textwidth-5\tabcolsep {r X}
\begin{longtabu} to \textwidth {r X} % Таблицу не прорисовываем!
% Жирное начертание для математических символов может иметь
% дополнительный смысл, поэтому они приводятся как в тексте
% диссертации
\textbf{DOI} & Digital Object Identifier. \\
\textbf{WoS} & Web of Science. \\
\textbf{ВКР}  & Выпускная квалификационная работа. \\
\textbf{ТГ-объект}  & Текстово-графический объект. \\
%$\begin{rcases}
%a_n\\
%b_n
%\end{rcases}$  & 
%\begin{minipage}{\linewidth}
%Коэффициенты разложения Ми в дальнем поле, соответствующие
%электрическим и магнитным мультиполям.
%\end{minipage}
%\\
%${\boldsymbol{\hat{\mathrm e}}}$ & Единичный вектор. \\
%$E_0$ & Амплитуда падающего поля.\\
%$\begin{rcases}
%a_n\\
%b_n
%\end{rcases}$  & 
%Коэффициенты разложения Ми в дальнем поле соответствующие
%электрическим и магнитным мультиполям ещё раз, но без окружения
%minipage нет вертикального выравнивания по центру.
%\\
%$j$ & Тип функции Бесселя.\\
%$k$ & Волновой вектор падающей волны.\\
%
%$\begin{rcases}
%a_n\\
%b_n
%\end{rcases}$  & 
%\begin{minipage}{\linewidth}
%\vspace{0.7em}
%Коэффициенты разложения Ми в дальнем поле соответствующие
%электрическим и магнитным мультиполям, теперь окружение minipage есть
%и добавленно много текста, так что описание группы условных
%обозначений значительно превысило высоту этой группы... Для отбивки
%пришлось добавить дополнительные отступы.
%\vspace{0.5em}
%\end{minipage}
%\\
%$L$ & Общее число слоёв.\\
%$l$ & Номер слоя внутри стратифицированной сферы.\\
%$\lambda$ & Длина волны электромагнитного излучения
%в вакууме.\\
%$n$ & Порядок мультиполя.\\
%$\begin{rcases}
%{\mathbf{N}}_{e1n}^{(j)}&{\mathbf{N}}_{o1n}^{(j)}\\
%{\mathbf{M}_{o1n}^{(j)}}&{\mathbf{M}_{e1n}^{(j)}}
%\end{rcases}$  & Сферические векторные гармоники.\\
%$\mu$  & Магнитная проницаемость в вакууме.\\
%$r,\theta,\phi$ & Полярные координаты.\\
%$\omega$ & Частота падающей волны.\\
%
%  \textbf{BEM} & Boundary element method, метод граничных элементов.\\
%  \textbf{CST MWS} & Computer Simulation Technology Microwave Studio.
\end{longtabu}
		         % Необязательная рубрика! Список сокращений и условных обозначений

%\chapter*{Словарь терминов}             % Заголовок
\addcontentsline{toc}{chapter}{Словарь терминов}  % Добавляем его в оглавление

\textbf{TeX} --- язык вёрстки текста и издательская система, разработанные Дональдом Кнутом.

\textbf{LaTeX} --- язык вёрстки текста и издательская система, разработанные Лэсли Лампортом как надстройка над TeX.

    		 % Необязательная рубрика! Словарь терминов
% По порядку после Списка сокращений и условных обозначений, если есть.	


%%% Не мянять - Do not modify
%%
%%
\clearpage                                  % В том числе гарантирует, что список литературы в оглавлении будет с правильным номером страницы
%\hypersetup{ urlcolor=black }               % Ссылки делаем чёрными
%\providecommand*{\BibDash}{}                % В стилях ugost2008 отключаем использование тире как разделителя 
\urlstyle{rm}                               % ссылки URL обычным шрифтом
%\ifdefmacro{\microtypesetup}{\microtypesetup{protrusion=false}}{} % не рекомендуется применять пакет микротипографики к автоматически генерируемому списку литературы
%\newcommand{\fullbibtitle}{Список литературы} % (ГОСТ Р 7.0.11-2011, 4)
%\insertbibliofull  
%\noindent
%\begin{group}
\chapter*{Список использованных источников}	
\label{references}
\addcontentsline{toc}{chapter}{Список использованных источников}	% в оглавление 
\printbibliography[env=SSTfirst]                         % Подключаем Bib-базы
%\ifdefmacro{\microtypesetup}{\microtypesetup{protrusion=true}}{}
%\urlstyle{tt}                               % возвращаем установки шрифта ссылок URL
%\hypersetup{ urlcolor={urlcolor} }          % Восстанавливаем цвет ссылок

%\printbibliography
%\urlstyle{rm}                               % ссылки URL обычным шрифтом
%\ifdefmacro{\microtypesetup}{\microtypesetup{protrusion=false}}{} % не рекомендуется применять пакет микротипографики к автоматически генерируемому списку литературы
%\insertbibliofull                           % Подключаем Bib-базы
%\ifdefmacro{\microtypesetup}{\microtypesetup{protrusion=true}}{}
%\urlstyle{tt}                               % возвращаем установки шрифта ссылок URL
		     % Список литературы

% Здесь можно поместить список иллюстративного материала

\appendix % не редактировать / keep unmodified


%\chapter{Краткие инструкции по настройке издательской системы \LaTeX}\label{appendix-MikTeX-TexStudio}							% Заголовок
%\addcontentsline{toc}{chapter}{Second call for chapters to participate in the book Machine learning in analysis of biomedical and socio-economic data}	% Добавляем его в оглавление

В SPbPU-BCI-template {\itshape автоматически выставляются необходимые настройки и в исходном тексте шаблона приведены примеры оформления текстово-графических объектов}, поэтому авторам достаточно заполнить имеющийся шаблон текстом главы (статьи), не вдаваясь в детали оформления, описанные далее. Возможный <<быстрый старт>> оформления главы (статьи) под Windows следующий\footnote{Внимание! Пример оформления подстрочной ссылки (сноски).}:

\begin{enumerate}
	\item Установка полной версии MikTeX  \cite{latex-miktex}.  В процессе установки лучше выставить параметр доустановки пакетов <<на лету>>.
	
	\item Установка TexStudio \cite{latex-texstudio}.
	
%		\item установка шрифтов PSCyr для работы с TimesNew\-Roman\-PSMT  	\href{https://github.com/AndreyAkinshin/Russian-Phd-LaTeX-Dissertation-Template/blob/master/PSCyr/Windows.md}{по данной инструкции}. В итоговом документе будет, скорее всего, использован Newton.
	
%	\item Переименование следующих файлов, где вместо \texttt{AuthorsSur\-names} необходимо подставить фамилии авторов (можно сокращать до первых четырех букв): 
%	
%	\begin{enumerate}
%		\item Основной файл \texttt{Book\_title\_ch\_Authors\-Sur\-names.tex}.
%		\item Библиография \texttt{biblio\textbackslash{}Book\_title\_bib\_Authors\-Sur\-na\-mes\-.bib}.
%		\item Пользовательские настройки (при необходимости), \texttt{common\textbackslash{}Book\_\-tit\-le\_ext\_Authors\-Sur\-names.tex}. 
%	\end{enumerate}
%	
%	\item После открытия основного файла \texttt{Book\_title\_ch\_Authors\-Sur\-names.tex} (с новым названием)   переименовать названия по аналогии в следующих командах \texttt{\textbackslash{}input\{\}}:
%	
%	\begin{enumerate}
%		\item \texttt{biblio/Book\_title\_bib\_Authors\-Sur\-names.bib},
%		\item \texttt{common/Book\_title\_ext\_Authors\-Sur\-names.tex (при необходимости) }.
%	\end{enumerate}
%	
	
	\item Запуск TexStudio и компиляция \verb|my_chapter.tex| с помощью команды <<Build\&View>> (например, с помощью двойной зелёной стрелки в верхней панели). {\itshape Иногда, для достижения нужного результата необходимо несколько раз скомпилировать документ.}
	
	\item В случае, если не отобразилась библиография, можно
	
	\begin{itemize}
		\item воспользоваться командой Tools $\to$ Commands $\to$ Biber, затем запустив Build\&View;
		
		\item настроить автоматическое включение библиографии в настройках Options $\to$ Configure TexStudio $\to$ Build $\to$  Build\&View (оставить по умолчанию, если сборка происходит слишком долго): \texttt{txs:///pdflatex | txs:///biber | txs:///pdflatex | txs:///pdflatex | txs:///\-view-pdf}.
	\end{itemize}
	
\end{enumerate}

В случае возникновения ошибок, попробуйте скомпилировать документ до последних действий или внимательно ознакомьтесь с описанием проблемы в log-файле. Бывает полезным переход (по подсказке TexStudio) в нужную строку в pdf-файле или запрос с текстом ошибке в поисковиках. Наиболее вероятной проблемой при первой компиляции может быть отсутствие какого-либо установленного пакета \LaTeX. 

В случае корректной работы настройки <<установка на лету>> все дополнительные пакеты будут скачиваться и устанавливаться в автоматическом режиме. Если доустановка пакетов осуществляется медленно (несколько пакетов за один запуск компилятора), то можно попробовать установить их в ручном режиме следующим образом:

\begin{enumerate}[1.]
	\item Запустите программу: меню $\to$ все программы $\to$ MikTeX $\to$ Maintenance (Admin) $\to$ MiKTeX Package Manager (Admin).
	\item Пользуясь поиском, убедитесь, что нужный пакет присутствует, но не установлен (если пакет отсутствует воспользуйтесь сначала MiKTeX Update (Admin)).
	\item Выделив строку с пакетом (возможно выбрать несколько или вообще все неустановленные пакеты), выполните установку Tools $\to$ Install или с помощью контекстного меню.
	\item После завершения установки запустите программу MiKTeX Settings (Admin).
	\item Обновите базу данных имен файлов Refresh FNDB.
\end{enumerate}


Для проверки текста статьи на русском языке полезно также воспользоваться настройками Options $\to$ Configure TexStudio $\to$ Language Checking $\to$  Default Language. Если русский язык <<ru\_RU>> не будет доступен в меню выбора, то необходимо вначале выполнить Import Dictionary, скачав из интернета любой русскоязычный словарь. 


%\chapter{\normalfont\normalsize{}Часто задаваемые вопросы (FAQ)}\label{Appendix-FAQ}							% Заголовок
%%\addcontentsline{toc}{chapter}{Second call for chapters to participate in the book Machine learning in analysis of biomedical and socio-economic data}	% Добавляем его в оглавление


Далее приведены формулы \eqref{eq:Pi-app2}, \eqref{eq:Pi-app2-},  \firef{fig:spbpu_hydrotower-app2}, \firef{fig:spbpu_hydrotower-app2-}, \taref{tab:ToyCompare-app2}, \taref{tab:ToyCompare-app2-}.


\begin{equation}% лучше не оставлять пропущенную строку (\par) перед окружениями для избежания лишних отсупов в pdf
\label{eq:Pi-app2-} % eq - equations, далее название, ch поставлено для избежания дублирования
\pi \approx 3,141.
\end{equation}

%
\begin{figure}[ht!] 
	\center
	\includegraphics [scale=0.27] {my_folder/images//spbpu_hydrotower}
	\caption{Вид на гидробашню СПбПУ \cite{spbpu-gallery}} 
	\label{fig:spbpu_hydrotower-app2-}  
\end{figure}

\begin{table} [htbp]% Пример оформления таблицы
	\centering\small
	\caption{Представление данных для сквозного примера по ВКР \cite{Peskov2004}}%
	\label{tab:ToyCompare-app2-}		
	\begin{tabular}{|l|l|l|l|l|l|}
		\hline
		$G$&$m_1$&$m_2$&$m_3$&$m_4$&$K$\\
		\hline
		$g_1$&0&1&1&0&1\\ \hline
		$g_2$&1&2&0&1&1\\ \hline
		$g_3$&0&1&0&1&1\\ \hline
		$g_4$&1&2&1&0&2\\ \hline
		$g_5$&1&1&0&1&2\\ \hline
		$g_6$&1&1&1&2&2\\ \hline		
	\end{tabular}	
	\normalsize% возвращаем шрифт к нормальному
\end{table}




\section{Параграф приложения}\label{app-2-1}							


\subsection{Название подпараграфа} \label{ch2:subsec-title-abbr} %название по-русски


Название подпараграфа оформляется с помощью команды  \texttt{\textbackslash{}subsection\{...\}}.

Использование подподпараграфов в основной части крайне не рекомендуется.
\subsubsection{Название подподпараграфа}\label{ch2:subsubsec-title-abbr} %название по-русски

\begin{equation}% лучше не оставлять пропущенную строку (\par) перед окружениями для избежания лишних отсупов в pdf
\label{eq:Pi-app2} % eq - equations, далее название, ch поставлено для избежания дублирования
\pi \approx 3,141.
\end{equation}
%
%
\begin{figure}[ht!] 
	\center
	\includegraphics [scale=0.27] {my_folder/images//spbpu_hydrotower}
	\caption{Вид на гидробашню СПбПУ \cite{spbpu-gallery}} 
	\label{fig:spbpu_hydrotower-app2}  
\end{figure}
%




\begin{table}[t!]% Пример оформления таблицы
	\centering\small
	\caption{Представление данных для сквозного примера по ВКР \cite{Peskov2004}}%
	\label{tab:ToyCompare-app2}		
	\begin{tabular}{|l|l|l|l|l|l|}
		\hline
		$G$&$m_1$&$m_2$&$m_3$&$m_4$&$K$\\
		\hline
		$g_1$&0&1&1&0&1\\ \hline
		$g_2$&1&2&0&1&1\\ \hline
		$g_3$&0&1&0&1&1\\ \hline
		$g_4$&1&2&1&0&2\\ \hline
		$g_5$&1&1&0&1&2\\ \hline
		$g_6$&1&1&1&2&2\\ \hline		
	\end{tabular}	
	\normalsize% возвращаем шрифт к нормальному
\end{table}


%% В случае, когда таблица (рисунок) размещаются на последней странице, для переноса названия приложения на новую строку используем:
\NewPage % начать новое приложение с новой страницы 			     % Приложение 1

%\chapter{Некоторые дополнительные примеры}\label{appendix-extra-examples}							% 

В приложении\footnote{Внимание! Пример оформления подстрочной ссылки (сноски).} приведены формулы \eqref{eq:Pi-app}, \eqref{eq:Pi-app-}, \firef{fig:spbpu_hydrotower-app}, \firef{fig:spbpu_hydrotower-app-}, \taref{tab:ToyCompare-app}, \taref{tab:ToyCompare-app-}


\begin{equation}% лучше не оставлять пропущенную строку (\par) перед окружениями для избежания лишних отсупов в pdf
\label{eq:Pi-app-} % eq - equations, далее название, ch поставлено для избежания дублирования
\pi \approx 3,141.
\end{equation}
%
%
\begin{figure}[ht!] 
	\center
	\includegraphics [scale=0.27] {my_folder/images//spbpu_hydrotower}
	\caption{Вид на гидробашню СПбПУ \cite{spbpu-gallery}} 
	\label{fig:spbpu_hydrotower-app-}  
\end{figure}

\begin{table} [htbp]% Пример оформления таблицы
	\centering\small
	\caption{Представление данных для сквозного примера по ВКР \cite{Peskov2004}}%
	\label{tab:ToyCompare-app-}		
	\begin{tabular}{|l|l|l|l|l|l|}
		\hline
		$G$&$m_1$&$m_2$&$m_3$&$m_4$&$K$\\
		\hline
		$g_1$&0&1&1&0&1\\ \hline
		$g_2$&1&2&0&1&1\\ \hline
		$g_3$&0&1&0&1&1\\ \hline
		$g_4$&1&2&1&0&2\\ \hline
		$g_5$&1&1&0&1&2\\ \hline
		$g_6$&1&1&1&2&2\\ \hline		
	\end{tabular}	
	\normalsize% возвращаем шрифт к нормальному
\end{table}




\section{Подраздел приложения}\label{app-2-1}							


\begin{equation}% лучше не оставлять пропущенную строку (\par) перед окружениями для избежания лишних отсупов в pdf
\label{eq:Pi-app} % eq - equations, далее название, ch поставлено для избежания дублирования
\pi \approx 3,141.
\end{equation}
%
%
\begin{figure}[ht!] 
	\center
	\includegraphics [scale=0.27] {my_folder/images//spbpu_hydrotower}
	\caption{Вид на гидробашню СПбПУ \cite{spbpu-gallery}} 
	\label{fig:spbpu_hydrotower-app}  
\end{figure}

\begin{table} [htbp]% Пример оформления таблицы
	\centering\small
	\caption{Представление данных для сквозного примера по ВКР \cite{Peskov2004}}%
	\label{tab:ToyCompare-app}		
	\begin{tabular}{|l|l|l|l|l|l|}
		\hline
		$G$&$m_1$&$m_2$&$m_3$&$m_4$&$K$\\
		\hline
		$g_1$&0&1&1&0&1\\ \hline
		$g_2$&1&2&0&1&1\\ \hline
		$g_3$&0&1&0&1&1\\ \hline
		$g_4$&1&2&1&0&2\\ \hline
		$g_5$&1&1&0&1&2\\ \hline
		$g_6$&1&1&1&2&2\\ \hline		
	\end{tabular}	
	\normalsize% возвращаем шрифт к нормальному
\end{table}

			 	 % Приложение 2


\end{document} % конец документа


%%% Удачной защиты ВКР! - Good luck on the thesis defense!
%%
%%% Поддержать проект
%%
%% Запросы на добавление / изменение просим писать на следующей странице:
%% https://github.com/ParkhomenkoV/SPbPU-student-thesis-template/issues
%%
%% Список пожеланий в файле шаблона <<TO-DO-list.tex>>
%%
%% Благодарности просим указывать в виде 
%%
%% 1. Добавление <<Звезды>> проекту https://github.com/ParkhomenkoV/SPbPU-student-thesis-template/stargazers
%%
%% 2. Добавления <<Сердечка>> и репоста проекта в социальных сетях:
%%		https://vk.com/latex_polytech 
%%		https://www.fb.com/groups/latex.polytech
%%

%%% Support project
%%
%% Requests on adding / modifications is better to be publishen on the following web-page:
%% https://github.com/ParkhomenkoV/SPbPU-student-thesis-template/issues
%%
%% Wishlist is in the template's file called <<TO-DO-list.tex>>
%%
%% Acknowledgements are better to be done in the form of 
%%
%% 1. Adding <<Star>> to the project https://github.com/ParkhomenkoV/SPbPU-student-thesis-template/stargazers
%%
%% 2. Adding <<Likes>> and Project repost in the social networks:
%%		https://vk.com/latex_polytech 
%%		https://www.fb.com/groups/latex.polytech
%% 

% Check list при передаче ВКР:
% - Количество страниц в Задании 2. Если нет, то комментирование последней строки в my_task.tex
% - Зачистка всех вспомогательных файлов (Clear auxilary files) и компиляция ВКР не менее 3х раз